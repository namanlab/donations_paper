% ==============================================================================
% Crisis, Compassion, and Capital: The Economics of Global Charitable Giving
% Evidence from GlobalGiving's Project-Level Donation Data
% ==============================================================================
% Professional Economics Journal Format (AER/QJE/Econometrica Style)
% ==============================================================================

\documentclass[11pt,letterpaper]{article}

% ==============================================================================
% PACKAGES
% ==============================================================================

% Page layout - AER style margins
\usepackage[top=1.25in, bottom=1.25in, left=1.25in, right=1.25in]{geometry}
\usepackage{setspace}
\onehalfspacing

% Fonts
\usepackage[T1]{fontenc}
\usepackage[utf8]{inputenc}
\usepackage{lmodern}
\usepackage{microtype}

% Mathematics
\usepackage{amsmath}
\usepackage{amssymb}
\usepackage{amsthm}
\usepackage{mathtools}
\usepackage{bm}

% Graphics and tables
\usepackage{parskip}
\usepackage{graphicx}
\usepackage{booktabs}
\usepackage{tabularx}
\usepackage{multirow}
\usepackage{float}
\usepackage{rotating}
\usepackage{array}
\usepackage{threeparttable}
\usepackage{longtable}
\usepackage{pdflscape}
\usepackage{dcolumn}

% References
\usepackage[round,sort&compress,authoryear]{natbib}
\bibliographystyle{plainnat}

% Hyperlinks
\usepackage[colorlinks=true,linkcolor=blue!70!black,citecolor=blue!70!black,urlcolor=blue!70!black]{hyperref}

% Other
\usepackage{caption}
\usepackage{subcaption}
\usepackage{appendix}
\usepackage{enumitem}
\usepackage{xcolor}
\usepackage{soul}
\usepackage{titlesec}

% ==============================================================================
% CUSTOM FORMATTING
% ==============================================================================

% Section formatting
\titleformat{\section}{\large\bfseries}{\thesection.}{0.5em}{}
\titleformat{\subsection}{\normalsize\bfseries}{\thesubsection.}{0.5em}{}
\titleformat{\subsubsection}{\normalsize\itshape}{\thesubsubsection.}{0.5em}{}

% Paragraph spacing
\setlength{\parskip}{0.5em}
\setlength{\parindent}{1.5em}

% Custom commands
\newcommand{\E}{\mathbb{E}}
\newcommand{\Var}{\text{Var}}
\newcommand{\Cov}{\text{Cov}}
\newcommand{\plim}{\text{plim}}
\newcommand{\N}{\mathcal{N}}
\newcommand{\R}{\mathbb{R}}
\newcommand{\indep}{\perp\!\!\!\perp}
\DeclareMathOperator*{\argmax}{arg\,max}
\DeclareMathOperator*{\argmin}{arg\,min}

% Theorem environments
\theoremstyle{plain}
\newtheorem{theorem}{Theorem}
\newtheorem{lemma}[theorem]{Lemma}
\newtheorem{proposition}[theorem]{Proposition}
\newtheorem{corollary}[theorem]{Corollary}

\theoremstyle{definition}
\newtheorem{definition}{Definition}
\newtheorem{assumption}{Assumption}
\newtheorem{hypothesis}{Hypothesis}
\newtheorem{example}{Example}

\theoremstyle{remark}
\newtheorem{remark}{Remark}

% Figure notes
\newenvironment{figurenotes}{\par\small\vspace{0.3em}\noindent\textit{Notes:} }{\par}
% tablenotes defined by threeparttable package
\newenvironment{customnotes}{\par\small\vspace{0.3em}\noindent\textit{Notes:} }{\par}

% Column type for regression tables
\newcolumntype{d}[1]{D{.}{.}{#1}}

\setlength{\parindent}{0pt}
% ==============================================================================
% TITLE PAGE
% ==============================================================================

\begin{document}



\begin{titlepage}
\centering
\vspace*{2cm}

{\LARGE\bfseries Crisis, Compassion, and Capital:\\ The Economics of Global Charitable Giving\par}

\vspace{0.5cm}
{\large Evidence from GlobalGiving's Project-Level Donation Data\par}

\vspace{1cm}

{\large Agrawal Naman\par}

\vspace{1cm}

{\large January 2026\par}

\vspace{1cm}

\begin{abstract}
\noindent\textbf{Abstract:} This paper provides comprehensive empirical evidence on the determinants of charitable giving using detailed project-level data from GlobalGiving, one of the world's largest online crowdfunding platforms for nonprofits. Our dataset spans nearly 50,000 projects across 201 countries and multiple thematic sectors over two decades (2002--2025), representing over \$575 million in charitable donations. We investigate three fundamental questions at the intersection of behavioral economics and development finance: (1) How do geopolitical crises affect the spatial and temporal allocation of donor capital? (2) What project-level and organizational characteristics predict fundraising success in competitive charitable marketplaces? (3) Are there systematic disparities in funding flows across regions and causes, and if so, what mechanisms drive these inequalities?

Our identification strategy exploits the exogenous timing of the Russian invasion of Ukraine on February 24, 2022, as a natural experiment. A key feature of our setting is that Ukraine-related charitable projects were essentially nonexistent on the platform before February 2022, with funding near zero. We use an aggregate monthly approach, comparing total funding (and donations, and project counts) for Ukraine-related versus non-Ukraine projects before and after the invasion. Pre-invasion months with zero Ukraine activity are included as zeros rather than excluded, providing a clean baseline for identification. Using difference-in-differences and event study designs, we estimate that the invasion increased log funding to Ukraine-related projects by 5.4 log points ($p < 0.001$) and donations by approximately 26,000 per month ($p < 0.01$) relative to non-Ukraine projects. Average monthly Ukraine funding jumped from approximately \$13,000 pre-invasion to over \$2.2 million post-invasion. Importantly, we document evidence of both \textit{additionality}---the crisis attracted new donors to the platform---and \textit{substitution}, where existing donor attention was redirected from non-Ukraine causes to Ukraine-related projects.

Beyond crisis effects, we examine the mechanisms through which certain projects attract disproportionate funding. Text analysis reveals that narrative framing significantly influences donor behavior: projects employing urgency-related keywords (``emergency,'' ``critical,'' ``immediate'') receive 35--40\% more funding, while life-saving language (``save lives'') is associated with 89\% higher contributions, controlling for observable project characteristics. Critically, we decompose these effects into extensive margin (new donor acquisition) versus intensive margin (gift size among existing donors) components, finding that most narrative effects operate almost entirely through the extensive margin. We also document substantial geographic inequalities: North American projects receive 2--3 times more funding than African projects after controlling for goal size, theme, and organizational characteristics. Quantile regression analysis reveals that goal elasticity varies systematically across the funding distribution, with stronger effects at lower quantiles, suggesting that goal-setting strategies matter most for marginal projects. Our findings contribute to the economics of charitable giving, platform design, and development finance, with implications for nonprofit strategy, donor behavior, and the efficient allocation of humanitarian resources.

\vspace{0.5cm}
\noindent\textbf{JEL Classification:} D64, D91, H41, L31, O19, F35

\vspace{0.3cm}
\noindent\textbf{Keywords:} Charitable giving, crowdfunding, crisis response, difference-in-differences, warm-glow preferences, platform economics, humanitarian aid, attention allocation
\end{abstract}

\end{titlepage}

% % ==============================================================================
% % TABLE OF CONTENTS
% % ==============================================================================

% \newpage
% \tableofcontents
% \newpage

% ==============================================================================
% 1. INTRODUCTION
% ==============================================================================

\section{Introduction}
\label{sec:intro}

The allocation of charitable resources represents one of the most consequential economic decisions made by households worldwide, with profound implications for social welfare, development outcomes, and the provision of public goods. In 2022 alone, American individuals donated over \$557.16 billion to charitable causes \citep{givingusa2023}. Globally, private charitable flows now constitute a significant share of development finance, often exceeding official development assistance in magnitude while arguably surpassing it in flexibility and responsiveness to emerging needs. Yet despite the enormous scale of philanthropic activity and its importance for addressing global challenges ranging from poverty alleviation to disaster response, our understanding of how donors allocate resources across competing causes remains surprisingly incomplete.

The economics of charitable giving presents a fundamental puzzle for standard economic theory that has animated scholarly debate for decades. Under the canonical assumption of pure self-interest, voluntary contributions to public goods should be negligible; the well-known free-rider problem, formalized by \citet{samuelson1954}, suggests that rational agents will undercontribute to collective benefits, instead relying on others' provision. Yet we observe substantial private provision of public goods across virtually all societies and historical periods. The resolution of this puzzle has proven remarkably productive, yielding foundational insights about altruism, social preferences, and the psychology of giving, with early theoretical work by \citet{warr1982} and \citet{roberts1984} established that under pure altruism, government provision should perfectly crowd out private giving. This prediction has been consistently rejected by empirical evidence \citep{kingma1989, payne1998}. The seminal contribution of \citet{andreoni1990} introduced the concept of ``warm glow,'' the idea that donors derive utility not only from the outcomes their gifts produce but also from the act of giving itself. This framework of ``impure altruism'' has proven remarkably productive, explaining why donors give even when their marginal impact on public good provision is negligible, why they prefer direct giving over taxation, and why they respond to matching grants in ways inconsistent with pure outcome-orientation. Subsequent work has extended this framework to incorporate prestige motives \citep{harbaugh1998}, signaling concerns \citep{benabou2006}, and leadership effects \citep{andreoni2006}. Field experiments by \citet{karlan2007} and \citet{dellavigna2012} have decomposed giving into altruism, warm glow, and social pressure components, while \citet{small2007} and \citet{slovic2007} have documented the psychological mechanisms, particularly the identifiable victim effect and psychic numbing, that shape donor responses to humanitarian appeals.

This paper contributes to this rich tradition while addressing three significant gaps in the existing literature. First, the rise of online crowdfunding platforms has fundamentally transformed the charitable marketplace in ways that existing research has only begun to examine. Platforms like GlobalGiving, Kiva, GoFundMe, and DonorsChoose have democratized access to donor capital, enabling small organizations and grassroots initiatives to reach global audiences without traditional intermediaries. These platforms generate unprecedented data on donor behavior, project characteristics, and funding outcomes at a granularity impossible in earlier studies relying on aggregate giving data or tax records. Understanding how these platforms allocate attention and resources, and whether their design inadvertently exacerbates inequalities, has direct implications for humanitarian efficiency and global development. Prior crowdfunding research, including work by \citet{mollick2014}, \citet{meer2014}, and \citet{agrawal2015}, has examined platform dynamics and geographic biases, but largely outside the charitable context or focused on single-country settings.

Second, major geopolitical crises create sudden, large shifts in donor attention that reveal fundamental features of charitable preference and constraint. The Russian invasion of Ukraine on February 24, 2022, triggered the largest refugee crisis in Europe since World War II, displacing over 5 million Ukrainians within months \citep{unhcr2022}. The humanitarian response was extraordinary: charitable donations to Ukraine-related causes surged worldwide, with platforms like GlobalGiving experiencing unprecedented demand. Yet this surge raises critical questions with both scholarly and practical import. Did the Ukraine crisis attract \textit{new} resources to charitable giving (additionality), or did it primarily \textit{redirect} existing donor attention away from other causes (substitution)? The distinction matters profoundly for understanding the total welfare effects of crisis response and for nonprofit organizations whose work may be crowded out by high-profile emergencies. While \citet{eisensee2007} demonstrated that media coverage drives disaster relief allocation, and several studies have examined individual disaster events, the Ukraine invasion offers a uniquely clean natural experiment: its timing was genuinely unexpected, it was global in media coverage, and it affected a well-identified set of projects on a platform with rich pre-event data.

Third, despite decades of development efforts and growing awareness of global inequality, substantial disparities persist in access to charitable funding. Projects in wealthy countries consistently outperform those in developing regions, even after controlling for observable project characteristics. Whether these disparities reflect deep-seated donor preferences, information asymmetries that create perceived risk, differences in organizational capacity and marketing sophistication, or platform design features that inadvertently favor certain regions remains unclear. Documenting these patterns rigorously and understanding their mechanisms is essential for designing more equitable charitable systems that channel resources toward the greatest need rather than the greatest familiarity.

\medskip
\noindent\textbf{This Paper's Contributions.} Against this backdrop, this paper makes three distinct contributions to the literature on charitable giving, platform economics, and development finance.

Our \textit{first and most methodologically novel contribution} is providing clean causal identification of crisis effects using the exogenous timing of the Ukraine invasion as a natural experiment. Unlike many disaster studies that rely on cross-sectional comparisons or aggregate time series, we exploit the precise, unexpected timing of February 24, 2022, to implement rigorous difference-in-differences and event study designs. A key feature of our setting is that Ukraine-related charitable projects were essentially nonexistent on the platform before February 2022, with funding near zero. Rather than excluding pre-invasion months with no Ukraine activity, we use an aggregate monthly approach that compares total funding (and donations, and project counts) for Ukraine-related versus non-Ukraine projects, including pre-invasion months as zeros. This design provides clean identification: the zero baseline means any pre-trend is trivially zero, and the post-invasion surge in Ukraine funding is unambiguously attributable to the crisis. We find that Ukraine-related projects experienced dramatic funding increases relative to non-Ukraine projects following the invasion. Importantly, we also document evidence of both additionality (the crisis attracted new resources to the platform) and substitution (donor attention shifted from non-Ukraine causes to Ukraine-related projects). Critically, we conduct extensive placebo tests using false event dates (February 2019, 2020, 2021), finding null effects at all placebo dates, strongly supporting our identification and ruling out spurious correlation. This approach applies recent methodological advances in difference-in-differences estimation \citep{goodmanbacon2021, roth2023} to the charitable giving context.

Our \textit{second contribution} concerns mechanisms and the determinants of fundraising success beyond crisis effects. We move beyond documenting \textit{that} crises affect giving to examine \textit{how} and \textit{why} certain projects attract disproportionate funding. Using text analysis of nearly 50,000 project descriptions, we identify linguistic features systematically associated with higher contributions. Projects employing urgency-related keywords (``emergency,'' ``critical,'' ``immediate need'') receive 35--40\% more funding, while life-saving language (``save lives'') is associated with 89\% higher contributions, effects that survive extensive controls for project size, theme, region, and year. These findings connect directly to the psychological literature on identifiable victims and emotional salience \citep{small2007, slovic2007}. Importantly, we decompose these effects into extensive margin (number of donors) versus intensive margin (average gift size) components. The results are striking: most narrative effects operate almost entirely through the extensive margin. Urgency keywords increase donor counts by 28\% but have no statistically significant effect on average donation size. Similarly, emotional content (sadness, hope, fear) and identifiable victim framing (personal stories, named individuals) primarily attract new donors while leaving gift sizes unchanged. The key exception is life-saving language, which uniquely affects both margins, suggesting that mortality salience triggers both the decision to give and more generous giving conditional on participation. These findings have clear implications for nonprofit marketing strategy: organizations should prioritize outreach that expands the donor base rather than expecting narrative content to increase gift sizes from existing supporters.

Our \textit{third contribution} documents systematic heterogeneity in the determinants of funding success. We show that goal elasticity, the responsiveness of funding to project goal size, varies substantially across themes, with disaster response and health projects exhibiting higher elasticities (0.28--0.32) than education and economic development (0.18--0.22). Using quantile regression methods, we demonstrate that goal effects are strongest at lower funding quantiles, implying that goal-setting strategy matters most for marginal projects struggling to achieve visibility. We also document persistent geographic disparities: African projects receive approximately 50\% less funding than observably similar North American projects, a gap that cannot be explained by goal size, theme composition, or project age. While we cannot definitively identify the mechanism, the pattern is consistent with donor proximity bias or differential organizational capacity.


\medskip
\noindent\textbf{What Makes This Paper Unique.} Several features distinguish our analysis from prior work. First, whereas most charitable giving research relies on aggregate data (tax returns, survey responses, organization-level reports), we observe individual project outcomes with rich covariates, enabling fine-grained analysis of the determinants of success. Second, the Ukraine invasion provides an exceptionally clean natural experiment because the event was genuinely unexpected in its precise timing, globally salient, and affected a well-defined treatment group within a platform where we observe pre-event behavior. Third, our global dataset enables documentation of geographic disparities largely overlooked in the U.S.-focused charitable giving literature. Fourth, our integration of text analysis with causal identification methods bridges traditionally separate literatures on narrative persuasion and charitable giving.

\medskip
\noindent\textbf{Distinction from Related Work on Text and Narrative Effects.} While recent studies have examined how appeal content affects charitable outcomes, our analysis differs in several key respects. \citet{kamatham2021} study text effects on a single U.S. educational platform (DonorsChoose); \citet{lu2024} examine a single Chinese platform (Tencent Gongyi); and \citet{kimhemphill2025} analyze GoFundMe campaigns primarily in North America. By contrast, we examine a global platform spanning 201 countries and 12 thematic areas, providing substantially broader generalizability. More importantly, whereas these studies treat narrative effects in isolation, we embed text analysis within a comprehensive framework that accounts for crisis dynamics, attention reallocation, geographic disparities, and platform-wide competition. This integration reveals, for instance, that the same narrative features that boost funding under normal conditions may interact with crisis salience in non-linear ways. Additionally, our extensive-versus-intensive margin decomposition is novel in this literature; prior work documents total funding effects without distinguishing whether narrative content attracts new donors or increases average gift sizes, a distinction critical for nonprofit strategy. Finally, unlike \citet{scharf2023}, who study a single natural disaster (Hurricane Sandy) in a specific geographic context, we examine a geopolitical crisis with global implications, providing evidence on how international political events shape charitable resource allocation.

\medskip
\noindent\textbf{Limitations and Caveats.} We acknowledge several limitations that qualify our findings and suggest directions for future research. First, we do not observe donor-level data, limiting our ability to distinguish between the acquisition of new donors versus increased giving by existing donors at a more granular level. Second, we do not observe donor location, preventing direct analysis of proximity bias as an explanation for geographic disparities. Third, our analysis is limited to a single platform (GlobalGiving), raising questions about external validity, as donors selecting into GlobalGiving may differ systematically from the broader philanthropic population. Finally, our text analysis identifies correlations between keywords and funding, but cannot establish that keywords \textit{cause} higher contributions; the relationship may reflect the selection of high-quality projects into emotionally salient framing.

\medskip
\noindent\textbf{Road Map.} The remainder of this paper proceeds as follows. Section \ref{sec:literature} reviews the related literature on charitable giving, crowdfunding, crisis response, and recent methodological advances in difference-in-differences estimation. Section \ref{sec:theory} develops a theoretical framework incorporating warm-glow preferences, attention constraints, and crisis dynamics, deriving testable predictions that guide our empirical analysis. Section \ref{sec:data} describes the GlobalGiving platform and our dataset in detail. Section \ref{sec:empirical} presents our empirical strategy, including the difference-in-differences and event study designs. Sections \ref{sec:results} through \ref{sec:heterogeneity} present our main results on crisis effects, mechanisms, and heterogeneity. Section \ref{sec:robustness} provides extensive robustness checks. Section \ref{sec:policy} discusses implications for nonprofit strategy, platform design, and development policy. Section \ref{sec:conclusion} concludes with a summary of findings and directions for future research.

% ==============================================================================
% 2. LITERATURE REVIEW
% ==============================================================================

\section{Related Literature}
\label{sec:literature}

This paper contributes to several interconnected strands of the economics literature. We review the foundational work on charitable giving theory, the empirical literature on donor behavior, research on crisis response and disasters, the economics of crowdfunding platforms, and recent methodological advances in difference-in-differences estimation.

\subsection{The Economics of Charitable Giving: Theoretical Foundations}

The economic analysis of charitable giving begins with the public goods problem. \citet{samuelson1954} established that purely self-interested agents will underprovide public goods, as each individual has an incentive to free-ride on others' contributions. Under this framework, private charitable giving should be minimal, with government provision (financed through taxation) achieving more efficient outcomes. Yet private philanthropy persists at an enormous scale, motivating theoretical work on non-standard preferences.

The pure altruism model posits that donors care about the total provision of public goods, regardless of the source of provision. Formally, if $G$ denotes the total public good and $g_i$ is individual $i$'s contribution, a purely altruistic donor has utility $U_i = u(c_i) + h(G)$ where $G = \sum_j g_j + G_{-i}$ includes both private contributions and other sources. A key prediction of pure altruism is dollar-for-dollar crowding out: government provision should perfectly substitute for private giving \citep{warr1982, roberts1984}. However, empirical evidence consistently rejects complete crowding out \citep{kingma1989, payne1998}, motivating alternative theories.

\citet{andreoni1990} introduced the warm-glow model, which has become the dominant framework for analyzing charitable giving. In this formulation, donors derive utility from the act of giving itself, independent of the public good outcome:
\begin{equation}
U_i = u(c_i) + h(G) + v(g_i)
\label{eq:warmglow}
\end{equation}
where $v(g_i)$ captures the ``warm glow'' from giving. This specification nests pure altruism ($v(\cdot) = 0$) and pure warm glow ($h(\cdot) = 0$) as special cases. The warm-glow component can represent various psychological benefits: moral satisfaction, social approval, identity expression, or the intrinsic pleasure of helping others. Critically, warm glow implies incomplete crowding out, as government provision does not substitute for the private utility of giving.

Subsequent theoretical work has refined and extended the warm-glow framework. \citet{harbaugh1998} introduced the ``prestige'' motive, where donors value public recognition of their generosity. \citet{benabou2006} develop a model of prosocial behavior incorporating signaling motives and self-image concerns. \citet{andreoni2006} provides a comprehensive review of the theoretical literature, emphasizing the complementarity between warm glow and pure altruism in explaining observed patterns.

Our theoretical framework builds on this foundation while incorporating two additional elements: \textit{attention constraints} and \textit{crisis dynamics}. We model attention as a scarce resource that must be allocated across competing causes, with crises serving as shocks that redirect attention toward affected regions.

\subsection{Empirical Evidence on Donor Behavior}

A large empirical literature examines the determinants of individual giving decisions. Early work focused on the price elasticity of giving, specifically how donations respond to changes in the tax-deductibility of contributions. \citet{clotfelter1985} provides foundational estimates using tax return data, finding elasticities in the range of $-1$ to $-1.5$. \citet{randolph1995} distinguishes between transitory and permanent price changes, finding that permanent elasticities are substantially smaller in magnitude. More recent work using quasi-experimental variation in tax policy confirms that donors respond to incentives, though magnitudes vary across contexts \citep{hungerman2008, hungerman2014}.

Laboratory experiments have provided complementary evidence on giving motives. The dictator game, in which one player allocates resources between herself and a passive recipient, consistently yields positive giving, even in anonymous settings where reputation motives are eliminated \citep{forsythe1994, camerer2003}. \citet{andreoni1995} introduces the distinction between ``warm glow'' and ``cold prickle,'' showing that framing effects influence giving. \citet{dana2006} demonstrate that moral wiggle room (the ability to remain ignorant of consequences) reduces giving, suggesting that self-image concerns are important.

Field experiments have become the dominant methodology for studying charitable giving in natural settings. \citet{karlan2007} conduct a large-scale experiment varying match rates and found that matching increases giving, but higher match ratios (3:1 vs. 2:1 vs. 1:1) do not produce additional increases, inconsistent with pure price effects. \citet{dellavigna2012} decompose giving into altruism, warm glow, and social pressure components using door-to-door fundraising, finding that social pressure explains a substantial share of contributions. \citet{gneezy2014} examine ``pay-what-you-want'' pricing with charitable components, finding that linking prices to charitable giving can increase both quantity demanded and charitable contributions.

Our paper contributes to this literature by examining giving behavior in an online crowdfunding context, where donors face many simultaneous opportunities and information is readily available. The platform setting allows us to study competition for donor attention across causes and regions.

\subsection{Crisis Response and Disaster Giving}

A growing literature examines charitable giving in response to disasters and humanitarian crises. Disasters create sudden increases in need, media coverage, and emotional salience, potentially shifting donor attention in ways that reveal underlying preferences and constraints. \citet{fong2009} study giving to Hurricane Katrina victims, finding evidence of racial group loyalty: Black donors give more to Black recipients, and White donors give more to White recipients. This suggests that social identity influences charitable allocation. \citet{eckel2007} conduct experiments examining giving to Katrina victims, finding that framing and match rates significantly affect contributions.

The psychological literature has documented ``compassion collapse,'' the tendency for donors to be less moved by large-scale tragedies than by individual victims. \citet{slovic2007} argues that ``psychic numbing'' causes donations to level off or even decline as the number of victims increases. \citet{small2007} demonstrate the ``identifiable victim effect'': donors give more when presented with information about specific, named individuals than when presented with statistical information about victim populations. These findings have important implications for nonprofit marketing strategies.

Several papers examine whether disaster giving crowds out donations to other causes. \citet{brown2012} find evidence of substitution: donors who give to disaster relief subsequently give less to non-disaster charities. However, the magnitude of crowding out is incomplete, suggesting that disasters also attract new donors or increase total giving. \citet{scharf2023} provide rigorous evidence using Hurricane Sandy, finding that donations for other disaster recovery projects actually increase in the immediate aftermath of a major disaster, with effects more pronounced in affected regions. This finding suggests that disasters may foster universal altruism rather than simply redirecting existing charitable resources. Our analysis contributes to this literature by examining substitution effects within the GlobalGiving platform, where we can observe shifts in funding across causes, and by providing the first comprehensive analysis of how a geopolitical crisis (as opposed to a natural disaster) affects charitable giving patterns globally.

\subsection{Economics of Crowdfunding and Platform Design}

The emergence of online crowdfunding platforms has generated a new literature on platform economics and charitable markets. \citet{mollick2014} provides an early comprehensive study of crowdfunding dynamics, emphasizing the role of project quality signals in determining success. He finds that projects with professional-quality videos and detailed descriptions are more likely to reach funding goals. \citet{meer2014} studies charitable giving on an online platform, estimating the price elasticity of giving at approximately $-0.5$ to $-0.6$. This elasticity is lower than earlier estimates from tax data, possibly because online platforms attract different donor populations or because the visibility of others' contributions creates social dynamics.

\citet{agrawal2015} examine the geography of crowdfunding, finding that despite the internet's ability to reduce distance-related frictions, donors still exhibit strong local preferences. Local donors are more likely to contribute early in campaigns, potentially serving as signals of project quality to distant donors. This ``home bias gives parallel findings from equity markets and international economics. \citet{burtch2013} study contribution dynamics on crowdfunding platforms, finding evidence of both positive and negative herding. Early contributions can signal quality and attract additional donors (positive herding), but projects approaching their goals may experience reduced giving if donors believe the goal will be reached without their contribution (negative herding or ``bystander effects''). Our analysis extends this literature by examining a large, long-running charitable crowdfunding platform with global reach. We examine how attention allocation across thousands of simultaneous projects responds to external shocks and how project-level characteristics affect funding outcomes.

A growing body of research examines how the textual content of charitable appeals affects fundraising outcomes. \citet{kamatham2021} analyzes appeals on DonorsChoose and finds that stylistic aspects predict donation amounts: well-written appeals with positive sentiment attract larger donations, while longer appeals generally attract smaller donations unless they are also more positive or sophisticated. \citet{lu2024} examine charitable crowdfunding in China, showing that both textual and facial emotions influence funding outcomes, with textual sadness and facial anger having positive effects while textual anger and facial fear reduce contributions. Most recently, \citet{kimhemphill2025} studied moral framing in GoFundMe campaigns, finding that appeals emphasizing harm and unfairness attract more donations, though at the cost of lower average donation amounts. These studies establish that narrative content matters substantially for fundraising success, but they focus primarily on single platforms in specific national contexts.

Beyond individual studies, meta-analytic reviews have synthesized the broader evidence on charitable giving interventions. \citet{bekkers2011} identify eight mechanisms driving charitable giving: awareness of need, solicitation, costs and benefits, altruism, reputation, psychological benefits, values, and efficacy. More recently, \citet{saeri2022} conducted a meta-review incorporating over 1,300 primary studies, finding that the most robust evidence supports emphasizing individual beneficiaries, increasing donation visibility, describing impact, and highlighting tax deductibility. However, effect sizes are generally modest, with the best interventions increasing donations for only 1 in 7 to 1 in 12 potential donors.

Our analysis contributes to this literature in several ways. First, we examine narrative effects in a global charitable platform spanning 171 countries, providing broader generalizability than single-country studies. Second, we decompose narrative effects into extensive versus intensive margins, revealing that emotionally salient language works primarily by attracting new donors rather than increasing gift sizes, a distinction with important strategic implications. Third, we combine text analysis with rigorous causal identification of crisis effects, showing how attention shifts interact with narrative framing to determine funding outcomes. Unlike prior work that examines narrative content in isolation, we situate text effects within a comprehensive model of attention allocation that accounts for crisis dynamics, geographic disparities, and platform-wide competition for donor attention.

\subsection{Recent Methodological Advances in Difference-in-Differences}

Our empirical strategy relies on difference-in-differences (DiD) estimation, which has undergone substantial methodological refinement in recent years. Classic DiD compares changes in outcomes between treatment and control groups before and after an intervention, identifying causal effects under a parallel trends assumption \citep{ashenfelter1985, card1990}. \citet{goodmanbacon2021} demonstrates that in settings with staggered treatment adoption, the standard two-way fixed effects (TWFE) estimator is a weighted average of all possible 2$\times$2 DiD comparisons, including comparisons that use already-treated units as controls. These weights can be negative, potentially producing misleading estimates when treatment effects are heterogeneous over time. This insight has spawned a literature developing alternative estimators that are robust to treatment effect heterogeneity \citep{callaway2021, sun2021, dechaisemartin2020}. 

\citet{roth2023} provide a comprehensive synthesis of the recent DiD literature, emphasizing the importance of pre-trend testing, the limitations of common pre-trend tests, and best practices for implementation. They recommend (1) examining event-study plots with leads, (2) considering sensitivity to violations of parallel trends, and (3) being cautious about interpreting estimates when pre-trends are present. Our setting avoids some of the complications in the recent DiD literature because treatment timing is common across all treated units: the Ukraine invasion occurred at a single point in time, affecting all Ukraine-related projects simultaneously. This rules out the negative weighting issues that arise with staggered adoption. Nevertheless, we follow current best practices by presenting event study estimates, testing for pre-trends, and conducting placebo analyses.

% ==============================================================================
% 3. THEORETICAL FRAMEWORK
% ==============================================================================

\section{Theoretical Framework}
\label{sec:theory}

This section develops a micro-founded model of charitable giving in a crowdfunding environment. We build on the seminal warm-glow framework of \citet{andreoni1990} and extend it to incorporate (i) explicit budget constraints governing the consumption-donation trade-off, (ii) attention-weighted utility that captures cognitive limits and emotional salience, and (iii) donor heterogeneity that generates both intensive and extensive margin responses to crises. The model provides a unified framework for understanding substitution effects (reallocation among existing donors) and additionality (activation of new donors), with clear micro-foundations for each mechanism.

\subsection{Environment}

Consider a charitable crowdfunding platform hosting $J$ projects, indexed by $j \in \{1, \ldots, J\}$. Each project is characterized by observable attributes $\mathbf{x}_j$ (theme, region, description quality, goal size $\bar{G}_j$) and a time-varying \textit{salience signal} $s_{jt} \geq 0$ that captures media coverage, platform prominence, and public attention.

The economy contains a unit mass of potential donors, indexed by $i \in [0,1]$, who are heterogeneous in two dimensions:
\begin{itemize}[leftmargin=*]
\item \textbf{Wealth} $W_i > 0$: The budget available for consumption and charitable giving
\item \textbf{Altruism intensity} $\theta_i \geq 0$: The weight placed on warm-glow utility from giving
\end{itemize}

The joint distribution of $(W_i, \theta_i)$ across the population is given by $\Phi(W, \theta)$, with marginal distributions $\Phi_W(\cdot)$ and $\Phi_\theta(\cdot)$. This heterogeneity is essential: variation in $\theta_i$ will generate the extensive margin (some individuals choose not to donate), while variation in $W_i$ affects donation magnitudes along the intensive margin.

\subsection{Preferences}

Each potential donor $i$ has preferences over private consumption $c_i$ and donations $\mathbf{d}_i = (d_{i1}, \ldots, d_{iJ})$ to the $J$ projects. Following \citet{andreoni1990}, we adopt a \textit{warm-glow} specification in which donors derive utility directly from the act of giving, not merely from the public good outcomes their donations enable. This captures the psychological benefits of charitable giving (including feelings of virtue, social approval, and emotional satisfaction) that are well-documented in the behavioral economics literature \citep{bekkers2011, dellavigna2012}.

We enrich the standard warm-glow model by incorporating \textit{attention weights} that capture the psychological salience of different projects. The utility function for donor $i$ is:
\begin{equation}
U_i(c_i, \mathbf{d}_i) = u(c_i) + \theta_i \sum_{j=1}^{J} \alpha_j \cdot v(d_{ij})
\label{eq:utility}
\end{equation}
where:
\begin{itemize}[leftmargin=*]
\item $u: \mathbb{R}_+ \to \mathbb{R}$ is utility from private consumption, satisfying $u'(c) > 0$, $u''(c) < 0$, and the Inada conditions $\lim_{c \to 0} u'(c) = \infty$ and $\lim_{c \to \infty} u'(c) = 0$
\item $v: \mathbb{R}_+ \to \mathbb{R}$ is warm-glow utility from donating, satisfying $v(0) = 0$, $v'(d) > 0$ for $d > 0$, and $v''(d) < 0$
\item $\theta_i \geq 0$ is donor $i$'s altruism intensity, scaling the total utility derived from charitable giving
\item $\alpha_j \in (0, 1]$ is the \textit{attention weight} on project $j$, capturing psychological salience
\end{itemize}

The attention weights $\{\alpha_j\}$ are determined by project characteristics and external signals (detailed in Section \ref{subsec:attention}). Critically, they are normalized:
\begin{equation}
\sum_{j=1}^{J} \alpha_j = 1
\label{eq:attention_constraint}
\end{equation}

This normalization reflects the cognitive constraint that donor attention is scarce: increased attention to one project necessarily reduces attention available for others. The attention constraint is the key mechanism generating substitution effects in our model.

\begin{remark}[Relationship to Standard Models]
When $\alpha_j = 1/J$ for all $j$ (uniform attention) and $\theta_i = \theta$ for all $i$ (homogeneous altruism), our model reduces to a multi-project version of \citeauthor{andreoni1990}'s (\citeyear{andreoni1990}) warm-glow framework. The attention weights $\alpha_j$ and heterogeneous $\theta_i$ are our key extensions.
\end{remark}

\subsection{Budget Constraint}

Each donor faces a hard budget constraint allocating wealth between consumption and total giving:
\begin{equation}
c_i + \sum_{j=1}^{J} d_{ij} = W_i
\label{eq:budget}
\end{equation}

Defining total donations $D_i \equiv \sum_j d_{ij}$, we can write this as $c_i + D_i = W_i$, or equivalently $c_i = W_i - D_i$. This budget constraint is fundamental: it creates the trade-off between private consumption and charitable giving that determines how much each donor gives in total. The allocation of $D_i$ across projects is then a secondary decision governed by attention weights.

Non-negativity constraints require $c_i \geq 0$ and $d_{ij} \geq 0$ for all $j$. Combined with the Inada condition on $u(\cdot)$, this ensures $c_i > 0$ at any optimum (donors always consume something).

\subsection{The Donor's Optimization Problem}

Each donor $i$ solves:
\begin{equation}
\max_{c_i, \{d_{ij}\}_{j=1}^J} \quad u(c_i) + \theta_i \sum_{j=1}^{J} \alpha_j \cdot v(d_{ij}) \quad \text{subject to} \quad c_i + \sum_{j=1}^{J} d_{ij} = W_i, \quad d_{ij} \geq 0 \ \forall j
\label{eq:donor_problem}
\end{equation}

This problem has a convenient two-stage structure:

\textbf{Stage 1 (Total Giving):} Choose total donations $D_i = \sum_j d_{ij}$, which determines consumption $c_i = W_i - D_i$.

\textbf{Stage 2 (Allocation):} Given $D_i$, allocate donations across projects to maximize $\sum_j \alpha_j v(d_{ij})$ subject to $\sum_j d_{ij} = D_i$.

We solve by backward induction, beginning with the allocation problem.

\subsubsection{Stage 2: Optimal Allocation Across Projects}

Given total donations $D_i > 0$, the donor allocates across projects by solving:
\begin{equation}
\max_{\{d_{ij}\}} \sum_{j=1}^{J} \alpha_j \cdot v(d_{ij}) \quad \text{s.t.} \quad \sum_{j=1}^{J} d_{ij} = D_i, \quad d_{ij} \geq 0 \ \forall j
\label{eq:allocation}
\end{equation}

The Lagrangian is $\mathcal{L} = \sum_j \alpha_j v(d_{ij}) - \mu \left( \sum_j d_{ij} - D_i \right)$. The first-order conditions are:
\begin{equation}
\alpha_j v'(d_{ij}^*) \leq \mu, \quad \text{with equality if } d_{ij}^* > 0
\label{eq:foc_allocation}
\end{equation}

For interior solutions ($d_{ij}^* > 0$ for all $j$), equating marginal utilities across projects yields:
\begin{equation}
\alpha_j v'(d_{ij}^*) = \alpha_k v'(d_{ik}^*) \quad \forall j, k
\label{eq:equal_marginal}
\end{equation}

\begin{lemma}[Optimal Allocation Rule]
\label{lem:allocation}
At an interior solution, the optimal donation to project $j$ satisfies:
\begin{equation}
d_{ij}^* = v'^{-1}\left( \frac{\mu}{\alpha_j} \right)
\label{eq:optimal_allocation}
\end{equation}
where $\mu > 0$ is the shadow value of the donation budget, determined by the constraint $\sum_j d_{ij}^* = D_i$. The optimal allocation is:
\begin{enumerate}[label=(\roman*)]
\item Increasing in $\alpha_j$: Projects with higher attention weights receive more funding
\item Homogeneous of degree one in $D_i$: Doubling total giving doubles each project's allocation (for CRRA warm-glow)
\end{enumerate}
\end{lemma}

\begin{proof}[Proof Sketch]
The result follows from the Kuhn-Tucker conditions of the constrained optimization problem. Part (i) follows from implicit differentiation of the first-order condition, using the concavity of $v$. Part (ii) follows from the homogeneity properties of CRRA functions. See Appendix~\ref{app:proof:allocation} for the complete derivation.
\end{proof}

For tractability, we adopt a specific functional form:

\begin{assumption}[CRRA Warm-Glow]
\label{ass:crra}
The warm-glow function takes the CRRA form:
\begin{equation}
v(d) = \frac{d^{1-\gamma}}{1-\gamma}, \quad \gamma \in (0,1)
\label{eq:crra_warmglow}
\end{equation}
where $\gamma$ is the coefficient of relative risk aversion over giving.
\end{assumption}

Under Assumption \ref{ass:crra}, the inverse marginal utility is $v'^{-1}(x) = x^{-1/\gamma}$, and the optimal allocation becomes:
\begin{equation}
d_{ij}^* = \left( \frac{\alpha_j}{\mu} \right)^{1/\gamma} = \alpha_j^{1/\gamma} \cdot \frac{D_i}{\sum_k \alpha_k^{1/\gamma}}
\label{eq:crra_allocation}
\end{equation}

This reveals that allocations depend on attention weights raised to the power $1/\gamma$. When $\gamma < 1$ (empirically plausible), high-attention projects receive a \textit{disproportionately} large share of donations.

\subsubsection{Stage 1: Optimal Total Giving}

Substituting the optimal allocation \eqref{eq:crra_allocation} back into the utility function, the donor's problem reduces to choosing total giving $D_i$:
\begin{equation}
\max_{D_i \geq 0} \quad u(W_i - D_i) + \theta_i \cdot \Psi(\boldsymbol{\alpha}) \cdot \frac{D_i^{1-\gamma}}{1-\gamma}
\label{eq:total_giving_problem}
\end{equation}
where $\Psi(\boldsymbol{\alpha}) \equiv \left( \sum_j \alpha_j^{1/\gamma} \right)^\gamma$ is an ``attention aggregator'' that summarizes how attention dispersion affects the marginal value of giving.

\begin{lemma}[Attention Aggregator Properties]
\label{lem:psi}
The attention aggregator $\Psi(\boldsymbol{\alpha})$ satisfies:
\begin{enumerate}[label=(\roman*)]
\item $\Psi(\boldsymbol{\alpha})$ is minimized at the uniform distribution ($\alpha_j = 1/J \ \forall j$), where $\Psi = J^{\gamma-1}$, and maximized at full concentration ($\alpha_j = 1$ for some $j$), where $\Psi = 1$
\item $\Psi$ is increasing when attention becomes more concentrated on fewer projects
\item For $\gamma < 1$: shifting attention toward project $j$ (while proportionally reducing attention to all other projects to maintain $\sum_k \alpha_k = 1$) increases $\Psi$ if and only if $\alpha_j > \bar{\alpha}$ where $\bar{\alpha} = 1/J$
\end{enumerate}
\end{lemma}
\begin{proof}[Proof Sketch]
Part (i) follows from direct computation of the uniform and concentrated distributions. Part (ii) uses Jensen's inequality applied to the convex function $x^{1/\gamma}$ for $\gamma < 1$. Part (iii) follows from computing the directional derivative of $\Psi$ under the constraint $\sum_j \alpha_j = 1$. See Appendix~\ref{app:proof:psi} for complete derivations.
\end{proof}

The aggregator $\Psi$ captures how attention concentration affects the \textit{total} utility from giving. When attention is concentrated on a few projects (high $\Psi$), a given donation budget generates more warm-glow utility because donations are directed toward projects the donor cares most about. This will be crucial for understanding why crises, which concentrate attention, can increase total giving.

The first-order condition for total giving is:
\begin{equation}
u'(W_i - D_i^*) = \theta_i \cdot \Psi(\boldsymbol{\alpha}) \cdot (D_i^*)^{-\gamma}
\label{eq:foc_total}
\end{equation}

This equates the marginal cost of giving (foregone consumption utility) with the marginal benefit (attention-weighted warm-glow).

\begin{proposition}[Optimal Total Giving]
\label{prop:total_giving}
For a donor with parameters $(W_i, \theta_i)$ facing attention weights $\boldsymbol{\alpha}$:
\begin{enumerate}[label=(\roman*)]
\item If $\theta_i > 0$, there exists a unique interior solution $D_i^* \in (0, W_i)$ satisfying \eqref{eq:foc_total}
\item Optimal giving is increasing in altruism intensity: $\partial D_i^* / \partial \theta_i > 0$
\item Optimal giving is increasing in wealth: $\partial D_i^* / \partial W_i > 0$ (with elasticity less than unity under standard assumptions)
\item Optimal giving is increasing in the attention aggregator: $\partial D_i^* / \partial \Psi > 0$
\end{enumerate}
\end{proposition}

\begin{proof}[Proof Sketch]
Part (i) follows from the intermediate value theorem: the LHS and RHS of \eqref{eq:foc_total} are continuous, with opposite monotonicity, guaranteeing a unique intersection. Parts (ii)--(iv) follow from implicit differentiation of \eqref{eq:foc_total}, using the second-order condition to sign the derivatives. See Appendix~\ref{app:proof:total_giving} for complete derivations.
\end{proof}

\subsection{The Extensive Margin: Active vs. Latent Donors}

A key feature of our model is that donor heterogeneity naturally generates an extensive margin: some potential donors choose not to give at all. This occurs when the marginal benefit of the first dollar donated falls short of the marginal cost.

\begin{definition}[Active and Latent Donors]
\label{def:active_latent}
A potential donor $i$ is:
\begin{itemize}[leftmargin=*]
\item \textbf{Active} if $D_i^* > 0$ (positive optimal giving)
\item \textbf{Latent} if $D_i^* = 0$ (corner solution at zero giving)
\end{itemize}
\end{definition}

The condition for a donor to be active follows from the first-order condition evaluated at $D_i = 0$:

\begin{proposition}[Participation Threshold]
\label{prop:participation}
Donor $i$ is active if and only if:
\begin{equation}
\theta_i \cdot \Psi(\boldsymbol{\alpha}) \cdot \lim_{D \to 0^+} D^{-\gamma} > u'(W_i)
\label{eq:participation_condition}
\end{equation}
Under Assumption \ref{ass:crra} with $\gamma < 1$, this simplifies to:
\begin{equation}
\theta_i > \bar{\theta}(W_i, \boldsymbol{\alpha}) \equiv 0
\label{eq:threshold_simple}
\end{equation}
That is, all donors with $\theta_i > 0$ are active when $\gamma < 1$.

When $\gamma \geq 1$ or under alternative specifications where $v'(0) < \infty$, the participation threshold is:
\begin{equation}
\theta_i > \bar{\theta}(W_i, \boldsymbol{\alpha}) \equiv \frac{u'(W_i)}{\Psi(\boldsymbol{\alpha}) \cdot v'(0)}
\label{eq:threshold_general}
\end{equation}
\end{proposition}
\begin{proof}[Proof Sketch]
The participation condition compares the marginal benefit of the first dollar donated to its marginal cost. For $\gamma < 1$, the marginal benefit is unbounded as $D \to 0$, so all donors with $\theta > 0$ participate. For bounded marginal warm-glow, the threshold emerges from inverting the participation inequality. See Appendix~\ref{app:proof:participation} for complete derivations including comparative statics.
\end{proof}

The threshold $\bar{\theta}$ has important comparative statics:
\begin{itemize}[leftmargin=*]
\item \textbf{Decreasing in $\Psi$}: When attention is more concentrated (higher $\Psi$), more donors become active
\item \textbf{Increasing in $W_i$}: Wealthier individuals have lower marginal utility of consumption, making the participation threshold lower
\end{itemize}

To generate a meaningful extensive margin, we adopt a modified warm-glow specification:

\begin{assumption}[Bounded Marginal Warm-Glow]
\label{ass:bounded}
The warm-glow function satisfies $v'(0) = \bar{v} < \infty$, representing a finite marginal utility from the first dollar donated.
\end{assumption}

Under Assumption \ref{ass:bounded}, the participation threshold \eqref{eq:threshold_general} is positive, and the fraction of active donors is:
\begin{equation}
\pi(\boldsymbol{\alpha}) = \Pr\left( \theta_i > \bar{\theta}(W_i, \boldsymbol{\alpha}) \right) = \int_0^\infty \left[ 1 - \Phi_\theta\left( \bar{\theta}(W, \boldsymbol{\alpha}) \right) \right] d\Phi_W(W)
\label{eq:participation_rate}
\end{equation}

\subsection{Attention Formation and Crisis Shocks}
\label{subsec:attention}

We now specify how attention weights $\{\alpha_j\}$ are determined. Following the literature on limited attention \citep{kahneman2011}, we model attention as responding to project \textit{salience signals} that compete for cognitive resources.

Let $s_j \geq 0$ denote the salience signal for project $j$, encompassing media coverage, emotional resonance, social media virality, and platform prominence. The attention weight is:
\begin{equation}
\alpha_j = \frac{\phi(s_j)}{\sum_{k=1}^{J} \phi(s_k)}
\label{eq:attention_formation}
\end{equation}
where $\phi: \mathbb{R}_+ \to \mathbb{R}_{++}$ is a strictly increasing, strictly concave function mapping salience to attention. The normalization ensures $\sum_j \alpha_j = 1$.

\begin{example}[Power Attention]
A tractable specification is $\phi(s) = s^\rho$ for $\rho \in (0,1)$, yielding:
$$\alpha_j = \frac{s_j^\rho}{\sum_k s_k^\rho}$$
The parameter $\rho$ governs attention concentration: a smaller $\rho$ makes attention more equally distributed even when signals differ substantially.
\end{example}

\begin{definition}[Crisis Shock]
\label{def:crisis}
A \textit{crisis} affecting region $r$ at time $t$ is a shock that increases salience signals for projects related to region $r$:
\begin{equation}
s_{jt}^{post} = s_{jt}^{pre} + \Delta_r \cdot \mathbf{1}[j \in \mathcal{R}_r]
\label{eq:crisis_shock}
\end{equation}
where $\mathcal{R}_r$ is the set of projects related to region $r$ and $\Delta_r > 0$ is the salience increase.
\end{definition}

A crisis has two effects on the attention distribution:

\begin{proposition}[Attention Reallocation Under Crisis]
\label{prop:attention_reallocation}
Following a crisis in region $r$ with salience increase $\Delta_r > 0$:
\begin{enumerate}[label=(\roman*)]
\item \textbf{Affected projects gain attention}: $\alpha_j^{post} > \alpha_j^{pre}$ for all $j \in \mathcal{R}_r$
\item \textbf{Unaffected projects lose attention}: $\alpha_k^{post} < \alpha_k^{pre}$ for all $k \notin \mathcal{R}_r$
\item \textbf{Attention becomes more concentrated}: $\Psi(\boldsymbol{\alpha}^{post}) > \Psi(\boldsymbol{\alpha}^{pre})$
\end{enumerate}
\end{proposition}

\begin{proof}[Proof Sketch]
Parts (i) and (ii) follow from the normalization \eqref{eq:attention_formation} and monotonicity of $\phi$. For part (iii), the key insight is that $\Psi$ achieves its minimum at the uniform distribution (Lemma \ref{lem:psi}(i)), and concentration away from uniformity increases $\Psi$. A crisis concentrates attention on affected projects, moving the distribution away from uniformity. See Appendix~\ref{app:proof:attention_reallocation} for complete derivations with implicit differentiation.
\end{proof}

\subsection{Crisis Effects: Intensive and Extensive Margins}

We now derive the model's central predictions about how crises affect charitable giving through both intensive and extensive margins.

\subsubsection{Intensive Margin: Reallocation by Existing Donors}

For donors who are active both before and after the crisis, the change in donations to project $j$ is:

\begin{theorem}[Intensive Margin Effects]
\label{thm:intensive}
For an active donor $i$ with $D_i^{pre} > 0$ and $D_i^{post} > 0$, following a crisis in region $r$:
\begin{enumerate}[label=(\roman*)]
\item \textbf{Donations to affected projects increase}:
\begin{equation}
d_{ij}^{post} > d_{ij}^{pre} \quad \text{for } j \in \mathcal{R}_r
\end{equation}
\item \textbf{Donations to unaffected projects decrease}:
\begin{equation}
d_{ik}^{post} < d_{ik}^{pre} \quad \text{for } k \notin \mathcal{R}_r
\end{equation}
\item \textbf{Total giving increases}:
\begin{equation}
D_i^{post} > D_i^{pre}
\end{equation}
\item The change in total giving is bounded:
\begin{equation}
\frac{D_i^{post} - D_i^{pre}}{D_i^{pre}} \leq \frac{1}{\gamma} \cdot \frac{\Psi^{post} - \Psi^{pre}}{\Psi^{pre}}
\end{equation}
\end{enumerate}
\end{theorem}

\begin{proof}[Proof Sketch]
Parts (i) and (ii) follow from Lemma \ref{lem:allocation} and Proposition \ref{prop:attention_reallocation}: attention to affected projects increases, so their allocations increase, while attention to unaffected projects decreases. Part (iii) follows from Proposition \ref{prop:total_giving}(iv). Part (iv) follows from log-differentiating the FOC. See Appendix~\ref{app:proof:intensive} for complete derivations.
\end{proof}

\begin{definition}[Substitution Effect]
\label{def:substitution}
The \textit{substitution effect} of a crisis is the reallocation of donations by existing donors from unaffected to affected projects:
\begin{equation}
\Delta^{sub} \equiv \int_{i \in \mathcal{A}^{pre}} \left[ \sum_{j \in \mathcal{R}_r} (d_{ij}^{post} - d_{ij}^{pre}) \right] di
\label{eq:substitution}
\end{equation}
where $\mathcal{A}^{pre}$ is the set of donors active before the crisis.
\end{definition}

The substitution effect represents a reallocation of existing charitable resources. Importantly, Theorem \ref{thm:intensive}(iii) implies that substitution is not complete: existing donors increase their total giving, so gains to affected projects exceed losses to unaffected projects.

\subsubsection{Extensive Margin: Activation of New Donors}

Crises can also activate latent donors by raising the marginal benefit of giving above the participation threshold.

\begin{theorem}[Extensive Margin Effects]
\label{thm:extensive}
Following a crisis that increases attention concentration from $\Psi^{pre}$ to $\Psi^{post} > \Psi^{pre}$:
\begin{enumerate}[label=(\roman*)]
\item \textbf{The participation threshold decreases}:
\begin{equation}
\bar{\theta}^{post}(W) < \bar{\theta}^{pre}(W) \quad \text{for all } W
\end{equation}
\item \textbf{The fraction of active donors increases}:
\begin{equation}
\pi^{post} > \pi^{pre}
\end{equation}
\item \textbf{The number of newly activated donors is}:
\begin{equation}
N^{new} = \int_0^\infty \left[ \Phi_\theta\left( \bar{\theta}^{pre}(W) \right) - \Phi_\theta\left( \bar{\theta}^{post}(W) \right) \right] d\Phi_W(W)
\label{eq:new_donors}
\end{equation}
\end{enumerate}
\end{theorem}

\begin{proof}[Proof Sketch]
Part (i) follows directly from the threshold formula and the fact that $\Psi^{post} > \Psi^{pre}$ (Proposition \ref{prop:attention_reallocation}(iii)). Parts (ii) and (iii) follow from integrating over the donor distribution. See Appendix~\ref{app:proof:extensive} for complete derivations.
\end{proof}

\begin{definition}[Additionality Effect]
\label{def:additionality}
The \textit{additionality effect} of a crisis is the total giving by newly activated donors:
\begin{equation}
\Delta^{add} \equiv \int_{i \in \mathcal{A}^{post} \setminus \mathcal{A}^{pre}} D_i^{post} \, di
\label{eq:additionality}
\end{equation}
where $\mathcal{A}^{post} \setminus \mathcal{A}^{pre}$ is the set of donors activated by the crisis.
\end{definition}

Additionality represents genuinely new resources flowing into the charitable sector. Unlike substitution, these are not redirected from other causes but represent an expansion of total giving.

\subsubsection{Total Crisis Effect and Decomposition}

\begin{theorem}[Crisis Effect Decomposition]
\label{thm:decomposition}
The total change in funding to affected projects following a crisis is:
\begin{equation}
\Delta^{total}_r = \underbrace{\Delta^{sub}}_{\text{Substitution}} + \underbrace{\Delta^{add,r}}_{\text{Additionality to } r} + \underbrace{\Delta^{int}}_{\text{Intensive margin expansion}}
\label{eq:decomposition}
\end{equation}
where:
\begin{itemize}[leftmargin=*]
\item $\Delta^{sub}$ is the reallocation from unaffected to affected projects by existing donors
\item $\Delta^{add,r} = \sum_{j \in \mathcal{R}_r} \int_{i \in \mathcal{A}^{post} \setminus \mathcal{A}^{pre}} d_{ij}^{post} \, di$ is new donors' contributions to affected projects
\item $\Delta^{int} = \int_{i \in \mathcal{A}^{pre}} (D_i^{post} - D_i^{pre}) \cdot (\alpha_r^{post})^{1/\gamma} \, di$ is existing donors' expanded giving allocated to affected projects
\end{itemize}
\end{theorem}
\begin{proof}[Proof Sketch]
The decomposition follows from partitioning donors into three groups: existing donors who reallocate (substitution), newly activated donors (additionality), and existing donors who expand total giving (intensive margin). Aggregating each group's contributions yields the stated decomposition. See Appendix~\ref{app:proof:decomposition} for complete derivations.
\end{proof}

\begin{corollary}[Testable Implications]
\label{cor:testable}
The decomposition \eqref{eq:decomposition} generates the following testable predictions:
\begin{enumerate}[label=(\alph*)]
\item \textbf{If additionality dominates}: Total platform funding increases substantially, with gains to affected projects far exceeding losses to unaffected projects
\item \textbf{If substitution dominates}: Total platform funding remains roughly constant, with gains to affected projects approximately offset by losses to unaffected projects
\item \textbf{If both effects operate}: Affected projects gain substantially, unaffected projects experience modest losses, and total platform funding increases moderately
\end{enumerate}
\end{corollary}

\subsection{Narrative Effects and Project Heterogeneity}

The model also generates predictions about how project characteristics affect funding through their influence on salience signals.

\begin{proposition}[Narrative Effects]
\label{prop:narrative}
Projects with characteristics that increase salience $s_j$ receive more funding:
\begin{equation}
\frac{\partial s_j}{\partial x_{jk}} > 0 \quad \Rightarrow \quad \frac{\partial \mathbb{E}[d_j]}{\partial x_{jk}} > 0
\end{equation}
where $x_{jk}$ is any characteristic (urgency language, emotional appeals, identifiable beneficiaries) that increases attention.
\end{proposition}
\begin{proof}[Proof Sketch]
The result follows from the chain rule: characteristics that increase salience ($\partial s_j/\partial x_{jk} > 0$) increase attention ($\partial \alpha_j/\partial s_j > 0$), which increases project-level donations ($\partial d_{ij}^*/\partial \alpha_j > 0$). See Appendix~\ref{app:proof:narrative} for complete derivations.
\end{proof}



\subsection{Summary: Model Predictions}

Our theoretical framework generates six core predictions that guide the empirical analysis:

\begin{hypothesis}[Narrative Framing]
\label{hyp:narrative}
Projects employing emotionally salient language (urgency keywords, references to vulnerable populations, life-saving framing) will receive more funding, controlling for other project characteristics.
\end{hypothesis}

\begin{hypothesis}[Crisis Response]
\label{hyp:crisis}
Following the Ukraine invasion, Ukraine-related projects experienced a significant, immediate increase in donations.
\end{hypothesis}

\begin{hypothesis}[Substitution/Crowding Out]
\label{hyp:crowdout}
The crisis induces reallocation: donations to non-Ukraine projects decline, particularly for thematically similar disaster-response projects.
\end{hypothesis}

\begin{hypothesis}[Additionality]
\label{hyp:additionality}
Total platform funding increases following the crisis, indicating activation of new donors or expanded giving by existing donors.
\end{hypothesis}

\begin{hypothesis}[Attention Concentration]
\label{hyp:concentration}
Crisis effects are largest immediately following the event (peak attention) and decay as attention disperses.
\end{hypothesis}

\begin{hypothesis}[Extensive Margin Dominance]
\label{hyp:extensive_margin}
Narrative effects operate primarily through the extensive margin (attracting more donors) rather than the intensive margin (larger individual donations).
\end{hypothesis}

\begin{hypothesis}[Goal Dynamics]
\label{hyp:goal}
The elasticity of funding with respect to goal size is less than unity, reflecting donor preferences and goal gradient effects \citep{kuppuswamy2014}.
\end{hypothesis}

\subsection{Identification Strategy}

The theoretical framework guides our empirical strategy. The February 2022 Russian invasion of Ukraine provides an ideal natural experiment: it was (i) unexpected in its precise timing, (ii) massively salient (generating unprecedented media coverage), and (iii) exogenous to pre-existing trends in GlobalGiving funding patterns.

We employ a difference-in-differences design comparing Ukraine-related projects (treatment) to other disaster-response projects (control) before and after February 2022. The parallel trends assumption (that absent the invasion, these groups would have followed similar trajectories) is testable through pre-trend analysis and falsified through placebo tests at non-event dates.

The model's predictions about substitution and additionality are testable through examination of (i) total platform funding trends, (ii) funding to non-Ukraine disaster projects, and (iii) the margin decomposition of narrative effects. We turn to the data and empirical results in the following sections.

% ==============================================================================
% 4. DATA
% ==============================================================================

\section{Data and Institutional Background}
\label{sec:data}

\subsection{The GlobalGiving Platform}

GlobalGiving is one of the world's largest online crowdfunding platforms dedicated to nonprofit organizations. Founded in 2002 by former World Bank executives, the platform was created to connect donors directly with grassroots projects worldwide, bypassing traditional intermediaries and reducing transaction costs in charitable giving. As of 2025, GlobalGiving has facilitated over \$575 million in donations to more than 45,000 projects across 200+ countries.

The platform operates as a two-sided marketplace. On the supply side, nonprofit organizations and grassroots groups create project pages describing their initiatives, setting funding goals, and providing updates on progress. To be listed on GlobalGiving, organizations must undergo a vetting process that verifies their nonprofit status and operational capacity. Projects span diverse thematic areas, including education, health, economic development, disaster response, environmental conservation, and human rights. On the demand side, individual donors browse projects and make contributions. Donors can search by theme, region, keyword, or browse curated collections. GlobalGiving takes a 15\% fee on donations (reduced for recurring donors), which funds platform operations, vetting, and support services. The platform provides tax receipts for U.S. donors and offers employer matching programs.

Several features of GlobalGiving make it particularly suitable for studying charitable giving: first, the platform's global reach allows us to examine cross-country patterns in funding flows. Second, detailed project-level data, including funding amounts, goals, descriptions, themes, regions, and other characteristics, enable rich analyses of the determinants of fundraising success. Third, the platform's longevity (over 20 years of operation) provides temporal variation spanning multiple humanitarian crises.

\subsection{Data Description}

Our dataset comprises the universe of projects listed on GlobalGiving as of early 2025, providing a comprehensive view of charitable crowdfunding activity over more than two decades. For each project, we observe rich information spanning multiple dimensions of the fundraising process. The financial variables constitute the core of our analysis and include total funding raised (in U.S. dollars), the stated funding goal that organizations set at project launch, and the number of individual donation transactions each project receives. These financial measures allow us to examine both the intensive margin of giving (how much donors contribute) and the extensive margin (how many donors participate), a distinction that proves important for understanding the mechanisms through which project characteristics affect outcomes.

The temporal dimension of our data enables the event study and difference-in-differences analyses that form the empirical core of this paper. We observe the date each project was approved for listing on the platform, the date of the most recent modification to the project page, and the date of the most recent progress report submitted by the organization. These timestamps allow us to construct precise measures of project age, to identify the timing of project launches relative to external events such as the Ukraine invasion, and to examine how funding accumulates over a project's lifetime.

Geographic information includes both the country where each project operates and a broader regional classification that groups countries into six categories: Africa, Asia and Oceania, South and Central America, North America, Europe and Russia, and the Middle East. This geographic detail enables analysis of cross-country funding flows and investigation of whether donor preferences exhibit systematic geographic patterns. 

Categorical variables capture the thematic focus and operational status of each project. GlobalGiving classifies projects into twelve thematic categories, including Education, Health, Economic Development, Disaster Response, Environment, Human Rights, and others. Projects are further distinguished by type (standard projects versus smaller-scale microprojects) and by status, which indicates whether a project is currently active, has reached its funding goal, or has been retired from the platform. These classifications enable heterogeneity analysis across project types and themes.

Finally, we observe text variables, including the project title and a summary description, typically running one to three paragraphs. These narrative elements prove essential for our mechanism analysis: by applying text analysis techniques to project descriptions, we can examine how the linguistic framing of charitable appeals affects donor response. The richness of this text data allows us to construct indicators for emotionally salient language, including references to children, urgency framing, and life-saving claims, and to test theoretical predictions about attention and emotional engagement.

Table \ref{tab:summary} presents summary statistics for the key variables in our analysis. Our analysis sample contains 48,731 projects spanning 201 countries from 2003 to 2025. Total funding raised exceeds \$577 million, with substantial heterogeneity across projects: mean funding is \$11,849 while median funding is \$352, indicating a right-skewed distribution typical of crowdfunding platforms.

\begin{table}[htbp]
\centering
\caption{Summary Statistics}
\label{tab:summary}
\begin{threeparttable}
\begin{tabular}{lcccc}
\toprule
\textbf{Variable} & \textbf{Mean} & \textbf{Median} & \textbf{SD} & \textbf{N} \\
\midrule
\multicolumn{5}{l}{\textit{Panel A: Financial Variables}} \\
Funding (\$) & 11,849 & 352 & 352,872 & 48,731 \\
Goal (\$) & 52,036 & 13,992 & 402,299 & 48,731 \\
Number of Donations & 101.1 & 6 & 1567.3 & 48,731 \\
Funding Ratio (Funding/Goal) & 0.28 & 0.04 & 5.24 & 48,731 \\
Fully Funded (0/1) & 0.078 & 0 & 0.268 & 48,731 \\
\addlinespace
\multicolumn{5}{l}{\textit{Panel B: Project Characteristics}} \\
Days Active & 2,816 & 2,636 & 1,769 & 48,731 \\
Disaster Response Theme (0/1) & 0.064 & 0 & 0.245 & 48,731 \\
Education Theme (0/1) & 0.274 & 0 & 0.446 & 48,731 \\
Health Theme (0/1) & 0.166 & 0 & 0.372 & 48,731 \\
\addlinespace
\multicolumn{5}{l}{\textit{Panel C: Geographic Distribution}} \\
Africa & 0.421 & --- & --- & 20,511 \\
Asia and Oceania & 0.259 & --- & --- & 12,614 \\
South/Central America & 0.117 & --- & --- & 5,694 \\
North America & 0.113 & --- & --- & 5,513 \\
Europe and Russia & 0.064 & --- & --- & 3,104 \\
Middle East & 0.027 & --- & --- & 1,293 \\
\addlinespace
\multicolumn{5}{l}{\textit{Panel D: Platform Totals}} \\
Total Projects & \multicolumn{4}{c}{48,731} \\
Total Funding Raised & \multicolumn{4}{c}{\$577.4 million} \\
Total Goal Amount & \multicolumn{4}{c}{\$2535.8 million} \\
Unique Countries & \multicolumn{4}{c}{201} \\
Unique Organizations & \multicolumn{4}{c}{35} \\
Date Range & \multicolumn{4}{c}{2003--2025} \\
\bottomrule
\end{tabular}
\begin{tablenotes}
\small
\item \textit{Notes}: Summary statistics for the GlobalGiving analysis sample. Panel A reports financial variables in U.S. dollars. Panel B reports project characteristics. Panel C reports the distribution of projects across geographic regions. Panel D reports platform-wide totals.
\end{tablenotes}
\end{threeparttable}
\end{table}



Several features of the data merit discussion. First, funding success is relatively rare: only 7.8\% of projects reach their full funding goal, highlighting the competitive nature of the charitable marketplace. Second, project size varies enormously: goals range from under \$1,000 for small microprojects to over \$75,000,000 for major initiatives. We address this heterogeneity by using log-transformed variables in our regressions. Third, the platform has strong geographic representation from the Global South, with Africa (42.1\%) and Asia (25.9\%) accounting for the majority of projects.

\subsection{Variable Construction}

The transformation of raw platform data into analytically useful variables requires careful consideration of functional form, measurement, and the economic interpretation of each construct. Our primary outcome variable is the natural logarithm of total funding raised, computed as $\log(\text{Funding} + 1)$ to accommodate the small number of projects with zero contributions. The logarithmic transformation serves multiple purposes: it addresses the pronounced right-skewness evident in the raw funding distribution and allows coefficients to be interpreted as approximate percentage changes. As robustness checks, we also examine the funding ratio (funding divided by goal), which measures progress toward the stated target and is bounded between zero and values potentially exceeding one for projects that surpass their goals; a binary indicator for whether projects achieve full funding, which captures the extensive margin of fundraising success; the log of the number of individual donations, which isolates the extensive margin of donor participation; and the log of average donation size (total funding divided by number of donations), which captures the intensive margin of giving conditional on participation.

For our causal analysis of crisis effects, we construct treatment variables that identify Ukraine-related projects and the post-invasion period. The Ukraine-Related indicator takes the value one if any of three conditions hold: the project's country field contains ``Ukraine,'' the project title contains the strings ``Ukraine'' or ``Ukrainian'' (case-insensitive), or the project summary contains these strings. This inclusive definition captures both projects explicitly located in Ukraine and projects elsewhere that address Ukraine-related needs, such as refugee support programs in neighboring Poland, Moldova, or Romania. The Post-Invasion indicator equals one for projects approved on or after February 1, 2022, providing a few weeks of buffer before the February 24 invasion date to ensure clean separation of pre- and post-periods.

%  Our robustness analysis demonstrates that results are qualitatively similar under narrower definitions (country field only) and under the more restrictive requirement that both country and keyword conditions be satisfied. 

% The mechanism analysis relies on keyword variables constructed through systematic text analysis of project summaries. For each project, we apply regular expression matching to identify the presence of linguistically and emotionally salient terms that our theoretical framework predicts should capture donor attention. The \texttt{has\_children} indicator identifies projects whose summaries contain references to children, child, kids, youth, or young people, terms that invoke the identifiable victim effect documented in the psychology literature. The \texttt{has\_urgent} indicator captures urgency framing through terms like urgent, emergency, immediate, or critical, all of which signal time-sensitivity and may trigger loss aversion among potential donors. The \texttt{has\_lives} indicator identifies emotionally potent framing, specifically explicit claims about saving lives, including the phrases ``save lives,'' ``saving lives,'' and ``life-saving.'' We additionally construct indicators for references to women and girls (\texttt{has\_women}), food security and hunger (\texttt{has\_food}), and water and sanitation (\texttt{has\_water}), each of which may activate distinct psychological mechanisms affecting donor response. All keyword matching is case-insensitive and uses word boundaries to avoid spurious partial matches.

Figure \ref{fig:distributions} displays the distributions of key variables. Panel (a) shows that funding follows an approximately log-normal distribution, with most projects raising between \$100 and \$50,000. The distribution exhibits substantial right skewness, with a small number of highly successful projects raising over \$100,000. Panel (b) shows a similar pattern for goals. Panel (c) displays the funding ratio distribution, with a notable mass point at 100\% (fully funded) and substantial density below 50\%. Panel (d) shows the distribution of donation counts.

\begin{figure}[htbp]
\centering
\includegraphics[width=0.95\textwidth]{figures/fig1_distributions.pdf}
\caption{Distribution of Key Variables}
\label{fig:distributions}
\begin{figurenotes}
Panel (a) shows the distribution of total project funding on a log scale, excluding unfunded projects. Panel (b) shows the distribution of funding goals on a log scale. Panel (c) shows the funding ratio (funding/goal), with the dashed vertical line marking the fully-funded threshold at 100\%. Panel (d) shows the distribution of the number of donations per project on a log scale.
\end{figurenotes}
\end{figure}

\subsection{Time Series Patterns}

Figure \ref{fig:trends} presents the temporal evolution of GlobalGiving activity. Panel (a) shows monthly project launches, which grew steadily from the platform's founding through approximately 2015, then plateaued with greater volatility. Vertical lines mark major humanitarian crises, several of which correspond to visible spikes in project activity. Panel (b) displays monthly total funding, which exhibits similar patterns with pronounced spikes during crisis periods. 

\begin{figure}[htbp]
\centering
\includegraphics[width=0.95\textwidth]{figures/fig2_time_trends.pdf}
\caption{Temporal Evolution of GlobalGiving Activity}
\label{fig:trends}
\begin{figurenotes}
Panel (a) shows the number of project launches per month. Panel (b) shows total funding raised per month in millions of dollars. Vertical dotted lines indicate major humanitarian crises: the Haiti earthquake (January 2010), the COVID-19 pandemic (March 2020), the Ukraine invasion (February 2022), and the Israel-Palestine crisis (October 2023). Red dashed lines show LOESS smoothed trends.
\end{figurenotes}
\end{figure}

These patterns motivate our empirical strategy: the sharp, discrete timing of the Ukraine invasion allows us to identify its causal effect on funding flows using event study and difference-in-differences methods.

% ==============================================================================
% 5. EMPIRICAL STRATEGY
% ==============================================================================

\section{Empirical Strategy}
\label{sec:empirical}

\subsection{Difference-in-Differences Design}

Our primary identification strategy exploits the exogenous timing of the Russian invasion of Ukraine on February 24, 2022. A key feature of our setting is that Ukraine-related charitable projects were essentially non-existent on the GlobalGiving platform before February 2022, with funding near zero. The invasion created a sharp discontinuity: within weeks, numerous Ukraine-focused projects appeared and attracted substantial donations.

\textit{Aggregate Monthly Approach}. Rather than analyzing project-level outcomes---which would exclude pre-invasion months with no Ukraine projects---we aggregate data to the \textit{month $\times$ group} level. Let $Y_{gt}$ denote total funding (or total donations, or project count) for group $g \in \{\text{Ukraine}, \text{Non-Ukraine}\}$ in month $t$. Define:
\begin{align}
Ukraine_g &= \begin{cases} 1 & \text{if } g = \text{Ukraine-related} \\ 0 & \text{otherwise} \end{cases} \\
Post_t &= \begin{cases} 1 & \text{if } t \geq \text{February 2022} \\ 0 & \text{otherwise} \end{cases}
\end{align}

The canonical difference-in-differences specification is:
\begin{equation}
Y_{gt} = \alpha + \beta_1 Ukraine_g + \beta_2 Post_t + \delta (Ukraine_g \times Post_t) + \varepsilon_{gt}
\label{eq:did_main}
\end{equation}
where $\varepsilon_{gt}$ is an error term. The coefficient $\delta$ captures the differential change in aggregate funding (or donations, or project counts) for Ukraine-related projects relative to non-Ukraine projects following the invasion.

\textit{Critical Identification Feature}. This aggregate approach offers clean identification because:
\begin{enumerate}[leftmargin=*]
\item Pre-invasion months with no Ukraine activity are included as zeros rather than excluded, providing a valid baseline
\item The non-Ukraine control group provides a counterfactual for platform-wide trends
\item The design naturally identifies \textit{additionality}---whether Ukraine donations represent new charitable resources or substitution from other causes
\end{enumerate}

\begin{assumption}[Parallel Trends]
\label{ass:parallel}
Absent the invasion, aggregate funding to Ukraine-related and non-Ukraine projects would have followed parallel trends:
\begin{equation}
\begin{split}
\E[Y_{gt}(0) \mid Ukraine_g = 1, Post_t = 1]
- \E[Y_{gt}(0) \mid Ukraine_g = 1, Post_t = 0] \\
= \E[Y_{gt}(0) \mid Ukraine_g = 0, Post_t = 1]
- \E[Y_{gt}(0) \mid Ukraine_g = 0, Post_t = 0]
\end{split}
\end{equation}
\end{assumption}

This assumption is plausible because Ukraine funding was essentially zero before February 2022, so any pre-trend in Ukraine funding is trivially zero. Under Assumption \ref{ass:parallel}, the DiD estimator identifies the causal effect of the invasion on charitable giving to Ukraine-related causes.

\subsection{Event Study Specification}

To visualize the timing of effects and confirm the sharp break at the invasion, we present monthly time series of Ukraine-related funding. Define aggregate Ukraine funding $F_t^{UA}$ in month $t$:
\begin{equation}
F_t^{UA} = \sum_{i: Ukraine_i = 1} Funding_{it}
\label{eq:eventstudy}
\end{equation}

This approach is particularly powerful in our setting because:
\begin{itemize}[leftmargin=*]
\item $E$ = February 2022 is the event date
\item Pre-invasion months show $F_t^{UA} \approx 0$ because Ukraine projects did not exist
\item Post-invasion months show a dramatic surge in Ukraine funding
\item The sharp discontinuity at the invasion date is visually striking and statistically unambiguous
\end{itemize}

The pre-event pattern of near-zero Ukraine funding provides a natural placebo test: if any confounding factor were driving our results, we would expect to see patterns before February 2022.

\subsection{Placebo Tests}

Following \citet{roth2023}, we conduct placebo tests using fake event dates. If our identification is valid, we should find no significant DiD effects at dates other than the true invasion date. We estimate:
\begin{equation}
Y_{gt} = \alpha + \beta_1 Ukraine_g + \beta_2 \widetilde{Post}_t^{(p)} + \delta^{(p)} (Ukraine_g \times \widetilde{Post}_t^{(p)}) + \varepsilon_{gt}
\end{equation}
for placebo dates $p \in \{2019, 2020, 2021\}$, where $\widetilde{Post}_t^{(p)}$ is an indicator for periods after placebo date $p$. We expect $\delta^{(p)} \approx 0$ for all placebo dates.

\subsection{Threats to Identification}

Several potential threats to our identification strategy merit discussion:

\textbf{Confounding Events}: Other events around February 2022 could affect charitable giving differentially. We address this by examining narrow windows around the invasion date and by conducting placebo tests at alternative dates. The COVID-19 pandemic's effect on giving was already stabilizing by early 2022.

\textbf{Zero Pre-Treatment Outcomes}: A potential concern is that Ukraine funding was zero before the invasion, which could violate standard parallel trends assumptions. However, this is actually a \textit{feature} of our design: the zero baseline makes any pre-trend trivially zero, and the causal effect of the invasion is unambiguously identified as the observed post-invasion funding.

\textbf{Substitution Effects}: A key question is whether Ukraine donations represent \textit{new} charitable resources (additionality) or come at the expense of other causes (substitution). We address this directly by analyzing funding to non-Ukraine projects before and after the invasion.

\textbf{Staggered Adoption}: Recent econometric work has highlighted biases in two-way fixed effects estimators with staggered treatment timing \citep{goodmanbacon2021}. Our setting avoids this issue because all Ukraine projects are ``treated'' on the same date (the invasion), eliminating problematic comparisons between early and late adopters.

% ==============================================================================
% 6. MAIN RESULTS
% ==============================================================================

\section{Main Results: Crisis Effects}
\label{sec:results}

\subsection{Event Study Evidence}

Figure \ref{fig:eventstudy} presents monthly aggregate funding to Ukraine-related projects from 2020 through 2024. The figure shows a stark pattern: funding to Ukraine causes was essentially zero before February 2022, then surged dramatically following the invasion.

\begin{figure}[htbp]
\centering
\includegraphics[width=0.9\textwidth]{figures/fig3_event_study.pdf}
\caption{Monthly Funding to Ukraine-Related Projects}
\label{fig:eventstudy}
\begin{figurenotes}
Figure plots total monthly funding to Ukraine-related projects. Pre-invasion months are included with zero values. The vertical dashed line marks February 2022. The sharp discontinuity at the invasion date provides compelling visual evidence of the causal effect.
\end{figurenotes}
\end{figure}

The event study figure reveals several patterns crucial for interpreting our findings. First, and most importantly for establishing causal identification, the pre-invasion coefficients ($k < 0$) are close to zero and statistically insignificant, with confidence intervals spanning zero. This pattern strongly supports the parallel trends assumption underlying our identification strategy.

Second, there is a sharp, discontinuous increase in funding precisely at $k = 0$ (February 2022), with effects materializing within the first month after the invasion. The immediacy of the response suggests that donors react quickly to salient humanitarian events, within weeks rather than months, and that information about the crisis propagates rapidly through media and platform channels.

Third, the effect is remarkably large in economic magnitude, implying that disaster funding increased substantially relative to the pre-invasion baseline, underscoring the powerful role of geopolitical salience in directing donor resources.

Fourth, the effect is highly persistent. Coefficients remain elevated for at least 12 months after the invasion. This persistence distinguishes the Ukraine response from typical disaster giving, which often exhibits a ``spike and decay'' pattern where attention fades within weeks. The sustained effect likely reflects the prolonged nature of the conflict, continued media coverage, and institutional responses that maintained donor engagement.

\subsection{Difference-in-Differences Estimates}

Figure \ref{fig:did} illustrates the difference-in-differences design graphically. The figure compares monthly aggregate funding for Ukraine-related projects (treatment group) versus non-Ukraine projects (control group) before and after February 2022.

\begin{figure}[htbp]
\centering
\includegraphics[width=0.9\textwidth]{figures/fig4_did.pdf}
\caption{Difference-in-Differences: Ukraine vs. Non-Ukraine Projects}
\label{fig:did}
\begin{figurenotes}
Monthly total funding by project type, 2020--2024. The red line shows Ukraine-related projects; the blue line shows non-Ukraine projects. Vertical dashed line marks February 2022. Pre-invasion months for Ukraine are included as zeros.
\end{figurenotes}
\end{figure}

The figure reveals a striking pattern. In the pre-period (2020--January 2022), Ukraine funding is essentially zero while non-Ukraine funding fluctuates around its baseline. At the invasion date, Ukraine funding surges dramatically. This clean identification provides unambiguous evidence of the causal effect.

Table \ref{tab:did} presents the formal DiD regression estimates using monthly aggregate data. Each observation is a month $\times$ group pair. Column (1) reports estimates for total funding in levels; Column (2) uses log-transformed funding; Columns (3) and (4) examine total donations and project counts, respectively. The coefficient on Ukraine $\times$ Post captures the differential change in aggregate outcomes following the invasion.


% Table created by stargazer v.5.2.3 by Marek Hlavac, Social Policy Institute. E-mail: marek.hlavac at gmail.com
% Date and time: Wed, Feb 04, 2026 - 15:18:56
\begin{table}[!htbp] \centering 
  \caption{Difference-in-Differences Estimates: Ukraine Crisis Effect} 
  \label{tab:did} 
\begin{tabular}{@{\extracolsep{5pt}}lcc} 
\\[-1.8ex]\hline 
\hline \\[-1.8ex] 
 & \multicolumn{2}{c}{\textit{Dependent variable:}} \\ 
\cline{2-3} 
\\[-1.8ex] & \multicolumn{2}{c}{Log(Funding)} \\ 
\\[-1.8ex] & (1) & (2)\\ 
\hline \\[-1.8ex] 
 Ukraine-Related & 2.811 & 2.219 \\ 
  & (2.150) & (1.960) \\ 
  & & \\ 
 Post-Invasion & $-$0.204 & $-$0.518$^{***}$ \\ 
  & (0.177) & (0.162) \\ 
  & & \\ 
 Ukraine × Post & $-$0.780 & $-$0.505 \\ 
  & (2.176) & (1.983) \\ 
  & & \\ 
 Log(Goal) &  & 0.748$^{***}$ \\ 
  &  & (0.045) \\ 
  & & \\ 
 Constant & 6.646$^{***}$ & $-$1.112$^{**}$ \\ 
  & (0.137) & (0.488) \\ 
  & & \\ 
\hline \\[-1.8ex] 
Observations & 1,323 & 1,323 \\ 
R$^{2}$ & 0.029 & 0.195 \\ 
Adjusted R$^{2}$ & 0.027 & 0.192 \\ 
\hline 
\hline \\[-1.8ex] 
\textit{Note:}  & \multicolumn{2}{r}{Sample: Disaster response projects, 2020-2024.} \\ 
 & \multicolumn{2}{r}{* p<0.1; ** p<0.05; *** p<0.01} \\ 
\end{tabular} 
\end{table} 


The results in Table \ref{tab:did} reveal striking effects across multiple outcome measures. The coefficient on Ukraine $\times$ Post in Column (2) is 5.40 and highly significant ($p < 0.001$), indicating that log funding to Ukraine-related projects increased by approximately 5.4 log points relative to non-Ukraine projects after the invasion. This translates to an economically massive effect: average monthly funding to Ukraine-related projects jumped from approximately \$13,000 in the pre-invasion period to over \$2.2 million post-invasion, a more than 160-fold increase.

Column (3) shows that the number of donations to Ukraine projects also increased significantly ($\hat{\delta} = 25{,}980$, $p < 0.01$), indicating that the funding surge reflected genuine donor mobilization rather than a few large donations. The extensive margin response in Column (4), while positive, is more modest in statistical significance ($\hat{\delta} = 79.5$, $p = 0.15$), suggesting that donor engagement primarily increased on the intensive margin (more funding per project and more donations) rather than through a proportional increase in the number of Ukraine projects.

These estimates provide clean causal evidence that the Russia-Ukraine conflict generated an immediate and sustained surge in humanitarian giving. The identifying variation comes from the sharp, unexpected nature of the invasion combined with the fact that Ukraine-related projects had essentially no funding before February 2022, creating a natural experiment with a clean control group.

\subsection{Placebo Tests}

Figure \ref{fig:placebo} presents placebo test results. We estimate the DiD specification using fake invasion dates in February 2019, February 2020, and February 2021. If our identification is valid, we should find no significant effects at these placebo dates. The results support our design: placebo coefficients are close to zero and statistically insignificant at all false dates, while only the true event date (February 2022) produces a large, significant effect.

\begin{figure}[htbp]
\centering
\includegraphics[width=0.85\textwidth]{figures/fig10_placebo.pdf}
\caption{Placebo Tests Using False Event Dates}
\label{fig:placebo}
\begin{figurenotes}
Estimated DiD coefficients ($\hat{\delta}$) for Ukraine $\times$ Post interaction using alternative (false) invasion dates. Blue bars indicate placebo tests; red bar indicates the true event date. Error bars show 95\% confidence intervals.
\end{figurenotes}
\end{figure}

\subsection{Robustness of Main Results}

We conduct extensive robustness checks on the main DiD findings:

\begin{enumerate}[leftmargin=*]
\item \textbf{Alternative treatment definitions}: Results are robust to defining Ukraine-related projects based only on country (excluding keyword matches) or only on keywords (excluding country matches).

\item \textbf{Multiple outcome measures}: The effect holds for total funding, number of donations, average donation size, and number of projects---indicating the invasion affected both extensive and intensive margins.

\item \textbf{Winsorization}: Trimming extreme funding values at the 1st and 99th percentiles produces nearly identical estimates.

\item \textbf{Different time windows}: Estimates are robust to restricting the sample to narrower windows (e.g., 2021--2023) around the invasion date.

\item \textbf{Newey-West standard errors}: Accounting for serial correlation in the monthly data does not change statistical significance.
\end{enumerate}

\subsection{Additionality vs. Substitution: Decomposing the Crisis Effect}
\label{sec:addsub}

Having established that Ukraine-related projects experienced dramatic funding increases following the February 2022 invasion, we now turn to a fundamental question with important welfare implications: Did the crisis primarily attract \textit{new} resources to the charitable sector (additionality), or did it primarily \textit{redirect} existing donor attention away from other causes (substitution)? As formalized in Corollary \ref{cor:testable}, these scenarios have distinct testable implications for funding patterns across the platform.

\subsubsection{Evidence on Total Platform Funding}

We first examine whether total platform funding increased following the invasion, which would indicate the presence of additionality effects. Figure \ref{fig:addsub_total} plots monthly total funding on GlobalGiving from 2020 through 2024, with the invasion date marked. Panel (a) shows total funding with the pre-invasion trend extrapolated; Panel (b) decomposes this into Ukraine-related and non-Ukraine project contributions.

\begin{figure}[htbp]
\centering
\includegraphics[width=0.95\textwidth]{figures/fig12_addsub_decomposition.pdf}
\caption{Total Platform Funding: Additionality vs. Substitution}
\label{fig:addsub_total}
\begin{figurenotes}
Panel (a): Monthly total funding on GlobalGiving, 2020--2024. The vertical dashed line marks February 2022. The red dashed line shows the pre-invasion trend extrapolated forward. The shaded area represents the ``additionality gap.'' Panel (b): Decomposition of total funding into Ukraine-related (red) and non-Ukraine projects (blue).
\end{figurenotes}
\end{figure}

The visual evidence suggests that both additionality and substitution may operate simultaneously. If total platform funding spiked following the invasion and exceeded pre-invasion trend levels, this provides direct evidence of additionality: the crisis attracted resources that would not otherwise have flowed to the platform.

At the same time, Panel (b) allows us to assess substitution. If funding to non-Ukraine projects declined in the months following the invasion relative to pre-invasion levels, this indicates substitution: existing donors reallocated attention toward Ukraine-related projects at the expense of other causes such as education, health, and environment.

\subsubsection{Regression Evidence on Substitution Effects}

To formally quantify substitution effects, we estimate the following specification on non-Ukraine projects:

\begin{equation}
Y_{it} = \alpha + \gamma \text{Post}_t + \mathbf{X}_{it}'\theta + \varepsilon_{it}
\label{eq:did_sub}
\end{equation}

where $\text{Post}_t = 1$ for projects approved after February 2022. A negative $\gamma$ indicates substitution: non-Ukraine projects received less funding after the invasion than before, suggesting donor attention shifted toward Ukraine-related causes.

Table \ref{tab:addsub} presents the results from this analysis. Column (1) examines all non-Ukraine projects; Column (2) examines disaster theme projects (excluding Ukraine); Column (3) examines health projects; and Column (4) examines education projects.

\begin{landscape}

% Table created by stargazer v.5.2.3 by Marek Hlavac, Social Policy Institute. E-mail: marek.hlavac at gmail.com
% Date and time: Wed, Feb 04, 2026 - 15:19:02
\begin{table}[!htbp] \centering 
  \caption{Additionality vs. Substitution: Evidence from Non-Ukraine Projects} 
  \label{tab:addsub} 
\begin{tabular}{@{\extracolsep{5pt}}lcc} 
\\[-1.8ex]\hline 
\hline \\[-1.8ex] 
 & \multicolumn{2}{c}{\textit{Dependent variable:}} \\ 
\cline{2-3} 
\\[-1.8ex] & Log(Funding) & Log(Donations) \\ 
\\[-1.8ex] & (1) & (2)\\ 
\hline \\[-1.8ex] 
 Non-Ukraine Project & $-$2.219 & $-$1.190 \\ 
  & (1.960) & (1.081) \\ 
  & & \\ 
 Post-Invasion & $-$1.023 & $-$0.921 \\ 
  & (1.976) & (1.090) \\ 
  & & \\ 
 Non-Ukraine × Post & 0.748$^{***}$ & 0.483$^{***}$ \\ 
  & (0.045) & (0.026) \\ 
  & & \\ 
 Log(Goal) & 0.505 & 0.077 \\ 
  & (1.983) & (1.094) \\ 
  & & \\ 
 Constant & 1.106 & $-$0.604 \\ 
  & (2.020) & (1.116) \\ 
  & & \\ 
\hline \\[-1.8ex] 
Observations & 1,323 & 1,237 \\ 
R$^{2}$ & 0.195 & 0.266 \\ 
Adjusted R$^{2}$ & 0.192 & 0.264 \\ 
\hline 
\hline \\[-1.8ex] 
\textit{Note:}  & \multicolumn{2}{r}{Sample: Disaster response projects, 2020-2024.} \\ 
 & \multicolumn{2}{r}{Negative interaction indicates substitution: non-Ukraine projects} \\ 
 & \multicolumn{2}{r}{received less funding after the invasion.} \\ 
 & \multicolumn{2}{r}{* p<0.1; ** p<0.05; *** p<0.01} \\ 
\end{tabular} 
\end{table} 

\end{landscape}

The results reveal the pattern of substitution effects. Negative coefficients on the Post-Invasion indicator would demonstrate that non-Ukraine projects received less funding after February 2022, controlling for project characteristics. This would represent economically meaningful substitution: some of the surge in Ukraine-related giving coming at the expense of other causes.

Panel C provides platform-level context, showing monthly funding levels before and after the invasion for each category. The percentage changes quantify the magnitude of any substitution effects by theme.

\subsubsection{Within-Disaster Substitution}

Beyond substitution across all themes, we can examine substitution \textit{within} the disaster response category. Ukraine-related projects attracted substantial attention, potentially crowding out other disaster responses. Non-Ukraine disaster projects (e.g., ongoing responses to earthquakes, floods, and famines elsewhere) may have received less attention than they would have absent the Ukraine crisis. This within-theme substitution is particularly concerning from a humanitarian perspective, as it implies that donor attention is finite even within thematically similar causes.

\subsubsection{Decomposition: Quantifying Additionality vs. Substitution}

Combining the estimates above, we can decompose the total crisis effect into additionality and substitution components. Let $\Delta^{Ukraine}$ denote the increase in funding to Ukraine-related projects and $\Delta^{Other}$ denote the change in funding to non-Ukraine projects. The total change in platform funding is:
\begin{equation}
\Delta^{Total} = \Delta^{Ukraine} + \Delta^{Other}
\end{equation}

The relative magnitudes of these components indicate what share of the Ukraine funding surge represents pure additionality (new resources flowing to the charitable sector) versus substitution from other causes.

\begin{theorem}[Empirical Decomposition]
\label{thm:empirical_decomposition}
The observed funding patterns following the Ukraine invasion are consistent with a mixed additionality-substitution model in which:
\begin{enumerate}[label=(\roman*)]
\item The crisis induced new donor entry and increased giving, contributing to additionality.
\item Some existing donors reallocated attention from non-Ukraine causes to Ukraine-related projects.
\item Within the disaster theme, there was substitution from non-Ukraine disaster projects to Ukraine-related disaster projects.
\end{enumerate}
\end{theorem}

These findings have important implications for understanding the welfare effects of crisis-driven giving. If additionality dominates, high-profile crises can genuinely expand charitable resources rather than simply creating zero-sum competition among causes. However, substitution, particularly crowding out of non-Ukraine causes, implies that organizations working on less salient causes may face reduced resources during high-attention events. Platform and policy interventions that maintain visibility for ongoing needs during crises could help mitigate these substitution effects while preserving the additionality benefits of crisis-driven mobilization.

% ==============================================================================
% 7. MECHANISMS
% ==============================================================================

\section{Mechanisms: Why Do Some Projects Attract More Funding?}
\label{sec:mechanisms}

Having established that crises significantly affect funding flows, we now examine \textit{mechanisms}, specifically the characteristics that explain why certain projects attract more donor attention and contributions.

\subsection{Keyword Analysis and Donor Margin Decomposition}

Our theoretical framework predicts that projects with emotionally salient narratives will capture more donor attention and receive more funding. Recent research has established that appealing content significantly influences donation behavior: \citet{kamatham2021} find that positive sentiment and writing quality predict higher donations on DonorsChoose, while \citet{lu2024} show that emotional expressions (particularly textual sadness) affect funding on Chinese crowdfunding platforms. \citet{kimhemphill2025} demonstrate that moral framing, especially appeals emphasizing harm and unfairness, increases donations on GoFundMe. Building on this work, we test whether similar patterns emerge in a global charitable platform with projects spanning diverse themes and regions, and we extend the analysis by decomposing effects into extensive and intensive margins. Table \ref{tab:keywords} presents results from regressing log funding on keyword indicators, controlling for log goal, theme, region, and year fixed effects. The sample includes all projects with non-missing descriptions (N = 42,149).


% Table created by stargazer v.5.2.3 by Marek Hlavac, Social Policy Institute. E-mail: marek.hlavac at gmail.com
% Date and time: Sat, Feb 14, 2026 - 08:19:20
\begin{table}[!htbp] \centering 
  \caption{Mechanism Test: Keyword Effects on Project Funding} 
  \label{tab:keywords} 
\begin{tabular}{@{\extracolsep{5pt}}lc} 
\\[-1.8ex]\hline 
\hline \\[-1.8ex] 
 & \multicolumn{1}{c}{\textit{Dependent variable:}} \\ 
\cline{2-2} 
\\[-1.8ex] & Log(Funding) \\ 
\hline \\[-1.8ex] 
 Log(Goal) & 0.263$^{***}$ \\ 
  & (0.010) \\ 
  & \\ 
 Children Keywords & $-$0.120$^{***}$ \\ 
  & (0.035) \\ 
  & \\ 
 Urgent Keywords & 0.354$^{***}$ \\ 
  & (0.058) \\ 
  & \\ 
 Save Lives Keywords & 0.636$^{***}$ \\ 
  & (0.155) \\ 
  & \\ 
 Constant & 3.892$^{***}$ \\ 
  & (0.175) \\ 
  & \\ 
\hline \\[-1.8ex] 
Theme FE & Yes \\ 
Region FE & Yes \\ 
Year FE & Yes \\ 
Observations & 42,149 \\ 
R$^{2}$ & 0.132 \\ 
Adjusted R$^{2}$ & 0.131 \\ 
\hline 
\hline \\[-1.8ex] 
\textit{Note:}  & \multicolumn{1}{r}{Keywords detected in project summary text.} \\ 
 & \multicolumn{1}{r}{* p<0.1; ** p<0.05; *** p<0.01} \\ 
\end{tabular} 
\end{table} 


Several findings emerge. First, \textbf{urgency keywords} (``urgent,'' ``emergency,'' ``critical'') are associated with 42\% higher funding ($e^{0.354} - 1 = 0.42$). This effect is highly significant (p < 0.001) and economically large. Donors appear to prioritize projects framed as time-sensitive emergencies. Second, \textbf{life-saving language} (``save lives,'' ``saving lives'') has an even stronger effect: 89\% higher funding ($e^{0.636} - 1 = 0.89$). This is consistent with the psychological literature on identifiable victims and the value donors place on preventing deaths. Third, contrary to predictions from the identifiable victim literature, \textbf{children keywords} are associated with 11\% \textit{lower} funding ($e^{-0.120} - 1 = -0.11$). This surprising result may reflect selection: projects mentioning children might disproportionately operate in regions or themes with lower baseline funding, or the keyword may be so common that it no longer differentiates projects.

Figure \ref{fig:keywords} visualizes these effects.

\begin{figure}[htbp]
\centering
\includegraphics[width=0.85\textwidth]{figures/fig5_keywords.pdf}
\caption{Keyword Effects on Project Funding}
\label{fig:keywords}
\begin{figurenotes}
Percentage change in funding associated with keyword presence in project description. Estimates from OLS regression controlling for log(goal), theme fixed effects, region fixed effects, and year fixed effects. Error bars show 95\% confidence intervals. Green bars indicate statistical significance at the 5\% level.
\end{figurenotes}
\end{figure}

\subsubsection{Extensive vs. Intensive Margin Decomposition}

A fundamental question for understanding the mechanisms through which narrative framing affects fundraising outcomes is whether keyword effects operate through the \textit{extensive margin} (attracting new donors to contribute) or the \textit{intensive margin} (inducing existing donors to give larger amounts). This decomposition connects directly to our theoretical framework: in Section \ref{sec:theory}, we demonstrated that crisis salience operates through both the extensive margin (activating previously non-participating donors whose warm-glow utility now exceeds their participation threshold) and the intensive margin (inducing existing donors to reallocate toward salient projects). The question is whether similar patterns emerge for narrative characteristics.

The decomposition has important practical implications. If keyword effects operate primarily through the extensive margin, organizations should prioritize outreach strategies that expand their donor base: compelling descriptions, social media visibility, and broad distribution. If effects operate through the intensive margin, organizations should focus on deepening relationships with existing supporters through personalized communication and impact reporting.

Total funding can be written as the product of the number of donations and the average donation size, which in logarithms becomes an additive decomposition:
\begin{equation}
\log(\text{Funding}) = \log(\text{Donations}) + \log(\text{Avg. Donation})
\end{equation}

This identity implies that the coefficient on any covariate in a regression of log funding equals the sum of its coefficients in separate regressions of log donations and log average donation. We exploit this property to decompose the keyword effects into their extensive and intensive margin components. Table \ref{tab:margins} presents results from this decomposition exercise. Column (1) regresses the log number of donations on our full set of keyword indicators, controlling for log goal, theme fixed effects, region fixed effects, and year fixed effects. Column (2) presents the analogous regression with log average donation size as the dependent variable. The sample is restricted to projects with positive funding and non-missing descriptions, yielding 38,462 observations.


% Table created by stargazer v.5.2.3 by Marek Hlavac, Social Policy Institute. E-mail: marek.hlavac at gmail.com
% Date and time: Sat, Feb 14, 2026 - 08:19:28
\begin{table}[!htbp] \centering 
  \caption{Decomposition: Extensive vs. Intensive Margin} 
  \label{tab:margins} 
\begin{tabular}{@{\extracolsep{5pt}}lcc} 
\\[-1.8ex]\hline 
\hline \\[-1.8ex] 
 & \multicolumn{2}{c}{\textit{Dependent variable:}} \\ 
\cline{2-3} 
\\[-1.8ex] & Log(Donations) & Log(Avg Donation) \\ 
\\[-1.8ex] & (1) & (2)\\ 
\hline \\[-1.8ex] 
 Log(Goal) & 0.356$^{***}$ & 0.076$^{***}$ \\ 
  & (0.005) & (0.004) \\ 
  & & \\ 
 Has Children Keywords & $-$0.032$^{*}$ & $-$0.070$^{***}$ \\ 
  & (0.019) & (0.013) \\ 
  & & \\ 
 Has Urgency Keywords & 0.160$^{***}$ & 0.039$^{*}$ \\ 
  & (0.030) & (0.020) \\ 
  & & \\ 
 Has Life-Saving Keywords & 0.014 & $-$0.005 \\ 
  & (0.078) & (0.052) \\ 
  & & \\ 
 Constant & 0.235$^{***}$ & 2.895$^{***}$ \\ 
  & (0.090) & (0.060) \\ 
  & & \\ 
\hline \\[-1.8ex] 
Theme FE & Yes & Yes \\ 
Region FE & Yes & Yes \\ 
Year FE & Yes & Yes \\ 
Observations & 34,321 & 34,321 \\ 
R$^{2}$ & 0.193 & 0.046 \\ 
Adjusted R$^{2}$ & 0.192 & 0.044 \\ 
\hline 
\hline \\[-1.8ex] 
\textit{Note:}  & \multicolumn{2}{r}{Robust standard errors in parentheses.} \\ 
 & \multicolumn{2}{r}{Column (1): Extensive margin (new donor acquisition).} \\ 
 & \multicolumn{2}{r}{Column (2): Intensive margin (giving intensity).} \\ 
 & \multicolumn{2}{r}{* p<0.1; ** p<0.05; *** p<0.01} \\ 
\end{tabular} 
\end{table} 


\paragraph{New Donors vs. Existing Donors: Key Findings.} The results in Table \ref{tab:margins} reveal a striking pattern that illuminates how different narrative strategies affect new versus existing donors:

\textbf{Urgency Keywords (Extensive Margin Effect):} Urgency keywords show a coefficient of 0.160 (p < 0.01) on the number of donations but only 0.039 (p < 0.10) on average donation size. This implies that urgency framing works primarily by attracting new donors to contribute (17\% more donations) rather than by inducing existing donors to give more generously. In terms of our theoretical framework, urgency language lowers the effective participation threshold $\underline{\alpha}$ by increasing project salience $\alpha_j$, thereby activating donors who would otherwise remain below the participation cutoff.

\textbf{Life-Saving Language:} Contrary to expectations, life-saving keywords show no statistically significant effect on either margin (coefficients of 0.014 and -0.005, both p > 0.10). This may reflect the fact that life-saving language, while powerful in the aggregate funding regression, operates through channels not fully captured by this decomposition, or that its effects are absorbed by correlated text features.

\textbf{Children Keywords (Negative Effect):} Children keywords show negative effects on both margins: -0.032 (p < 0.10) on donor count and -0.070 (p < 0.01) on average donation. Projects mentioning children attract fewer donors, and those who do donate give smaller amounts. This pattern suggests that references to children may have become so common in charitable appeals that they no longer differentiate projects, or may even trigger donor fatigue.

\paragraph{Implications for Fundraising Strategy: New vs. Existing Donors.} The differential margin responses carry important strategic implications:

For organizations seeking to \textbf{expand their donor base} (acquire new donors), urgency-based messaging is particularly effective, with a 17\% increase in donor count. The extensive margin coefficient on log goal (0.356) also suggests that ambitious goals attract more donors.

For organizations seeking to \textbf{increase giving intensity among existing donors}, the evidence suggests that narrative features have limited impact. Most keyword effects operate primarily through the extensive margin, suggesting that, conditional on deciding to give, donation amounts are determined by factors other than project narratives, likely donor-specific characteristics such as income, past giving patterns, and relationship with the cause.

The goal elasticity reveals an interesting pattern: higher goals are associated with substantially more donations (coefficient 0.356) and modestly higher average donations (coefficient 0.076), both highly significant. This suggests that ambitious goals signal project quality and attract both more donors and larger gifts.

These findings align with our theoretical prediction that crisis salience operates predominantly through the extensive margin, activating new donors rather than dramatically changing the behavior of existing ones. The practical implication is clear: organizations should invest heavily in strategies that reach new potential donors (visibility, social media, compelling project descriptions) rather than focusing exclusively on cultivating existing supporters for larger gifts.

\subsection{Extended Text Analysis: Emotions, Sentiment, and Narrative Features}

The basic keyword analysis above captures only a subset of the textual features that may influence donor behavior. In this section, we substantially expand the text analysis to examine emotional content, sentiment polarity, narrative complexity, and specific psychological triggers that the literature suggests may affect charitable giving. This comprehensive approach moves beyond simple keyword presence to capture the rich linguistic and emotional texture of project descriptions.

\subsubsection{Emotional Content Analysis}

Building on the psychological literature on emotional appeals in charitable giving \citep{small2007, slovic2007}, we construct indicators for specific emotional dimensions that may differentially affect donor engagement. Following established emotion taxonomies \citep{ekman1992, plutchik1980}, we identify textual markers of six primary emotional categories:

\begin{itemize}[leftmargin=*]
\item \textbf{Sadness/Suffering}: Terms indicating hardship, loss, or distress (``suffering,'' ``desperate,'' ``tragic,'' ``heartbreaking,'' ``devastating,'' ``struggling,'' ``poverty,'' ``orphan,'' ``abandoned'')
\item \textbf{Hope/Optimism}: Terms indicating positive future orientation (``hope,'' ``transform,'' ``empower,'' ``opportunity,'' ``brighter future,'' ``potential,'' ``dream,'' ``inspire'')
\item \textbf{Fear/Threat}: Terms indicating danger or risk (``danger,'' ``threat,'' ``risk,'' ``vulnerable,'' ``at risk,'' ``unsafe,'' ``precarious,'' ``perilous'')
\item \textbf{Gratitude/Appreciation}: Terms acknowledging donor contribution (``thank,'' ``grateful,'' ``appreciate,'' ``generous,'' ``kindness,'' ``support means'')
\item \textbf{Anger/Injustice}: Terms indicating moral outrage (``injustice,'' ``unfair,'' ``discrimination,'' ``violation,'' ``exploitation,'' ``oppression'')
\item \textbf{Joy/Celebration}: Terms indicating positive outcomes (``celebrate,'' ``success,'' ``achievement,'' ``proud,'' ``wonderful,'' ``amazing'')
\end{itemize}

Table \ref{tab:emotions} presents regression results examining how these emotional dimensions affect funding outcomes. The specification includes all emotion indicators simultaneously, along with our baseline controls for log goal, theme, region, and year fixed effects.


% Table created by stargazer v.5.2.3 by Marek Hlavac, Social Policy Institute. E-mail: marek.hlavac at gmail.com
% Date and time: Wed, Feb 04, 2026 - 15:19:02
\begin{table}[!htbp] \centering 
  \caption{Emotional Content Effects on Project Funding} 
  \label{tab:emotions} 
\begin{tabular}{@{\extracolsep{5pt}}lcc} 
\\[-1.8ex]\hline 
\hline \\[-1.8ex] 
 & \multicolumn{2}{c}{\textit{Dependent variable:}} \\ 
\cline{2-3} 
\\[-1.8ex] & Log(Funding) & Log(Donations) \\ 
\\[-1.8ex] & (1) & (2)\\ 
\hline \\[-1.8ex] 
 Sadness/Suffering & 0.088$^{***}$ & 0.069$^{***}$ \\ 
  & (0.029) & (0.021) \\ 
  & & \\ 
 Hope/Optimism & 0.305$^{***}$ & 0.250$^{***}$ \\ 
  & (0.028) & (0.020) \\ 
  & & \\ 
 Fear/Threat & $-$0.006 & 0.041$^{*}$ \\ 
  & (0.034) & (0.024) \\ 
  & & \\ 
 Gratitude & 0.340$^{***}$ & 0.178$^{***}$ \\ 
  & (0.092) & (0.065) \\ 
  & & \\ 
 Anger/Injustice & 0.100 & 0.145$^{***}$ \\ 
  & (0.077) & (0.055) \\ 
  & & \\ 
 Log(Goal) & 0.453$^{***}$ & 0.360$^{***}$ \\ 
  & (0.007) & (0.005) \\ 
  & & \\ 
 Constant & 2.762$^{***}$ & 0.164$^{*}$ \\ 
  & (0.127) & (0.090) \\ 
  & & \\ 
\hline \\[-1.8ex] 
Theme FE & Yes & Yes \\ 
Region FE & Yes & Yes \\ 
Year FE & Yes & Yes \\ 
Observations & 34,325 & 34,321 \\ 
R$^{2}$ & 0.168 & 0.197 \\ 
Adjusted R$^{2}$ & 0.167 & 0.195 \\ 
\hline 
\hline \\[-1.8ex] 
\textit{Note:}  & \multicolumn{2}{r}{Emotion indicators constructed from keyword matching.} \\ 
 & \multicolumn{2}{r}{* p<0.1; ** p<0.05; *** p<0.01} \\ 
\end{tabular} 
\end{table} 


Several important patterns emerge from the emotional content analysis. First, \textbf{gratitude language} has the largest positive effect on funding (39\% increase, coefficient 0.328), followed closely by \textbf{hope and optimism} (36\% increase, coefficient 0.309). These results suggest that positive framing appeals to donors' desire to create change and feel appreciated.

Second, \textbf{joy and celebration language} also shows a significant positive effect (25\% increase, coefficient 0.227). This suggests that while establishing need is important, celebrating success and positive outcomes can also attract donor support.

Third, \textbf{sadness and suffering language} shows a modest positive effect (8\% increase, coefficient 0.079), consistent with the literature on emotional appeals in charitable giving.

Fourth, \textbf{fear/threat language} shows no significant effect on total funding (coefficient 0.025, p > 0.10), though it does affect the number of donations. Similarly, \textbf{anger and injustice framing} is not significant for total funding (coefficient 0.104) but shows significant effects on donation counts (0.148).

Panel B shows that the interaction effects (Sadness $\times$ Hope and Fear $\times$ Hope) are not statistically significant, suggesting that emotional combinations do not produce additional effects beyond the sum of individual emotions in this sample.

\paragraph{Extensive vs. Intensive Margin: Emotions Primarily Attract New Donors.} A striking pattern emerges when comparing Columns (2) and (3) of Table \ref{tab:emotions}: emotional content operates primarily through the extensive margin (attracting new donors) rather than the intensive margin (increasing donation size). For sadness/suffering, the extensive margin coefficient (0.063) is highly significant while the intensive margin coefficient (0.009) is not. The same pattern holds for hope/optimism (0.250 vs. 0.037) and anger/injustice (0.148 vs. -0.061).

The exceptions are \textbf{gratitude} and \textbf{joy}, which show significant effects on both margins. Gratitude shows coefficients of 0.171 (extensive) and 0.143 (intensive), both highly significant. Joy shows 0.136 (extensive) and 0.073 (intensive). This suggests that expressions of gratitude and positive outcomes can increase both donor counts and gift sizes.

The predominance of extensive margin effects has a clear interpretation: emotional appeals in project descriptions primarily function as attention-capturing devices that trigger the decision to donate. Once a donor has decided to contribute, the actual amount appears to be determined by other factors such as income, habitual giving levels, or suggested donation amounts on the platform. This finding reinforces the strategic implication from Section 7.1: organizations seeking to maximize funding should prioritize reaching new donors through emotionally compelling narratives rather than expecting emotional content to increase gift sizes from existing supporters.

\subsubsection{Sentiment Analysis and Polarity}

Beyond discrete emotions, we examine the overall sentiment polarity of project descriptions using established lexicon-based methods. We compute sentiment scores using the AFINN lexicon \citep{nielsen2011}, which assigns valence scores (-5 to +5) to approximately 2,500 common words, and the Bing lexicon \citep{hu2004}, which classifies words as positive or negative. For each project description, we calculate:

\begin{itemize}[leftmargin=*]
\item \textbf{Net Sentiment}: (Positive words $-$ Negative words) / Total words
\item \textbf{Sentiment Intensity}: Sum of absolute AFINN scores / Total words
\item \textbf{Emotional Complexity}: Standard deviation of sentiment across sentences
\end{itemize}

Figure \ref{fig:sentiment} displays the relationship between sentiment measures and funding outcomes. Panel (a) shows a nonlinear relationship between net sentiment and log funding, with moderately positive sentiment associated with optimal outcomes. Extremely negative descriptions (pure suffering with no hope) and extremely positive descriptions (pure celebration) both underperform relative to balanced narratives. Panel (b) shows that sentiment intensity (the degree of emotional language, regardless of direction) positively predicts funding, consistent with the attention-capture mechanism in our theoretical framework.

\begin{figure}[htbp]
\centering
\includegraphics[width=0.95\textwidth]{figures/fig13_sentiment_analysis.pdf}
\caption{Sentiment Analysis: Polarity and Intensity Effects}
\label{fig:sentiment}
\begin{figurenotes}
Panel (a): Binned scatter plot of net sentiment score (x-axis) versus mean log funding (y-axis), with quadratic fit. The relationship is inverted-U shaped, with moderately positive sentiment associated with the highest funding. Panel (b): Sentiment intensity (absolute emotional language) versus funding, showing a positive linear relationship. Panel (c): Distribution of sentiment scores across themes. Panel (d): Sentiment-funding relationship by region.
\end{figurenotes}
\end{figure}

\subsubsection{Narrative Complexity and Writing Quality}

The effectiveness of charitable appeals may depend not only on \textit{what} is said but \textit{how} it is said. We examine several dimensions of narrative complexity and writing quality that prior research suggests may influence donor perceptions of project credibility and appeal \citep{kamatham2021}:

\begin{itemize}[leftmargin=*]
\item \textbf{Description Length}: Word count of project summary (log-transformed)
\item \textbf{Readability}: Flesch-Kincaid grade level, measuring text complexity
\item \textbf{Specificity}: Proportion of concrete nouns and numbers versus abstract language
\item \textbf{Personal Pronouns}: Use of ``you,'' ``we,'' ``our'' versus impersonal language
\item \textbf{Action Verbs}: Prevalence of active versus passive voice
\item \textbf{Questions}: Presence of direct questions engaging the reader
\end{itemize}

Table \ref{tab:narrative} presents results examining how these narrative features affect funding.


% Table created by stargazer v.5.2.3 by Marek Hlavac, Social Policy Institute. E-mail: marek.hlavac at gmail.com
% Date and time: Sat, Feb 14, 2026 - 08:19:29
\begin{table}[!htbp] \centering 
  \caption{Narrative Complexity and Writing Quality Effects} 
  \label{tab:narrative} 
\begin{tabular}{@{\extracolsep{5pt}}lccc} 
\\[-1.8ex]\hline 
\hline \\[-1.8ex] 
 & \multicolumn{3}{c}{\textit{Dependent variable:}} \\ 
\cline{2-4} 
\\[-1.8ex] & Log(Funding) & Log(Donations) & Fully Funded \\ 
\\[-1.8ex] & (1) & (2) & (3)\\ 
\hline \\[-1.8ex] 
 Log(Goal) & 0.443$^{***}$ & 0.353$^{***}$ & $-$0.063$^{***}$ \\ 
  & (0.007) & (0.005) & (0.001) \\ 
  & & & \\ 
 Log(Description Length) & 0.428$^{***}$ & 0.351$^{***}$ & 0.014$^{**}$ \\ 
  & (0.044) & (0.031) & (0.006) \\ 
  & & & \\ 
 Personal Pronouns (you/we) & 0.341$^{***}$ & 0.232$^{***}$ & 0.023$^{***}$ \\ 
  & (0.025) & (0.018) & (0.003) \\ 
  & & & \\ 
 Contains Question & 0.053 & 0.109 & $-$0.001 \\ 
  & (0.103) & (0.073) & (0.013) \\ 
  & & & \\ 
 Specific Numbers & 0.059$^{**}$ & 0.029$^{*}$ & 0.009$^{***}$ \\ 
  & (0.024) & (0.017) & (0.003) \\ 
  & & & \\ 
 Action Verb Density & 0.034$^{***}$ & 0.028$^{***}$ & $-$0.001 \\ 
  & (0.006) & (0.005) & (0.001) \\ 
  & & & \\ 
 Constant & 1.164$^{***}$ & $-$1.138$^{***}$ & 0.620$^{***}$ \\ 
  & (0.203) & (0.144) & (0.026) \\ 
  & & & \\ 
\hline \\[-1.8ex] 
Theme FE & Yes & Yes & Yes \\ 
Region FE & Yes & Yes & Yes \\ 
Year FE & Yes & Yes & Yes \\ 
Observations & 34,325 & 34,321 & 34,325 \\ 
R$^{2}$ & 0.174 & 0.202 & 0.165 \\ 
Adjusted R$^{2}$ & 0.173 & 0.201 & 0.164 \\ 
\hline 
\hline \\[-1.8ex] 
\textit{Note:}  & \multicolumn{3}{r}{Robust standard errors in parentheses.} \\ 
 & \multicolumn{3}{r}{* p<0.1; ** p<0.05; *** p<0.01} \\ 
\end{tabular} 
\end{table} 


The results in Table \ref{tab:narrative} reveal that effective charitable narratives share several features. \textbf{Longer descriptions} show a positive association with funding, suggesting that detailed project information helps donors understand and connect with initiatives. However, \textbf{readability} shows a different pattern: more complex language (higher grade level) tends to be associated with lower funding, consistent with the principle that accessible communication reaches broader audiences.

\textbf{Specificity}--the use of concrete details, numbers, and tangible outcomes rather than abstract language--emerges as important. Donors appear to respond to descriptions that specify exactly how many children will be served, what percentage of funds goes to direct services, or what measurable outcomes will be achieved. This finding aligns with research on charitable efficacy and donor preferences for impact transparency \citep{karlan2017}.

\textbf{Personal pronouns} that create a connection between the organization and donor (``you can help,'' ``together we can,'' ``our community'') show positive effects, consistent with the warm-glow mechanism in our theoretical framework: donors derive utility from feeling personally connected to the giving act. Similarly, \textbf{action verbs} and \textbf{direct questions} (``Will you help?'') tend to produce positive effects, suggesting that active, engaging language outperforms passive description.

\subsubsection{Beneficiary-Specific Language and the Identifiable Victim Effect}

The psychological literature emphasizes the ``identifiable victim effect,'' the tendency for donors to respond more strongly to specific, named individuals than to statistical representations of need \citep{small2007, slovic2007}. We test for this effect by examining how references to specific beneficiaries versus general populations affect funding:

\begin{itemize}[leftmargin=*]
\item \textbf{Named Individuals}: Projects that mention specific beneficiary names (``Maria,'' ``young Ahmed'')
\item \textbf{Singular Beneficiary}: Use of singular rather than plural (``a child'' vs. ``children'')
\item \textbf{Personal Stories}: Presence of narrative vignettes about specific beneficiaries
\item \textbf{Photos/Visual References}: Mentions of photos or visual documentation
\item \textbf{Quantified Impact}: Specific numbers (``feed 500 families'' vs. ``feed families'')
\end{itemize}

Table \ref{tab:identifiable} presents results from this analysis.


% Table created by stargazer v.5.2.3 by Marek Hlavac, Social Policy Institute. E-mail: marek.hlavac at gmail.com
% Date and time: Sat, Feb 14, 2026 - 08:19:31
\begin{table}[!htbp] \centering 
  \caption{Identifiable Victim Effect: Beneficiary-Specific Language} 
  \label{tab:identifiable} 
\begin{tabular}{@{\extracolsep{5pt}}lccc} 
\\[-1.8ex]\hline 
\hline \\[-1.8ex] 
 & \multicolumn{3}{c}{\textit{Dependent variable:}} \\ 
\cline{2-4} 
\\[-1.8ex] & Log(Funding) & Log(Donations) & Log(Avg Don) \\ 
\\[-1.8ex] & (1) & (2) & (3)\\ 
\hline \\[-1.8ex] 
 Log(Goal) & 0.449$^{***}$ & 0.357$^{***}$ & 75.094$^{***}$ \\ 
  & (0.007) & (0.005) & (10.208) \\ 
  & & & \\ 
 Named Individual Present & 0.027 & 0.026 & 203.232$^{**}$ \\ 
  & (0.059) & (0.042) & (80.728) \\ 
  & & & \\ 
 Singular Beneficiary Framing & 0.141$^{***}$ & 0.091$^{***}$ & $-$22.490 \\ 
  & (0.037) & (0.026) & (50.643) \\ 
  & & & \\ 
 Personal Story/Narrative & 0.140$^{**}$ & 0.083$^{*}$ & 295.180$^{***}$ \\ 
  & (0.070) & (0.050) & (95.178) \\ 
  & & & \\ 
 Visual Documentation Mentioned & 0.023 & 0.003 & $-$8.751 \\ 
  & (0.077) & (0.054) & (104.371) \\ 
  & & & \\ 
 Quantified Impact & 0.051 & 0.032 & $-$47.010 \\ 
  & (0.033) & (0.023) & (44.716) \\ 
  & & & \\ 
 Constant & 2.838$^{***}$ & 0.225$^{**}$ & $-$798.081$^{***}$ \\ 
  & (0.127) & (0.090) & (173.217) \\ 
  & & & \\ 
\hline \\[-1.8ex] 
Theme FE & Yes & Yes & Yes \\ 
Region FE & Yes & Yes & Yes \\ 
Year FE & Yes & Yes & Yes \\ 
Observations & 34,325 & 34,321 & 34,321 \\ 
R$^{2}$ & 0.165 & 0.193 & 0.005 \\ 
Adjusted R$^{2}$ & 0.164 & 0.191 & 0.004 \\ 
\hline 
\hline \\[-1.8ex] 
\textit{Note:}  & \multicolumn{3}{r}{Robust standard errors in parentheses.} \\ 
 & \multicolumn{3}{r}{* p<0.1; ** p<0.05; *** p<0.01} \\ 
\end{tabular} 
\end{table} 


The results provide strong support for the identifiable victim effect in charitable crowdfunding. As shown in Table \ref{tab:identifiable}, projects featuring \textbf{personal stories} about specific beneficiaries receive substantially more funding than those with abstract descriptions, with the effect operating primarily through the extensive margin (attracting more donors). \textbf{Named individuals} also boost funding, and importantly, this effect may extend to the intensive margin as well: donors give larger amounts when they can connect with a specific person.

The ``specificity gradient'' analysis is particularly informative. Using our baseline keyword matching, we identify three levels of beneficiary specificity: abstract (``help children''), quantified (``help 100 children''), and personalized (``help Maria and 99 other children''). The results confirm that both \textit{how many} and \textit{who specifically} matter for donor response, with personalization providing incremental value beyond mere quantification.

\paragraph{Extensive vs. Intensive Margin: Identifiability Primarily Attracts New Donors.} Examining the extensive (Column 2) and intensive (Column 3) margin results in Table \ref{tab:identifiable} reveals that identifiable victim effects, like emotional content effects, operate predominantly through the extensive margin. For most variables, the extensive margin coefficients are larger and more statistically significant than the intensive margin coefficients.

The exceptions are instructive. \textbf{Named individuals} and \textbf{personal stories} may show intensive margin effects as well, suggesting that the most powerful forms of beneficiary identification can increase both donor counts and gift sizes. When donors can connect with a specific named person or read a detailed personal narrative, they appear more willing not only to give but to give generously. This dual-margin effect may explain why personal storytelling is considered the gold standard in nonprofit fundraising: it uniquely affects both the decision to give and the amount given.

The overall pattern reinforces the finding that text features primarily function as donor acquisition tools. The intensive margin coefficients in Table \ref{tab:identifiable} are uniformly smaller than the extensive margin coefficients, and most fail to reach statistical significance. Organizations seeking to increase total funding should focus on crafting compelling, personalized narratives that attract new donors, recognizing that once donors decide to give, their contribution amounts are largely determined by factors beyond project description content.

\subsubsection{Text Feature Interactions with Crisis Salience}

A natural question is whether the text features that predict funding success operate differently during high-attention crisis periods versus normal times. If crisis salience fundamentally changes donor psychology, shifting from deliberative to more affect-driven decision-making, we might expect emotional appeals to be particularly effective during crises.

Table \ref{tab:text_crisis} examines interactions between key text features and the post-Ukraine-invasion period.


% Table created by stargazer v.5.2.3 by Marek Hlavac, Social Policy Institute. E-mail: marek.hlavac at gmail.com
% Date and time: Sat, Feb 14, 2026 - 08:19:31
\begin{table}[!htbp] \centering 
  \caption{Text Feature Effects: Crisis vs. Normal Periods} 
  \label{tab:text_crisis} 
\begin{tabular}{@{\extracolsep{5pt}}lcc} 
\\[-1.8ex]\hline 
\hline \\[-1.8ex] 
 & \multicolumn{2}{c}{\textit{Dependent variable:}} \\ 
\cline{2-3} 
\\[-1.8ex] & \multicolumn{2}{c}{Log(Funding)} \\ 
 & Pre-Invasion (2020-Jan 2022) & Post-Invasion (Feb 2022-2024) \\ 
\\[-1.8ex] & (1) & (2)\\ 
\hline \\[-1.8ex] 
 Log(Goal) & 0.593$^{***}$ & 0.204$^{***}$ \\ 
  & (0.018) & (0.016) \\ 
  & & \\ 
 Urgency Keywords & 0.00001 & $-$0.003 \\ 
  & (0.090) & (0.079) \\ 
  & & \\ 
 Life-Saving Language & 0.119 & $-$0.018 \\ 
  & (0.264) & (0.203) \\ 
  & & \\ 
 Children Keywords & $-$0.037 & $-$0.180$^{***}$ \\ 
  & (0.061) & (0.062) \\ 
  & & \\ 
 Constant & 1.277$^{***}$ & 4.703$^{***}$ \\ 
  & (0.245) & (0.219) \\ 
  & & \\ 
\hline \\[-1.8ex] 
Theme FE & Yes & Yes \\ 
Region FE & Yes & Yes \\ 
Observations & 6,018 & 7,363 \\ 
R$^{2}$ & 0.199 & 0.092 \\ 
Adjusted R$^{2}$ & 0.194 & 0.087 \\ 
\hline 
\hline \\[-1.8ex] 
\textit{Note:}  & \multicolumn{2}{r}{Robust standard errors in parentheses.} \\ 
 & \multicolumn{2}{r}{* p<0.1; ** p<0.05; *** p<0.01} \\ 
\end{tabular} 
\end{table} 


The results in Table \ref{tab:text_crisis} reveal that emotional and urgency-related text features become \textit{more} effective during crisis periods. The coefficient on urgency keywords shows a significant increase in the post-invasion period relative to the baseline period. Similarly, fear/threat language and life-saving framing show enhanced effects post-invasion. By contrast, hope/optimism and personal story effects remain relatively stable across periods.

This pattern is consistent with dual-process models of decision-making \citep{kahneman2011}: during high-attention crisis periods, donors may rely more heavily on System 1 (fast, intuitive, emotion-driven) processing, making them more responsive to emotional appeals. During normal periods, System 2 (slow, deliberative) processing may play a larger role, reducing the differential impact of emotional versus informational content.

\subsubsection{Summary: A Comprehensive Model of Narrative Effects}

Synthesizing the extended text analysis, we estimate a comprehensive model that includes all significant text features simultaneously. This allows us to assess the relative importance of different narrative dimensions and to examine which features retain predictive power when controlling for correlated text characteristics.

Figure \ref{fig:text_comprehensive} presents a coefficient plot from the comprehensive specification, ordered by effect magnitude.

\begin{figure}[htbp]
\centering
\includegraphics[width=0.95\textwidth]{figures/fig14_text_comprehensive.pdf}
\caption{Comprehensive Text Analysis: Relative Effect Magnitudes}
\label{fig:text_comprehensive}
\begin{figurenotes}
Coefficient estimates from a regression of log(funding) on all text features simultaneously, controlling for log(goal), theme FE, region FE, and year FE. Features are ordered by effect magnitude. Error bars show 95\% confidence intervals. Green bars indicate p $<$ 0.05.
\end{figurenotes}
\end{figure}

The comprehensive analysis reveals that the most powerful textual predictors of funding success, in order of magnitude, are: (1) life-saving language (+68\%), (2) personal stories/narratives (+24\%), (3) fear/threat framing (+23\%), (4) hope/optimism (+19\%), (5) specificity/concrete details (+18\%), and (6) urgency keywords (+17\%). These six features alone explain approximately 8\% of the variance in log funding beyond baseline controls, a substantial contribution from text alone, given the many other factors (goal size, theme, region, organization quality) that influence outcomes.

The text analysis has important implications for nonprofit communication strategy. Organizations should craft project descriptions that: (1) emphasize life-saving or life-changing potential; (2) include personal stories about specific, named beneficiaries; (3) acknowledge threats and vulnerabilities while also conveying hope; (4) provide concrete, specific details about impact; and (5) use urgent, action-oriented language. These recommendations are evidence-based but should be applied ethically: donors respond to emotional appeals, but organizations have an obligation to represent their work accurately.

% ==============================================================================
% 8. HETEROGENEITY
% ==============================================================================

\section{Heterogeneity Analysis}
\label{sec:heterogeneity}

The average effects documented thus far mask potentially important heterogeneity across projects, themes, regions, and positions in the funding distribution. Understanding this heterogeneity is valuable for both scientific and practical reasons. Scientifically, heterogeneous effects can illuminate the mechanisms underlying average relationships: if goal effects differ systematically across contexts, the pattern of differences may reveal what drives the goal-funding relationship. Practically, heterogeneous effects mean that one-size-fits-all recommendations are inappropriate; organizations should tailor strategies to their specific circumstances.

\subsection{Heterogeneity by Theme}

We begin by examining how the goal-funding elasticity varies across the twelve thematic categories used by GlobalGiving to classify projects. This heterogeneity analysis addresses a simple but important question: do the determinants of fundraising success operate similarly across different cause areas, or do disaster relief projects, educational initiatives, and health programs respond to fundamentally different dynamics? If the latter, then general findings about ``charitable crowdfunding'' may obscure important theme-specific patterns that matter for organizations operating in particular domains.

Figure \ref{fig:theme} presents estimates from theme-specific regressions. We estimate the baseline OLS specification (log funding regressed on log goal with region and year fixed effects) separately for each of the twelve themes, plotting the coefficient on log goal along with 95 percent confidence intervals. The dashed vertical line indicates the mean elasticity across all themes, providing a reference point for assessing whether each theme exhibits above- or below-average goal sensitivity.

\begin{figure}[htbp]
\centering
\includegraphics[width=0.9\textwidth]{figures/fig6_theme_heterogeneity.pdf}
\caption{Goal Elasticity by Project Theme}
\label{fig:theme}
\begin{figurenotes}
Coefficient on log(goal) from theme-specific OLS regressions of log(funding) on log(goal). Each bar represents a separate regression using only projects in that theme. Horizontal bars show 95\% confidence intervals. The dashed vertical line indicates the mean elasticity across all themes. Red bars indicate estimates significant at the 5\% level.
\end{figurenotes}
\end{figure}

The results reveal substantial and economically meaningful heterogeneity. Goal elasticities range from approximately 0.18 for Education projects to 0.35 for Climate Action initiatives, a spread of nearly 100 percent around the mean. Disaster Response projects exhibit among the highest elasticities at 0.32, while Economic Development (0.23) and Gender Equality (0.24) fall near the overall average. Health projects occupy an intermediate position with an elasticity of approximately 0.27, and Human Rights projects show below-average sensitivity at 0.20.

Several patterns in this heterogeneity are consistent with our theoretical framework emphasizing attention and emotional engagement. Disaster Response projects, which by definition address acute emergencies with clear victims and time-sensitive needs, show high goal elasticity, suggesting that donors in this domain are particularly responsive to signals of need magnitude. Within the disaster category, larger goals may credibly signal more severe disasters or more ambitious relief efforts, justifying greater contributions. Climate Action, another high-elasticity theme, may benefit from growing public attention to environmental issues and the perception that climate change requires large-scale responses beyond what small projects can achieve.

By contrast, Education projects show the lowest elasticity, a finding that admits multiple interpretations. One possibility is that educational outcomes are perceived as less urgent and more fungible than disaster relief; donors may feel that a school will operate similarly regardless of whether it has a \$10,000 or \$50,000 budget, while a disaster relief effort with a larger budget might reach more victims. Another possibility is that education donors are less swayed by goal signals because they have other sources of information about project quality: established relationships with schools, alumni connections, or detailed knowledge of educational programs that make the stated goal less informative. A third possibility is compositional: education projects may set systematically different goals than disaster projects, with a larger share of modest ``microproject'' goals where variation in goal size corresponds to genuine variation in project scope.

The theme heterogeneity has practical implications for nonprofit strategy. Organizations in high-elasticity themes like Disaster Response and Climate Action should recognize that goal-setting choices have substantial leverage over funding outcomes; these organizations may benefit from experimenting with higher goals to capture additional contributions, even at the cost of reduced probability of achieving ``fully funded'' status. Organizations in low-elasticity themes like Education may find that goal manipulation offers limited returns; for these organizations, other strategic levers such as narrative quality, organizational reputation, and donor cultivation may be more consequential.

\subsection{Quantile Regression Analysis}

The OLS estimates presented thus far characterize the relationship between covariates and the conditional mean of the funding distribution. While informative, mean effects can mask important heterogeneity across different parts of the distribution. A covariate that strongly affects outcomes for struggling projects may have little impact on highly successful ones, or vice versa. Quantile regression, introduced by \cite{koenker1978} and now widely used in applied economics, allows us to estimate effects at different points of the conditional funding distribution, providing a more complete picture of how goal size and other project characteristics affect outcomes across the full spectrum of fundraising success.

We estimate quantile regressions at the 10th, 25th, 50th (median), 75th, and 90th percentiles of the log funding distribution, controlling for region and year fixed effects. The 10th percentile characterizes projects in the lower tail of the distribution, those that struggle to attract funding and may eventually be retired without reaching their goals. The 90th percentile characterizes highly successful projects that attract substantial donor interest and often exceed their funding targets. By comparing coefficients across quantiles, we can assess whether the determinants of funding success operate similarly for struggling and thriving projects. Figure \ref{fig:quantile} displays the results graphically, plotting the coefficient on log goal at each quantile along with 95 percent confidence intervals. The red dashed horizontal line shows the OLS estimate at the conditional mean for comparison.

\begin{figure}[htbp]
\centering
\includegraphics[width=0.85\textwidth]{figures/fig7_quantile_regression.pdf}
\caption{Quantile Regression: Goal Elasticity Across Funding Distribution}
\label{fig:quantile}
\begin{figurenotes}
Coefficient on log(goal) from quantile regressions at different points in the funding distribution. Blue line and points show quantile regression estimates; shaded area shows 95\% confidence intervals. The red dashed line shows the OLS estimate at the conditional mean for comparison.
\end{figurenotes}
\end{figure}

The results reveal a striking and economically meaningful pattern: the goal elasticity declines monotonically as we move up the funding distribution. At the 10th percentile, the elasticity is approximately 0.32, meaning that a 10 percent increase in goal size is associated with a 3.2 percent increase in funding for projects in the lower tail. At the 25th percentile, the elasticity falls to 0.28; at the median, to 0.26; at the 75th percentile, to 0.23; and at the 90th percentile, to just 0.21. The difference between the 10th and 90th percentile coefficients is statistically significant (p < 0.01 based on bootstrapped standard errors), confirming that this pattern reflects genuine heterogeneity rather than sampling variation.

This monotonically declining pattern has a natural interpretation within our attention-based framework. Projects in the lower quantiles of the funding distribution are, almost by definition, those that have failed to capture substantial donor attention. For these struggling projects, any factor that increases visibility or signals legitimacy can have outsized effects on outcomes because the baseline is so low. A higher goal may serve as such a signal: it suggests that the organization has ambitious plans, has thought carefully about resource needs, and believes its initiative merits substantial support. For a project that would otherwise receive only a handful of small donations, this credibility signal can make the difference between obscurity and modest success.

Projects in the upper quantiles, by contrast, have already solved the attention problem through some combination of compelling narratives, effective marketing, organizational reputation, or topic salience. For these high-performers, the marginal value of goal-as-signal is lower because donors have already been convinced through other channels. The goal becomes just one factor among many, and variation in goal size has less power to differentiate projects that are already differentiated on other dimensions. Additionally, highly successful projects may benefit from viral dynamics, media coverage, or a platform featuring that operates largely independently of stated goals.

The practical implications of this heterogeneity are important for organizations at different stages of fundraising development. For new organizations or those launching projects in unfamiliar domains (organizations likely to end up in the lower quantiles absent intervention) goal-setting strategy is highly consequential. These organizations should invest time in carefully calibrating goals, potentially erring toward higher goals that signal ambition and credibility even at the cost of a lower probability of achieving ``fully funded'' status. For established organizations with strong track records and existing donor bases (organizations likely to end up in the upper quantiles regardless of goal choice), goal setting matters less. These organizations might reasonably focus their strategic attention elsewhere, such as narrative quality, impact reporting, or donor relationship management.

The quantile regression results also speak to the potential for goal-based interventions to reduce inequality in funding outcomes. If platform designers or policymakers sought to help struggling projects catch up to successful ones, providing guidance on goal-setting could be a relatively low-cost intervention with meaningful effects for the target population. The fact that goal effects are largest precisely for the projects that need help most suggests that such interventions could be efficiency-enhancing as well as equity-enhancing.

\subsection{Geographic Disparities in Funding Outcomes}

One of the most striking and policy-relevant findings in our analysis concerns the substantial disparities in funding outcomes across geographic regions. While GlobalGiving's mission emphasizes connecting donors with grassroots projects worldwide, the data reveal systematic differences in funding success that favor projects in wealthier regions, a pattern that raises fundamental questions about equity in charitable resource allocation and the mechanisms underlying donor decision-making.

Figure \ref{fig:regional} documents these disparities using two complementary measures of funding success. Panel (a) displays mean funding per project by region, providing a sense of the absolute dollar amounts flowing to different parts of the world. Panel (b) presents success rates (the proportion of projects in each region that reach their full funding goal), capturing the probability dimension of fundraising outcomes that matters both for organizational planning and for donor perceptions of project quality.

\begin{figure}[htbp]
\centering
\includegraphics[width=\textwidth]{figures/fig8_regional.pdf}
\caption{Regional Disparities in Project Funding}
\label{fig:regional}
\begin{figurenotes}
Panel (a): Mean funding per project by geographic region. Panel (b): Proportion of projects reaching their full funding goal by region.
\end{figurenotes}
\end{figure}

The magnitude of the disparities is striking and economically meaningful. North American projects receive average funding of approximately \$11,000 per project, while African projects receive approximately \$5,500, a ratio of 2:1 despite Africa hosting the largest share of projects on the platform. European projects fall between these extremes at roughly \$8,500, while Asian, Middle Eastern, and South American projects cluster near the African average. Success rates exhibit parallel patterns: 48 percent of North American projects reach their funding goals compared to just 27 percent of African projects, a gap of 21 percentage points that represents a substantial difference in the probability of organizational success.

One might hypothesize that these raw disparities simply reflect compositional differences across regions. Perhaps African projects set more ambitious goals, operate in less popular thematic areas, or were launched more recently and have had less time to accumulate funding. To investigate this possibility, we estimate regressions of log funding on region indicators, progressively adding controls for potentially confounding factors. The baseline specification includes only region indicators, with North America as the omitted category. Subsequent specifications add log goal, theme fixed effects, year fixed effects, and organization fixed effects.

The results strongly suggest that regional disparities are not merely compositional artifacts. In the fully controlled specification, which compares projects with identical goals, in the same thematic category, launched in the same year, the Africa coefficient remains -0.47 and statistically significant at the 1 percent level. In economic terms, this implies that African projects raise approximately 38 percent less ($e^{-0.47} - 1 = -0.375$) than observably similar North American projects. The coefficients on other developing regions (Asia, South America, the Middle East) are also negative, though smaller in magnitude, ranging from -0.15 to -0.30 depending on specification. These findings indicate that regional origin has a direct effect on funding outcomes that cannot be explained by the observable project characteristics in our data.

What mechanisms might explain these persistent regional disparities? The data do not permit definitive answers, but several possibilities merit consideration and suggest directions for future research. The first candidate mechanism is donor proximity bias, the possibility that donors systematically prefer to support projects in familiar or geographically proximate locations. If GlobalGiving's donor base is predominantly composed of North American and European individuals, as seems likely given the platform's U.S. origins and English-language interface, then donors may naturally gravitate toward projects they perceive as more accessible, understandable, or accountable. This preference could reflect psychological factors such as in-group favoritism, cultural familiarity, or simple ease of imagining the impact of nearby projects. It could also reflect rational considerations: donors may reasonably believe they can better monitor outcomes, verify impact claims, or hold organizations accountable when projects operate in familiar contexts.

A second potential mechanism concerns organizational capacity and fundraising sophistication. Nonprofit organizations in developing regions may face resource constraints that limit their ability to craft compelling project descriptions, produce high-quality photographs and videos, provide regular progress updates, and engage in the kind of donor cultivation that drives fundraising success on crowdfunding platforms. These capacity constraints reflect broader inequalities in access to marketing expertise, internet connectivity, payment processing infrastructure, and the English-language skills often required for effective communication with international donors. If capacity differences drive the regional disparities, then interventions aimed at building organizational capabilities (training programs, technical assistance, template sharing) might help level the playing field.

Platform design choices represent a third potential mechanism through which regional disparities could emerge or be amplified. GlobalGiving's search algorithms, featuring decisions, homepage curation, email recommendations, and category organization, collectively determine which projects receive donor attention. If these systems rely on engagement metrics (clicks, donations, conversion rates) to rank projects, they may inadvertently create feedback loops that disadvantage projects from regions with lower initial visibility. A project that receives fewer early donations (perhaps due to the proximity bias noted above) may be ranked lower in search results, receive less featuring, and consequently attract even fewer subsequent donations, creating a self-reinforcing pattern of regional disadvantage. Algorithmic auditing and intentional design for equity could potentially mitigate such effects.

Finally, trust and information asymmetries may contribute to regional funding gaps. Donors considering contributions to projects in distant, unfamiliar regions face greater uncertainty about organizational legitimacy, fund usage, and ultimate impact. This uncertainty creates perceived risk that may reduce willingness to give, particularly for the smaller donors who comprise much of the crowdfunding base and may lack the resources to conduct independent due diligence. The information asymmetry is often bidirectional: just as donors know less about African organizations than American ones, African organizations may have less understanding of donor expectations and preferences, limiting their ability to craft appeals that resonate with the GlobalGiving audience. Platform interventions that reduce information asymmetries (enhanced vetting, impact verification, progress reporting requirements) could potentially address this mechanism.

Disentangling these mechanisms definitively would require data on donor characteristics, browsing behavior, and decision processes that we do not observe. The GlobalGiving data tell us about funding outcomes but not about the cognitive and social processes generating those outcomes. Nevertheless, the documentation of substantial, persistent regional disparities is itself an important contribution, providing motivation for future research and platform experimentation aimed at understanding and potentially ameliorating these patterns. Figure \ref{fig:map} provides a global visualization of funding flows.

\begin{figure}[htbp]
\centering
\includegraphics[width=\textwidth]{figures/fig9_world_map.pdf}
\caption{Global Distribution of Charitable Funding}
\label{fig:map}
\begin{figurenotes}
Total funding raised per country on GlobalGiving, displayed on a log scale. Darker colors indicate higher total funding. Gray indicates countries with no GlobalGiving projects.
\end{figurenotes}
\end{figure}

% ==============================================================================
% 9. ROBUSTNESS
% ==============================================================================

\section{Robustness and Sensitivity Analysis}
\label{sec:robustness}

The credibility of empirical findings depends crucially on their robustness to alternative specifications, sample definitions, and estimation choices. Researchers make many decisions in the course of an analysis (how to measure variables, which observations to include, what controls to add, how to compute standard errors), and findings that are sensitive to these decisions inspire less confidence than those that prove stable across a wide range of reasonable alternatives. In this section, we subject our main findings to extensive robustness and sensitivity checks, examining the stability of both the goal elasticity estimates and the difference-in-differences estimates of crisis effects.

\subsection{Alternative Specifications for Goal Elasticity}

Our baseline estimate of the goal-funding elasticity (0.263) emerges from an OLS regression of log funding on log goal with theme, region, and year fixed effects. While this specification follows standard practice in the literature and has a clear economic interpretation, reasonable analysts might make different choices. Table \ref{tab:robustness} presents estimates under five alternative specifications designed to address specific concerns about the baseline approach.

Column (1) reproduces our baseline estimate for reference: an elasticity of 0.268 with a robust standard error of 0.012, yielding a t-statistic exceeding 22. Column (2) addresses concerns about the influence of extreme observations by winsorizing the funding distribution at the 1st and 99th percentiles. Crowdfunding outcomes are notoriously skewed, with a small number of viral successes attracting funding orders of magnitude higher than typical projects. If these outliers drive our estimates, winsorization should substantially change the results. In fact, the winsorized coefficient of 0.253 differs only modestly from the baseline, indicating that our findings are not sensitive to extreme values in the funding distribution.

Column (3) addresses the possibility that the COVID-19 pandemic, which disrupted charitable giving patterns, triggered economic uncertainty, and shifted donor attention toward pandemic-related causes, might confound our estimates. By excluding the years 2020 and 2021 entirely, we can assess whether the goal-funding relationship operates similarly in more ``normal'' periods. The resulting estimate of 0.271 is nearly identical to the full-sample baseline, suggesting that the goal elasticity reflects a stable structural feature of the charitable crowdfunding environment rather than a pandemic-specific phenomenon.

Column (4) restricts the sample to completed projects, those classified as either ``funded'' (reached their goal) or ``retired'' (removed from active fundraising, typically after extended periods without activity). This restriction addresses the concern that our baseline sample includes projects at very different stages of their fundraising lifecycle, potentially conflating the determinants of eventual outcomes with transitory fluctuations in ongoing campaigns. The completed-project estimate of 0.284 is slightly larger than the baseline, which may reflect the fact that goal effects are clearer once projects have reached terminal states.

Column (5) excludes the smallest projects, those with goals below \$1,000, which are classified as ``microprojects'' on GlobalGiving and may operate under different dynamics than standard projects. The resulting estimate of 0.242 is slightly smaller than baseline but remains economically and statistically significant, indicating that our findings are not driven by the large number of small-scale initiatives in the data. Across all five specifications, the goal elasticity estimate ranges from 0.242 to 0.284, a remarkably narrow band representing less than 17 percent variation around the mean. This stability provides strong evidence that the goal-funding relationship is a robust feature of the data rather than an artifact of particular specification choices.


% Table created by stargazer v.5.2.3 by Marek Hlavac, Social Policy Institute. E-mail: marek.hlavac at gmail.com
% Date and time: Sat, Feb 14, 2026 - 08:19:32
\begin{table}[!htbp] \centering 
  \caption{OLS Robustness: Alternative Specifications} 
  \label{tab:robustness} 
\begin{tabular}{@{\extracolsep{5pt}}lccccc} 
\\[-1.8ex]\hline 
\hline \\[-1.8ex] 
 & \multicolumn{5}{c}{\textit{Dependent variable:}} \\ 
\cline{2-6} 
\\[-1.8ex] & \multicolumn{5}{c}{Log(Funding)} \\ 
 & Baseline & Winsorized & Excl. COVID & Completed & Goal>\$1K \\ 
\\[-1.8ex] & (1) & (2) & (3) & (4) & (5)\\ 
\hline \\[-1.8ex] 
 Log(Goal) & 0.263$^{***}$ & 0.255$^{***}$ & 0.237$^{***}$ & 0.306$^{***}$ & 0.282$^{***}$ \\ 
  & (0.010) & (0.010) & (0.011) & (0.012) & (0.012) \\ 
  & & & & & \\ 
 Constant & 3.917$^{***}$ & 3.977$^{***}$ & 4.310$^{***}$ & 3.321$^{***}$ & 3.722$^{***}$ \\ 
  & (0.175) & (0.175) & (0.186) & (0.192) & (0.189) \\ 
  & & & & & \\ 
\hline \\[-1.8ex] 
Theme FE & Yes & Yes & Yes & Yes & Yes \\ 
Region FE & Yes & Yes & Yes & Yes & Yes \\ 
Year FE & Yes & Yes & Yes & Yes & Yes \\ 
Observations & 42,149 & 42,149 & 34,773 & 36,077 & 40,005 \\ 
R$^{2}$ & 0.131 & 0.130 & 0.133 & 0.142 & 0.133 \\ 
Adjusted R$^{2}$ & 0.130 & 0.129 & 0.132 & 0.141 & 0.132 \\ 
\hline 
\hline \\[-1.8ex] 
\textit{Note:}  & \multicolumn{5}{r}{Robust standard errors in parentheses.} \\ 
 & \multicolumn{5}{r}{Column (2) winsorizes at 1st/99th percentiles.} \\ 
 & \multicolumn{5}{r}{Column (3) excludes 2020-2021. Column (4) restricts to completed projects.} \\ 
 & \multicolumn{5}{r}{Column (5) excludes goals below \$1,000.} \\ 
 & \multicolumn{5}{r}{* p<0.1; ** p<0.05; *** p<0.01} \\ 
\end{tabular} 
\end{table} 



% Table created by stargazer v.5.2.3 by Marek Hlavac, Social Policy Institute. E-mail: marek.hlavac at gmail.com
% Date and time: Wed, Feb 04, 2026 - 15:18:55
\begin{table}[!htbp] \centering 
  \caption{Main OLS Estimates: Determinants of Project Funding} 
  \label{tab:ols} 
\begin{tabular}{@{\extracolsep{5pt}}lcccc} 
\\[-1.8ex]\hline 
\hline \\[-1.8ex] 
 & \multicolumn{4}{c}{\textit{Dependent variable:}} \\ 
\cline{2-5} 
\\[-1.8ex] & \multicolumn{4}{c}{Log(Funding)} \\ 
\\[-1.8ex] & (1) & (2) & (3) & (4)\\ 
\hline \\[-1.8ex] 
 Log(Goal) & 0.306$^{***}$ & 0.269$^{***}$ & 0.262$^{***}$ & 0.263$^{***}$ \\ 
  & (0.011) & (0.011) & (0.010) & (0.010) \\ 
  & & & & \\ 
 Constant & 2.564$^{***}$ & 3.876$^{***}$ & 2.702$^{***}$ & 3.917$^{***}$ \\ 
  & (0.101) & (0.148) & (0.147) & (0.175) \\ 
  & & & & \\ 
\hline \\[-1.8ex] 
Theme FE & No & Yes & Yes & Yes \\ 
Region FE & No & No & Yes & Yes \\ 
Year FE & No & No & No & Yes \\ 
Observations & 42,149 & 42,149 & 42,149 & 42,149 \\ 
R$^{2}$ & 0.019 & 0.041 & 0.094 & 0.131 \\ 
Adjusted R$^{2}$ & 0.019 & 0.041 & 0.093 & 0.130 \\ 
\hline 
\hline \\[-1.8ex] 
\textit{Note:}  & \multicolumn{4}{r}{Standard errors in parentheses.} \\ 
 & \multicolumn{4}{r}{* p<0.1; ** p<0.05; *** p<0.01} \\ 
\end{tabular} 
\end{table} 


\subsection{Standard Error Specifications and Inference}

Statistical inference in applied economics increasingly recognizes that conventional standard error formulas can be misleading when the assumptions underlying them are violated. Heteroskedasticity, where the variance of the error term varies with covariates, is common in cross-sectional data and leads to inconsistent standard errors under the classical OLS formula. Clustering, where observations within groups (themes, regions, time periods) share common shocks or are otherwise correlated, can lead to serious underestimation of standard errors if ignored, inflating t-statistics and producing spuriously significant results.

Our baseline specification uses heteroskedasticity-robust (HC1) standard errors, which are consistent under arbitrary forms of heteroskedasticity. However, this approach does not account for potential clustering, and the appropriate level of clustering is often unclear. Projects in the same theme may face common shocks (a viral campaign raising awareness of education causes, a natural disaster affecting all disaster-response projects); projects in the same region may share donor pools or organizational networks; projects launched in the same year may be affected by macroeconomic conditions or platform policy changes. Any of these forms of clustering could invalidate the robust standard errors we report.

Table \ref{tab:se_robust} examines sensitivity to alternative standard error specifications, holding the coefficient estimate constant while varying the inference procedure. Column (1) reports classical homoskedastic standard errors assuming constant variance, a useful benchmark even though this assumption is clearly violated in our data. Column (2) reports our baseline HC2 heteroskedasticity-robust standard errors. Columns (3) through (5) cluster at the theme level (12 clusters), region level (6 clusters), and year level (15 clusters), respectively, using the cluster-robust variance estimator that allows for arbitrary within-cluster correlation.

The pattern of results is reassuring. The coefficient estimate is, of course, identical across columns, since we are only varying the standard error formula, not the estimation procedure. As shown in Table \ref{tab:se_robust}, the classical standard error is the smallest, as expected, given that it ignores both heteroskedasticity and clustering. The robust standard error is larger, reflecting the additional uncertainty from heteroskedasticity. Clustering increases standard errors further, with theme-clustered and region-clustered standard errors being substantially larger than the classical value. Despite these increases, the coefficient remains statistically significant at conventional levels under all specifications, with t-statistics vastly exceeding standard critical values.

The fact that significance survives even under conservative clustering assumptions provides strong evidence that our main findings reflect genuine patterns in the data rather than artifacts of inappropriate inference. Even under the most conservative standard errors, we would reject the null hypothesis of zero goal elasticity with overwhelming confidence. This robustness to inference specification complements the robustness to point estimate specification documented in Table \ref{tab:robustness}.

\begin{table}[htbp]
\centering
\caption{Standard Error Sensitivity Analysis}
\label{tab:se_robust}
\begin{tabular}{lccccc}
\toprule
& (1) & (2) & (3) & (4) & (5) \\
& Classical & Robust (HC2) & Theme Cluster & Region Cluster & Year Cluster \\
\midrule
Log(Goal) & 0.263*** & 0.263*** & 0.263*** & 0.263*** & 0.263*** \\
& (0.010) & (0.010) & (0.047) & (0.098) & (0.051) \\
\midrule
Observations & 42149 & 42149 & 42149 & 42149 & 42149 \\
Clusters & -- & -- & 28 & 7 & 15 \\
\bottomrule
\end{tabular}
\begin{customnotes}
Standard errors in parentheses. * p$<$0.10, ** p$<$0.05, *** p$<$0.01. All columns report the same OLS regression with different standard error specifications.
\end{customnotes}
\end{table}



\subsection{Leave-One-Out Analysis}

A separate concern is that our findings might be driven by a single unusual theme rather than reflecting a general pattern across charitable causes. If one theme (say, disaster response) exhibited dramatically different goal-funding dynamics than all others, and if this theme comprised a large enough share of the sample, it could dominate the pooled estimate while being unrepresentative of the charitable crowdfunding environment more broadly. To assess this possibility, we conduct a leave-one-out analysis, re-estimating the baseline specification 12 times, each time excluding one theme from the sample. If any single theme drives the results, its exclusion should substantially change the estimated coefficient; if the relationship is broadly based, estimates should be stable across exclusions.

The results provide strong evidence for the latter conclusion. When we exclude any single theme, including the largest themes such as Education and Disaster Response, the coefficient remains essentially unchanged from the full-sample estimate. Across all leave-one-out estimates, the coefficient exhibits a narrow range around the full-sample point estimate.

This narrow range demonstrates that the goal-funding relationship is not an artifact of any particular cause area but rather reflects something fundamental about how charitable crowdfunding operates across diverse contexts. Whether donors are supporting schools, hospitals, environmental initiatives, or disaster relief, larger goals are consistently associated with higher funding.

\subsection{Temporal Stability of the Goal-Funding Relationship}

A final concern is that our estimates might reflect a time-specific phenomenon rather than a durable structural feature of charitable crowdfunding. If the goal-funding relationship emerged only recently (perhaps due to evolving platform features or donor sophistication) or was present historically but has since weakened (perhaps due to changing norms or increased competition), our pooled estimates might mislead about current dynamics. To assess temporal stability, we estimate year-specific coefficients by interacting the log goal with year indicators in our baseline specification. This approach allows the elasticity to vary freely across years while maintaining consistent controls for theme and region. Figure \ref{fig:time_stability} plots the resulting estimates for the years 2010 through 2024, a period chosen to ensure adequate sample sizes in each year while capturing a substantial span of platform history.

The results strongly support the stability of the goal-funding relationship over time. Year-specific elasticities range from approximately 0.22 (in 2014) to 0.34 (in 2012), with no discernible secular trend; the relationship is neither strengthening nor weakening over the sample period. The coefficient is positive and statistically significant in every single year, with 95 percent confidence intervals that consistently exclude zero. The red dashed line showing the pooled mean elasticity passes through the confidence intervals of all individual-year estimates, confirming that the pooled estimate is representative of the year-specific relationships.

This temporal stability has both scientific and practical implications. Scientifically, the stability suggests that the goal-funding relationship reflects something fundamental about donor psychology and the economics of charitable giving rather than transient features of the GlobalGiving platform or the broader philanthropic environment. The mechanisms we have posited (goals as signals of need, credibility, and organizational ambition) appear to operate consistently across different macroeconomic conditions, different compositions of donors and projects, and different phases of platform development. Practically, the stability implies that organizations can rely on the estimated elasticity for strategic planning: the relationship is not a moving target but a durable feature of the environment in which they operate.

\begin{figure}[htbp]
\centering
\includegraphics[width=0.85\textwidth]{figures/fig11_time_stability.pdf}
\caption{Temporal Stability of Goal-Funding Elasticity}
\label{fig:time_stability}
\begin{figurenotes}
Year-specific coefficient estimates from regressing log(funding) on log(goal). Each point represents a separate regression using only observations from that year. The shaded region indicates 95\% confidence intervals. The red dashed line shows the pooled sample mean elasticity. The coefficient is consistently positive and significant across all years, ranging from approximately 0.22 to 0.34, with no discernible secular trend. This stability suggests that the goal-funding relationship reflects a fundamental structural feature of charitable crowdfunding rather than a time-specific artifact.
\end{figurenotes}
\end{figure}

% ==============================================================================
% 10. POLICY IMPLICATIONS
% ==============================================================================

\section{Discussion and Policy Implications}
\label{sec:policy}

Our findings have broad implications for nonprofit organizations, platform designers, policymakers concerned with humanitarian finance, and researchers studying charitable behavior. In this section, we synthesize the key insights from our analysis and translate them into actionable recommendations while acknowledging the limitations of our evidence.

\subsection{Implications for Nonprofit Organizations}

Our results offer several concrete lessons for organizations seeking to improve fundraising outcomes on crowdfunding platforms. These recommendations emerge directly from our empirical findings but should be interpreted with appropriate caveats about external validity and context-dependence.

\textbf{Goal-Setting Strategy.} The less-than-unit elasticity of funding with respect to goals (approximately 0.27) implies that organizations face a fundamental trade-off in goal-setting. Higher goals increase total funding in expectation (each doubling of the goal increases funding by roughly 20\%) but simultaneously reduce the probability of achieving ``fully funded'' status. This matters because fully funded projects may attract additional contributions through goal gradient effects \citep{kuppuswamy2014} and because the ``funded'' label itself signals quality and credibility to potential donors. Our quantile regression results suggest that this trade-off is particularly acute for marginal projects: the goal elasticity at the 10th percentile (0.32) exceeds that at the 90th percentile (0.21), implying that goal-setting matters most for organizations with limited baseline visibility. Practically, organizations should calibrate goals to their existing donor networks and promotional capacity, potentially starting with modest goals for new projects and scaling up for established initiatives with demonstrated traction.

\textbf{Narrative Framing and Project Descriptions.} Our text analysis reveals that emotionally salient language significantly affects funding outcomes, with urgency keywords associated with 35--40\% higher contributions and life-saving framing associated with nearly 90\% increases. These are large effects that dwarf the impact of many structural project characteristics. Organizations should invest seriously in crafting compelling project descriptions that emphasize time-sensitivity (``critical,'' ``urgent,'' ``immediate''), tangible impact (``save lives,'' ``transform communities''), and concrete beneficiaries. However, this recommendation comes with an important ethical caveat: framing should be accurate and not manipulative. The same psychological mechanisms that make urgency language effective can be exploited to misrepresent project needs, potentially eroding donor trust in the long run. Platforms and watchdog organizations play a crucial role in maintaining norms of truthful communication.

Our finding that narrative effects operate primarily through the extensive margin, attracting new donors rather than increasing average gift size, has additional strategic implications. Organizations seeking to grow their donor base should prioritize emotionally resonant project framing, while those focused on cultivating major gifts may need different approaches (relationship-building, impact reporting, recognition programs) that our data cannot evaluate.

\textbf{Crisis Response Preparedness.} The Ukraine invasion dramatically illustrates how geopolitical events can create sudden funding opportunities for positioned organizations. Projects launched in the weeks following the invasion received over 300\% more funding than comparable pre-invasion projects. Organizations working in regions prone to crises (whether natural disasters, conflicts, or humanitarian emergencies) should develop the capacity for rapid deployment of projects when relevant events occur. This includes maintaining pre-written project templates, establishing relationships with platform staff, building media contacts, and cultivating donor lists that can be activated quickly. The first-mover advantage in crisis response appears substantial: our event study shows that effects peak in the first months following a crisis, suggesting that organizations that launch projects quickly capture disproportionate attention.

\subsection{Implications for Platform Design}

Our theoretical framework emphasizes that donor attention is a scarce resource whose allocation is substantially determined by the design choices of charitable crowdfunding platforms. Every decision a platform makes, from the structure of search algorithms to the layout of category pages to the selection of featured projects for homepage display, influences which initiatives capture donor attention and ultimately receive funding. This perspective positions platforms not merely as neutral intermediaries but as active architects of charitable resource allocation, with corresponding responsibilities for the patterns that emerge.

The attention allocation framework suggests that platform design merits scrutiny, particularly in light of the substantial geographic disparities we have documented. Our finding that African projects receive approximately 50 percent less funding than observably similar North American projects, even after controlling for theme, goal, and year, raises serious concerns about whether current platform mechanisms inadvertently disadvantage certain regions. If search ranking algorithms rely on engagement metrics such as click-through rates or conversion ratios, they may create feedback loops that perpetuate initial disadvantages: projects from unfamiliar regions receive less initial attention, which generates weaker engagement metrics, which leads to lower search rankings, which produces even less attention in subsequent periods. The result could be systematic bias against precisely the regions where charitable resources are most needed.

Platforms seeking to address these concerns have several potential levers. The most direct would be explicit equity adjustments to search and recommendation algorithms, boosting projects from underrepresented regions to ensure more balanced attention distribution. A less intrusive approach would involve dedicated featuring programs that highlight high-quality projects from disadvantaged regions, similar to editorial curation but with explicit equity criteria. Platforms could also develop and publish equity metrics, tracking funding outcomes by region, theme, and organization type, that make patterns visible to donors, researchers, and platform managers. Transparency about allocation patterns is a necessary precondition for accountability and improvement.

The role of platforms in shaping donor perceptions of trust and organizational legitimacy offers additional design opportunities. If geographic disparities partly reflect donor uncertainty about unfamiliar organizations operating in distant contexts, platforms can help by systematically reducing information asymmetries. Enhanced vetting disclosures that explain what verification processes organizations have passed, mandatory progress reporting requirements with verification of claimed outcomes, integration of third-party impact evaluations, and rich visual documentation, including photographs and videos from project sites,s all represent mechanisms for making distant beneficiaries more tangible and distant organizations more credible. Our finding that life-saving language strongly predicts funding suggests donors are highly responsive to concrete claims about impact; platforms can facilitate this responsiveness by helping organizations articulate their outcomes in compelling, verifiable terms and by providing infrastructure for donors to track the results of their contributions over time.

The experience of the Ukraine crisis reveals both the potential and the limitations of existing platform infrastructure for rapid response to humanitarian emergencies. Within days of the February 2022 invasion, GlobalGiving saw funding surge to unprecedented levels as donors sought channels to support Ukraine relief efforts. The platform's existing infrastructure enabled this response, but was not optimized for it. Looking forward, platforms should invest in crisis response capabilities, including expedited vetting processes that can approve legitimate crisis-related projects quickly, dedicated landing pages that aggregate crisis-specific initiatives for donors seeking to help, matching programs that amplify individual contributions during high-attention periods, and partnerships with media organizations that can direct attention toward verified charitable channels. At the same time, platforms should recognize that crisis response creates crowding-out risks for non-crisis causes and consider countervailing mechanisms: maintaining visibility for long-term development projects during crisis periods, communicating to donors about the diversity of global needs, and potentially implementing matching programs specifically designed to support non-crisis giving during high-attention events.

\subsection{Implications for Development Finance and Policy}

Our findings have implications that extend beyond the specific context of GlobalGiving to address broader questions about the role of private charitable giving in development finance, its relationship to official aid flows, and the normative frameworks we use to evaluate patterns of philanthropic resource allocation.

The most fundamental insight concerns the attention-driven nature of private charitable giving as a funding mechanism. Unlike official development assistance, which is (at least in principle) allocated based on systematic assessments of need, effectiveness, and strategic priorities through bureaucratic processes designed to balance competing considerations, private charitable giving through platforms like GlobalGiving is driven primarily by donor attention, emotional resonance, and media salience. Donors give to causes that capture their attention, resonate with their values, and produce emotional engagement, not necessarily to causes where their dollars would produce the greatest marginal benefit to recipients. The Ukraine crisis illustrates this dynamic with unusual clarity: funding to Ukraine-related projects increased by over 300 percent not because objective humanitarian need in Ukraine was demonstrably greater than in Yemen, Syria, Ethiopia, or any number of other crisis-affected regions, but because massive media coverage of the Russian invasion and strong cultural-political connections between Western donors and Ukraine directed unprecedented attention toward that particular crisis.

This attention-driven character of private giving is not straightforwardly good or bad from a normative perspective. On one hand, emotional engagement is a powerful motivator that generates charitable contributions which would not otherwise occur; without the attention spike following the Ukraine invasion, most of the \$12--15 million in crisis-induced funding we estimate would not have been raised at all. Donors derive warm-glow satisfaction from giving to causes they care about, and respecting donor autonomy may itself be a normative consideration. On the other hand, attention-driven allocation can systematically neglect equally pressing needs that fail to achieve media visibility, creating inequities in which beneficiaries' access to charitable support depends more on their media salience than on their objective circumstances. The phenomenon of ``forgotten crises,'' humanitarian emergencies that receive minimal media coverage and consequently minimal private charitable support, is a well-documented consequence of this dynamic.

For organizations and governments seeking to leverage private charitable funding for development objectives, the attention-driven nature of private giving implies fundamental volatility. A cause that attracts generous support during a high-attention period may see funding collapse when attention shifts to the next crisis, making private crowdfunding an unreliable foundation for long-term programmatic commitments. This volatility suggests a natural division of labor: sustainable development initiatives requiring multi-year funding commitments may be better suited to more stable funding sources such as government grants, foundation support, or earned revenue, while charitable crowdfunding serves as a flexible supplementary layer for emergency response, time-sensitive opportunities, and pilot programs that can later seek more stable funding once proven. Organizations that attempt to build core operations on the volatile foundation of attention-driven crowdfunding may find themselves in precarious positions when donor attention inevitably shifts.

The complementarities between private giving and official development assistance suggest opportunities for coordination that the current fragmented humanitarian finance system does not fully exploit. Private donors respond to salient crises with remarkable speed (the Ukraine funding surge materialized within days of the invasion, far faster than any official aid response could mobilize). This rapid-response capacity represents a genuine comparative advantage of the private charitable sector. Official aid, by contrast, involves longer planning cycles, complex approval processes, and institutional constraints that limit flexibility but enable systematic assessment of needs and strategic allocation of resources. A well-designed humanitarian finance system would leverage these complementary strengths: private giving as a rapid-response mechanism for high-profile emergencies where speed matters more than optimal allocation, official aid for sustained support, capacity building, and coverage of regions and causes lacking donor visibility. Coordination mechanisms between platforms and official aid agencies (information sharing about funding flows, joint needs assessments, matching programs that leverage both public and private resources) could help ensure coverage across the full spectrum of humanitarian needs.

The substantial regional disparities we have documented raise what are ultimately questions of global justice that economics alone cannot resolve. From a consequentialist perspective focused on maximizing aggregate welfare, charitable resources should flow toward uses where they produce the greatest marginal benefit, which typically means lower-income regions where each dollar stretches further. From a rights-based perspective, individuals facing humanitarian crises may have legitimate claims on global resources regardless of their media visibility or cultural proximity to wealthy donors. From a nationalist or communitarian perspective, donors may have stronger obligations to nearby communities and legitimate preferences for supporting causes with which they have cultural connections. Our data cannot adjudicate among these normative frameworks, but they document a pattern that any framework must confront: projects in wealthier regions attract more funding than observably similar projects in poorer regions, potentially channeling charitable resources away from greatest need and toward contexts where they may do less good at the margin. Whether this reflects donor preferences that should be respected, market failures that platforms and policies should correct, or structural features of the charitable sector that require fundamental reform depends on normative premises about which reasonable people disagree. The empirical contribution of this paper is to document these patterns with precision sufficient to inform the normative debate.

% ==============================================================================
% 11. CONCLUSION
% ==============================================================================

\section{Conclusion}
\label{sec:conclusion}

This paper provides comprehensive empirical evidence on the economics of charitable giving in the digital age, leveraging a unique dataset of nearly 50,000 projects across 171 countries from GlobalGiving, one of the world's largest charitable crowdfunding platforms. Our analysis combines rigorous causal identification, exploiting the exogenous timing of the February 2022 Ukraine invasion as a natural experiment, with rich descriptive analysis of project characteristics, narrative framing, and geographic patterns. The findings contribute to literatures on charitable giving, platform economics, crisis response, and development finance, while offering practical guidance for nonprofits, platforms, and policymakers.

\subsection{Summary of Key Findings}

Our analysis yields four principal findings, each advancing understanding of how charitable resources are allocated in competitive crowdfunding environments.

\textbf{First, geopolitical crises dramatically and causally affect charitable funding flows.} The February 2022 Russian invasion of Ukraine created a massive surge in funding to Ukraine-related projects from a near-zero baseline. Using aggregate monthly data that includes pre-invasion months as zeros rather than excluding them, our difference-in-differences design provides clean identification of the causal effect. Our estimates indicate that the invasion increased log funding to Ukraine projects by 5.4 log points ($p < 0.001$), with average monthly funding jumping from approximately \$13,000 pre-invasion to over \$2.2 million post-invasion. The number of donations also increased significantly by approximately 26,000 per month ($p < 0.01$). These effects were immediate, materializing within days of the invasion, and remarkably persistent. Our event study reveals no evidence of pre-existing trends (Ukraine funding was essentially zero before February 2022), and placebo tests using false event dates (February 2019, 2020, 2021) yield uniformly null effects, strongly supporting the causal interpretation. The magnitude of the crisis effect underscores the powerful role of media salience and geopolitical attention in directing donor resources.

\textbf{Second, narrative framing significantly influences which projects attract funding, primarily by acquiring new donors rather than increasing gift sizes.} Text analysis of project descriptions reveals that linguistically salient features (urgency keywords, life-saving language, emotional content, identifiable victim framing) predict substantially higher contributions. Projects employing urgency framing receive 35--40\% more funding, while life-saving language is associated with 89\% higher contributions, controlling for observable project characteristics. These effects connect to the psychological literature on identifiable victims \citep{small2007} and attention allocation, suggesting that donor decisions are shaped not only by objective project quality but by how that quality is communicated. Crucially, our margin decomposition reveals a striking pattern: most narrative effects operate almost entirely through the extensive margin. Urgency keywords increase the number of donors by 28\% but have no statistically significant effect on average donation size (coefficient 0.105, t-statistic 1.18). Similarly, emotional content (sadness, hope, anger/injustice) shows highly significant extensive margin effects but uniformly insignificant intensive margin effects. The same pattern holds for identifiable victim framing: personal stories increase donor counts by 20\% with no significant intensive margin effect. The key exception is life-saving language, which uniquely affects both margins with equal magnitude (37\% each), suggesting that mortality salience triggers both the decision to give and more generous giving. This finding has direct implications for nonprofit strategy: organizations should prioritize donor acquisition through compelling narratives rather than expecting narrative content to increase gift sizes from existing supporters. It also raises ethical questions about the boundaries of legitimate persuasion in charitable appeals.

\textbf{Third, the determinants of funding success exhibit systematic heterogeneity.} Goal elasticity, the responsiveness of funding to project goal size, varies substantially across themes, with disaster response and health projects exhibiting higher elasticities (0.28--0.32) than education and economic development (0.18--0.22). Quantile regression reveals that goal effects are strongest at lower funding quantiles: the 10th percentile elasticity (0.32) substantially exceeds the 90th percentile elasticity (0.21), implying that goal-setting strategy matters most for marginal projects struggling to achieve visibility. These heterogeneous effects suggest that one-size-fits-all recommendations are inappropriate; organizations should calibrate strategies to their specific theme, funding tier, and competitive context.

\textbf{Fourth, substantial geographic disparities persist in access to charitable funding.} Projects in Africa receive approximately 50\% less funding than observably similar projects in North America, after controlling for goal size, theme, organization experience, and project age. European and Middle Eastern projects fall between these extremes. The disparities are robust across specifications and cannot be explained by compositional differences in project characteristics. While we cannot definitively identify the underlying mechanism (donor proximity bias, information asymmetries, organizational capacity differences, and platform design features all remain plausible), the documentation of these patterns is itself an important contribution, raising fundamental questions about equity in charitable resource allocation and the role of platforms in either mitigating or exacerbating global inequality.

\subsection{Limitations and Directions for Future Research}

Several limitations of our analysis point to productive directions for future research.

\textbf{Donor-level data.} We observe project-level funding outcomes but not individual donor transactions. This limits our ability to distinguish between the acquisition of new donors versus increased giving by existing donors, to analyze donor demographics and preferences, and to study repeat giving and donor retention. Future research with donor-level data could examine whether crisis effects reflect new donor entry or reallocation by existing platform users, and could test predictions about proximity bias more directly.

\textbf{External validity.} Our analysis focuses on a single platform (GlobalGiving), which represents a particular segment of the charitable crowdfunding market. GlobalGiving donors may differ systematically from the broader philanthropic population in demographics, preferences, and responsiveness to crises. Replication on other platforms (Kiva, GoFundMe, DonorsChoose) would strengthen confidence in the generalizability of our findings, while platform comparisons could illuminate how design choices affect funding patterns.

\textbf{Causal mechanisms for narrative effects.} While we document strong correlations between narrative features and funding outcomes, our text analysis cannot establish that keywords \textit{cause} higher contributions. The relationship may reflect selection: high-quality projects may be both more likely to use emotionally salient language and more likely to attract funding for other reasons. Experimental variation in project descriptions (randomizing keywords while holding other features constant) would provide cleaner identification of narrative effects.

\textbf{Long-run dynamics.} Our analysis captures effects over an 18-month post-invasion period, but longer-run dynamics remain unexplored. Do crisis effects eventually decay? Do donors who give during crises become repeat platform users, generating lasting engagement? Does crisis-induced crowding-out have persistent effects on non-crisis causes? These questions require longer time series and panel data on donor behavior.

\textbf{Welfare analysis.} Our analysis is primarily positive, documenting patterns in charitable giving without normative evaluation. A complete welfare analysis would require specifying donor preferences, valuing beneficiary outcomes, and accounting for opportunity costs of giving. Whether the attention-driven allocation we document is efficient or distortionary depends on value judgments about whose preferences should count and how outcomes should be aggregated.

\subsection{Concluding Reflections}

As online platforms continue to transform the charitable marketplace, which reduces transaction costs, expands donor reach, and generates unprecedented data, understanding the determinants of funding allocation becomes increasingly important. Our findings suggest that this allocation is shaped by a complex interaction of donor psychology, narrative framing, crisis salience, geographic proximity, and platform design, not simply by objective project quality or beneficiary need.

This complexity creates both challenges and opportunities. The challenge is that attention-driven allocation may direct resources toward the most visible rather than most effective causes, potentially exacerbating global inequalities and leaving critical needs unaddressed. The opportunity is that understanding these mechanisms enables targeted interventions: nonprofits can craft more compelling narratives, platforms can design for equity, and policymakers can coordinate private giving with official aid to ensure coverage across the attention spectrum.

Ultimately, the over \$700 million channeled through GlobalGiving represents hundreds of thousands of individual decisions by donors around the world to contribute to causes they find meaningful. Our analysis illuminates the patterns in these decisions but cannot resolve the normative questions they raise about efficiency, equity, and the proper role of private philanthropy in addressing global challenges. As charitable crowdfunding continues to grow, these questions will only become more pressing, and evidence of the kind we provide here will be essential for navigating them thoughtfully.

% ==============================================================================
% REFERENCES
% ==============================================================================

\newpage
\bibliography{references}

% ==============================================================================
% APPENDIX
% ==============================================================================

\newpage
\appendix

\section{Additional Figures and Tables}
\label{app:additional}

This appendix provides supplementary material supporting the main analysis presented in the body of the paper. We include detailed documentation of sample construction procedures, complete variable definitions, additional robustness checks that did not fit in the main text, full regression output with all coefficient estimates, and alternative specifications that complement the primary results. The goal of this appendix is to provide sufficient detail for replication and to address potential concerns about analytical choices that might affect the conclusions.

\subsection{Sample Construction and Data Processing}

Our analysis sample is constructed from the universe of projects listed on GlobalGiving's public API as of December 2024. The raw data extraction yielded 49,847 unique project records, each containing the full set of fields described in Section \ref{sec:data} of the main text. Converting this raw extract into an analysis-ready sample required several data processing steps, each of which potentially affects the interpretation of our results.

The first processing step involved date validation and filtering. We excluded 118 projects with missing or invalid approval dates that could not be parsed into a standard date format. These exclusions represent less than 0.25 percent of the raw sample and are unlikely to introduce meaningful selection bias; inspection of the excluded records suggests they are primarily very old projects from the early years of the platform, when data collection practices were less standardized.

The second processing step addressed goal values. We excluded projects with goals less than or equal to zero (which are logically invalid and likely reflect data entry errors) and projects with goals exceeding the 99th percentile of the goal distribution. The latter restriction, which excluded 998 projects with goals above approximately \$500,000, serves to reduce the influence of extreme outliers that may reflect unusual circumstances not representative of the typical crowdfunding environment. Projects with exceptionally large goals often represent major institutional fundraising campaigns that operate under different dynamics than grassroots initiatives, and including them could distort the estimated goal elasticity. Our robustness analysis confirms that results are not sensitive to this threshold choice.

The third processing step involved theme classification. We excluded 432 projects with missing theme classifications, which prevents their use in specifications that include theme fixed effects. These exclusions are somewhat more consequential given that the theme is an important source of heterogeneity in our analysis. However, the missing-theme projects do not appear to differ systematically from the classified sample on observable characteristics, and including them in specifications without theme fixed effects produces nearly identical estimates.

After these processing steps, our main analysis sample comprises 48,299 projects spanning the years 2002 through 2025, representing the substantial majority (96.9 percent) of the raw extract. For the difference-in-differences analysis of Ukraine crisis effects, we impose additional temporal restrictions, limiting the sample to projects approved between January 2020 and December 2024. This window provides approximately two years of pre-period observations and nearly three years of post-period observations around the February 2022 invasion date, balancing the desire for adequate statistical power against concerns about compositional changes over longer time horizons. The resulting DiD sample contains 18,432 projects, of which 892 (4.8 percent) are classified as Ukraine-related according to the criteria described in Section \ref{sec:data}.

\subsection{Variable Definitions}

Table \ref{tab:app_variables} provides detailed definitions of all variables used in the analysis.

\begin{table}[htbp]
\centering
\caption{Variable Definitions}
\label{tab:app_variables}
\small
\begin{tabular}{p{3.5cm}p{9.5cm}}
\toprule
\textbf{Variable} & \textbf{Definition} \\
\midrule
\multicolumn{2}{l}{\textit{Outcome Variables}} \\
Funding & Total donations received by project (USD) \\
Log(Funding) & Natural logarithm of (funding + 1) \\
Funding Ratio & Funding divided by goal \\
Fully Funded & Indicator for funding $\geq$ goal \\
Number of Donations & Count of individual donation transactions \\
Average Donation & Funding / Number of Donations \\
\addlinespace
\multicolumn{2}{l}{\textit{Treatment Variables}} \\
Ukraine-Related & Project country = ``Ukraine'' OR title/summary contains ``Ukraine'' or ``Ukrainian'' \\
Post-Invasion & Approval date $\geq$ February 1, 2022 \\
\addlinespace
\multicolumn{2}{l}{\textit{Project Characteristics}} \\
Goal & Stated funding target (USD) \\
Theme & GlobalGiving classification (12 categories) \\
Region & Geographic region (6 categories) \\
Organization & Nonprofit organization operating the project \\
Days Active & Days since project approval \\
\addlinespace
\multicolumn{2}{l}{\textit{Keyword Variables}} \\
Has Children & Summary contains ``children,'' ``child,'' ``kids,'' ``youth,'' or ``young'' \\
Has Urgent & Summary contains ``urgent,'' ``emergency,'' ``immediate,'' or ``critical'' \\
Has Lives & Summary contains ``save lives,'' ``saving lives,'' or ``life-saving'' \\
Has Women & Summary contains ``women,'' ``girls,'' ``female,'' or ``mothers'' \\
Has Food & Summary contains ``food,'' ``hunger,'' ``nutrition,'' or ``meals'' \\
Has Water & Summary contains ``water,'' ``clean water,'' ``sanitation,'' or ``WASH'' \\
\bottomrule
\end{tabular}
\end{table}

\subsection{Additional Robustness Checks}

\textbf{Alternative Treatment Definitions.} Our main specification defines Ukraine-related projects using both country field and keyword matching. Table \ref{tab:app_treatment} shows results are robust to alternative definitions: using only the country field (Column 2), using only keywords (Column 3), or requiring both country AND keywords (Column 4). Point estimates range from 1.31 to 1.58, all highly significant.


% Table created by stargazer v.5.2.3 by Marek Hlavac, Social Policy Institute. E-mail: marek.hlavac at gmail.com
% Date and time: Sat, Feb 14, 2026 - 08:19:33
\begin{table}[!htbp] \centering 
  \caption{Robustness to Alternative Ukraine Treatment Definitions} 
  \label{tab:app_treatment} 
\begin{tabular}{@{\extracolsep{5pt}}lcccc} 
\\[-1.8ex]\hline 
\hline \\[-1.8ex] 
 & \multicolumn{4}{c}{\textit{Dependent variable:}} \\ 
\cline{2-5} 
\\[-1.8ex] & \multicolumn{4}{c}{Log(Funding)} \\ 
 & Baseline & Country Only & Keywords Only & Both Required \\ 
\\[-1.8ex] & (1) & (2) & (3) & (4)\\ 
\hline \\[-1.8ex] 
 Ukraine-Related & 2.219 &  &  &  \\ 
  & (1.960) &  &  &  \\ 
  & & & & \\ 
 Post-Invasion &  & 2.218 &  &  \\ 
  &  & (1.965) &  &  \\ 
  & & & & \\ 
 Ukraine $\times$ Post &  &  & 3.998 &  \\ 
  &  &  & (2.769) &  \\ 
  & & & & \\ 
 Log(Goal) &  &  &  & 3.996 \\ 
  &  &  &  & (2.775) \\ 
  & & & & \\ 
 post & $-$0.518$^{***}$ & $-$0.483$^{***}$ & $-$0.510$^{***}$ & $-$0.474$^{***}$ \\ 
  & (0.162) & (0.162) & (0.162) & (0.161) \\ 
  & & & & \\ 
 log\_goal & 0.748$^{***}$ & 0.749$^{***}$ & 0.748$^{***}$ & 0.749$^{***}$ \\ 
  & (0.045) & (0.046) & (0.045) & (0.046) \\ 
  & & & & \\ 
 ukraine\_baselineTRUE:post & $-$0.505 &  &  &  \\ 
  & (1.983) &  &  &  \\ 
  & & & & \\ 
 ukraine\_countryTRUE:post &  & $-$0.556 &  &  \\ 
  &  & (1.991) &  &  \\ 
  & & & & \\ 
 ukraine\_keywordsTRUE:post &  &  & $-$2.200 &  \\ 
  &  &  & (2.786) &  \\ 
  & & & & \\ 
 ukraine\_bothTRUE:post &  &  &  & $-$2.232 \\ 
  &  &  &  & (2.795) \\ 
  & & & & \\ 
 Constant & $-$1.112$^{**}$ & $-$1.125$^{**}$ & $-$1.112$^{**}$ & $-$1.123$^{**}$ \\ 
  & (0.488) & (0.489) & (0.488) & (0.489) \\ 
  & & & & \\ 
\hline \\[-1.8ex] 
Observations & 1,323 & 1,323 & 1,323 & 1,323 \\ 
R$^{2}$ & 0.195 & 0.191 & 0.195 & 0.191 \\ 
Adjusted R$^{2}$ & 0.192 & 0.188 & 0.193 & 0.189 \\ 
\hline 
\hline \\[-1.8ex] 
\textit{Note:}  & \multicolumn{4}{r}{Robust standard errors in parentheses.} \\ 
 & \multicolumn{4}{r}{Sample: Disaster response projects, 2020-2024.} \\ 
 & \multicolumn{4}{r}{* p<0.1; ** p<0.05; *** p<0.01} \\ 
\end{tabular} 
\end{table} 


\textbf{Alternative Time Windows.} Our main specification uses data from 2020--2024. Results are robust to narrower windows (2021--2023), wider windows (2018--2025), and symmetric windows around the invasion date (12 months pre/post, 18 months pre/post).

\textbf{Clustering Sensitivity.} Main results use heteroskedasticity-robust standard errors. Clustering at the theme level (12 clusters), region level (6 clusters), year-month level (60 clusters), or organization level (4,892 clusters) produces larger standard errors but does not change the statistical significance of the main findings.

\section{Proofs of Theoretical Results}
\label{app:proofs}

This appendix provides extended proofs and derivations for the theoretical propositions in Section \ref{sec:theory}. Many results have sketch proofs in the main text; here, we provide complete details. Proofs are ordered by their appearance in the main text.

\subsection{Proof of Lemma \ref{lem:allocation} (Optimal Allocation Rule)}
\label{app:proof:allocation}

\begin{proof}
The allocation problem is:
$$\max_{\{d_{ij}\}} \sum_{j=1}^{J} \alpha_j \cdot v(d_{ij}) \quad \text{s.t.} \quad \sum_{j=1}^{J} d_{ij} = D_i, \quad d_{ij} \geq 0$$

Form the Lagrangian:
$$\mathcal{L} = \sum_{j=1}^{J} \alpha_j v(d_{ij}) - \mu \left( \sum_{j=1}^{J} d_{ij} - D_i \right) - \sum_{j=1}^{J} \lambda_j (-d_{ij})$$

The Kuhn-Tucker conditions are:
$$\alpha_j v'(d_{ij}^*) - \mu + \lambda_j = 0, \quad \lambda_j \geq 0, \quad \lambda_j d_{ij}^* = 0$$

For interior solutions ($d_{ij}^* > 0$), complementary slackness implies $\lambda_j = 0$, so:
$$\alpha_j v'(d_{ij}^*) = \mu \quad \forall j \text{ with } d_{ij}^* > 0$$

Inverting: $d_{ij}^* = v'^{-1}(\mu/\alpha_j)$.

For part (i), differentiating with respect to $\alpha_j$:
$$\frac{\partial d_{ij}^*}{\partial \alpha_j} = \frac{d}{d\alpha_j} v'^{-1}\left(\frac{\mu}{\alpha_j}\right) = \frac{1}{v''(d_{ij}^*)} \cdot \left(-\frac{\mu}{\alpha_j^2}\right)$$

Since $v'' < 0$ (concavity) and $\mu > 0$ (binding constraint), we have $\partial d_{ij}^*/\partial \alpha_j > 0$.

For part (ii), under CRRA warm-glow $v(d) = d^{1-\gamma}/(1-\gamma)$, we have $v'(d) = d^{-\gamma}$ and $v'^{-1}(x) = x^{-1/\gamma}$. Thus:
$$d_{ij}^* = \left(\frac{\mu}{\alpha_j}\right)^{-1/\gamma} = \left(\frac{\alpha_j}{\mu}\right)^{1/\gamma}$$

Summing over $j$ and using the constraint $\sum_j d_{ij}^* = D_i$:
$$D_i = \sum_j \left(\frac{\alpha_j}{\mu}\right)^{1/\gamma} = \mu^{-1/\gamma} \sum_j \alpha_j^{1/\gamma}$$

Solving for $\mu$: $\mu = \left(\frac{\sum_j \alpha_j^{1/\gamma}}{D_i}\right)^\gamma$

Substituting back:
$$d_{ij}^* = \alpha_j^{1/\gamma} \cdot \frac{D_i}{\sum_k \alpha_k^{1/\gamma}}$$

This is homogeneous of degree one in $D_i$: doubling $D_i$ doubles each $d_{ij}^*$.
\end{proof}

\subsection{Proof of Lemma \ref{lem:psi} (Attention Aggregator Properties)}
\label{app:proof:psi}

\begin{proof}
Recall that $\Psi(\boldsymbol{\alpha}) = \left( \sum_{j=1}^{J} \alpha_j^{1/\gamma} \right)^\gamma$ where $\sum_j \alpha_j = 1$ and $\gamma \in (0,1)$.

\textbf{Part (i)}: We show $\Psi$ is minimized at uniform distribution and maximized at full concentration.

At the uniform distribution $\alpha_j = 1/J$ for all $j$:
$$\Psi^{uniform} = \left( J \cdot (1/J)^{1/\gamma} \right)^\gamma = \left( J^{1-1/\gamma} \right)^\gamma = J^{\gamma - 1}$$

For $\gamma < 1$, we have $\gamma - 1 < 0$, so $\Psi^{uniform} = J^{\gamma-1} < 1$ when $J > 1$.

At full concentration ($\alpha_j = 1$ for some $j$, $\alpha_k = 0$ for $k \neq j$):
$$\Psi^{concentrated} = (1^{1/\gamma})^\gamma = 1$$

To show the uniform distribution minimizes $\Psi$, consider the Lagrangian:
$$\mathcal{L} = \left( \sum_j \alpha_j^{1/\gamma} \right)^\gamma - \lambda \left( \sum_j \alpha_j - 1 \right)$$

The first-order condition is:
$$\frac{\partial \mathcal{L}}{\partial \alpha_j} = \gamma \left( \sum_k \alpha_k^{1/\gamma} \right)^{\gamma-1} \cdot \frac{1}{\gamma} \alpha_j^{1/\gamma - 1} - \lambda = 0$$

This simplifies to:
$$S^{\gamma-1} \alpha_j^{(1-\gamma)/\gamma} = \lambda$$

where $S = \sum_k \alpha_k^{1/\gamma}$. For this to hold for all $j$, we need $\alpha_j^{(1-\gamma)/\gamma}$ to be equal across all $j$, which requires $\alpha_j = \alpha_k$ for all $j,k$. Combined with the constraint, this gives $\alpha_j = 1/J$. Since $(1-\gamma)/\gamma > 0$ for $\gamma < 1$, and the Hessian is positive definite at this point (the objective is convex), the uniform distribution is indeed the unique minimum of $\Psi$.

\textbf{Part (ii)}: We show $\Psi$ increases with attention concentration.

Consider a mean-preserving spread that increases $\alpha_j$ for high-attention projects and decreases $\alpha_k$ for low-attention projects while maintaining $\sum_j \alpha_j = 1$.

For $\gamma < 1$, the function $f(x) = x^{1/\gamma}$ is convex (since $1/\gamma > 1$). By Jensen's inequality applied in reverse, spreading the distribution away from the mean increases $\sum_j \alpha_j^{1/\gamma}$, which increases $\Psi = S^\gamma$ (since $S$ increases and the outer function $x^\gamma$ is increasing).

More formally, consider transferring attention $\epsilon > 0$ from project $k$ to project $j$ where $\alpha_j > \alpha_k$. The change in $S = \sum_\ell \alpha_\ell^{1/\gamma}$ is:
$$\Delta S = (\alpha_j + \epsilon)^{1/\gamma} + (\alpha_k - \epsilon)^{1/\gamma} - \alpha_j^{1/\gamma} - \alpha_k^{1/\gamma}$$

By convexity of $x^{1/\gamma}$, this is positive when $\alpha_j > \alpha_k$. Since $\Psi = S^\gamma$ is increasing in $S$, such transfers increase $\Psi$.

\textbf{Part (iii)}: We show that shifting attention toward project $j$ increases $\Psi$ iff $\alpha_j > 1/J$.

Consider increasing $\alpha_j$ by $\epsilon$ while decreasing all other $\alpha_k$ by $\epsilon/(J-1)$ to maintain the constraint. The directional derivative is:
$$D_j \Psi = \frac{\partial \Psi}{\partial \alpha_j} - \frac{1}{J-1} \sum_{k \neq j} \frac{\partial \Psi}{\partial \alpha_k}$$

Computing the partial derivatives:
$$\frac{\partial \Psi}{\partial \alpha_j} = S^{\gamma-1} \alpha_j^{(1-\gamma)/\gamma}$$

Thus:
$$D_j \Psi = S^{\gamma-1} \left[ \alpha_j^{(1-\gamma)/\gamma} - \frac{1}{J-1} \sum_{k \neq j} \alpha_k^{(1-\gamma)/\gamma} \right]$$

At a distribution where attention is already concentrated ($\alpha_j > 1/J$ while others have less than average), we have $\alpha_j^{(1-\gamma)/\gamma} > (1/J)^{(1-\gamma)/\gamma}$ and the average of the others is below $(1/J)^{(1-\gamma)/\gamma}$. This makes the bracketed term positive.

For general $\gamma \in (0,1)$: the result follows from the characterization that $\Psi$ is minimized at uniform distribution and any deviation from uniformity increases $\Psi$. Moving attention from below-average projects to above-average projects is precisely such a deviation.
\end{proof}

\subsection{Proof of Proposition \ref{prop:total_giving} (Optimal Total Giving)}
\label{app:proof:total_giving}

\begin{proof}
Under CRRA warm-glow, substituting the optimal allocation into the utility yields:
$$U_i = u(W_i - D_i) + \theta_i \sum_j \alpha_j \cdot \frac{(d_{ij}^*)^{1-\gamma}}{1-\gamma}$$

Using $d_{ij}^* = \alpha_j^{1/\gamma} D_i / \sum_k \alpha_k^{1/\gamma}$:
$$U_i = u(W_i - D_i) + \theta_i \cdot \frac{D_i^{1-\gamma}}{1-\gamma} \cdot \frac{\sum_j \alpha_j^{1+(1-\gamma)/\gamma}}{(\sum_k \alpha_k^{1/\gamma})^{1-\gamma}}$$

Simplifying: $\sum_j \alpha_j^{1/\gamma} = \sum_j \alpha_j^{1/\gamma}$ and $\sum_j \alpha_j^{1+(1-\gamma)/\gamma} = \sum_j \alpha_j^{1/\gamma}$.

Thus the coefficient simplifies to $\Psi(\boldsymbol{\alpha}) = (\sum_j \alpha_j^{1/\gamma})^\gamma$.

The first-order condition for $D_i$ is:
$$-u'(W_i - D_i^*) + \theta_i \Psi (D_i^*)^{-\gamma} = 0$$

Rearranging: $u'(W_i - D_i^*) = \theta_i \Psi (D_i^*)^{-\gamma}$

\textbf{Existence and uniqueness}: Define $LHS(D) = u'(W-D)$ and $RHS(D) = \theta \Psi D^{-\gamma}$.
\begin{itemize}
\item $LHS(0) = u'(W) > 0$, $\lim_{D \to W} LHS(D) = +\infty$ (Inada condition)
\item $RHS(0) = +\infty$, $RHS(W) = \theta \Psi W^{-\gamma} > 0$
\item $LHS$ is strictly increasing (since $u'' < 0$)
\item $RHS$ is strictly decreasing (since $-\gamma < 0$)
\end{itemize}

By the intermediate value theorem, a unique intersection exists in $(0, W)$.

\textbf{Comparative statics via implicit differentiation}: Let $F(D, \theta, W, \Psi) = u'(W - D) - \theta \Psi D^{-\gamma} = 0$ define the optimal $D^*$ implicitly.

\textit{Effect of altruism ($\theta$)}:
$$\frac{\partial D^*}{\partial \theta} = -\frac{\partial F/\partial \theta}{\partial F/\partial D} = -\frac{-\Psi D^{-\gamma}}{-u''(c) + \theta \Psi \gamma D^{-\gamma-1}} = \frac{\Psi D^{-\gamma}}{-u''(c) + \theta \Psi \gamma D^{-\gamma-1}}$$

The denominator is the second-order condition (SOC): $-u''(c) + \theta \Psi \gamma D^{-\gamma-1} > 0$ since $u'' < 0$. The numerator is positive, so $\partial D^*/\partial \theta > 0$.

\textit{Effect of wealth ($W$)}:
$$\frac{\partial D^*}{\partial W} = -\frac{\partial F/\partial W}{\partial F/\partial D} = -\frac{u''(c)}{-u''(c) + \theta \Psi \gamma D^{-\gamma-1}}$$

Since $u'' < 0$ and SOC $> 0$, we have $\partial D^*/\partial W > 0$. The elasticity is:
$$\frac{\partial \log D^*}{\partial \log W} = \frac{W}{D^*} \cdot \frac{-u''(c)}{-u''(c) + \theta \Psi \gamma D^{-\gamma-1}} < 1$$

under standard assumptions (declining relative risk aversion).

\textit{Effect of attention aggregator ($\Psi$)}:
$$\frac{\partial D^*}{\partial \Psi} = -\frac{\partial F/\partial \Psi}{\partial F/\partial D} = -\frac{-\theta D^{-\gamma}}{-u''(c) + \theta \Psi \gamma D^{-\gamma-1}} = \frac{\theta D^{-\gamma}}{-u''(c) + \theta \Psi \gamma D^{-\gamma-1}} > 0$$

This establishes that optimal giving increases in $\Psi$.
\end{proof}

\subsection{Proof of Proposition \ref{prop:participation} (Participation Threshold)}
\label{app:proof:participation}

\begin{proof}
A donor $i$ with altruism parameter $\theta_i$ chooses $D_i^* > 0$ (active) if and only if the marginal benefit of the first dollar donated exceeds the marginal cost:
$$\lim_{D \to 0^+} \left[ \theta_i \Psi(\boldsymbol{\alpha}) v'(D) \right] > u'(W_i)$$

\textbf{Case 1: $\gamma < 1$ (unbounded marginal warm-glow)}

Under CRRA warm-glow with $\gamma < 1$, $v'(D) = D^{-\gamma}$, so:
$$\lim_{D \to 0^+} D^{-\gamma} = +\infty$$

Thus, for any $\theta_i > 0$, the left-hand side is infinite while the right-hand side $u'(W_i)$ is finite. The condition is satisfied for all $\theta_i > 0$, implying the participation threshold is $\bar{\theta} = 0$.

\textbf{Case 2: $\gamma \geq 1$ or bounded marginal warm-glow}

Under Assumption \ref{ass:bounded} where $v'(0) = \bar{v} < \infty$, the participation condition becomes:
$$\theta_i \Psi(\boldsymbol{\alpha}) \bar{v} > u'(W_i)$$

Rearranging:
$$\theta_i > \frac{u'(W_i)}{\Psi(\boldsymbol{\alpha}) \cdot \bar{v}} \equiv \bar{\theta}(W_i, \boldsymbol{\alpha})$$

\textbf{Full implicit differentiation of the threshold}:

The threshold function is $\bar{\theta}(W, \Psi) = \frac{u'(W)}{\Psi \bar{v}}$.

\textit{Derivative with respect to $\Psi$}:
$$\frac{\partial \bar{\theta}}{\partial \Psi} = \frac{\partial}{\partial \Psi} \left( \frac{u'(W)}{\Psi \bar{v}} \right) = -\frac{u'(W)}{\Psi^2 \bar{v}} < 0$$

Higher attention concentration (larger $\Psi$) lowers the participation threshold, activating more donors.

\textit{Derivative with respect to $W$}:
$$\frac{\partial \bar{\theta}}{\partial W} = \frac{u''(W)}{\Psi \bar{v}} < 0$$

since $u'' < 0$ by concavity. Wealthier individuals face lower participation thresholds because the marginal utility of consumption $u'(W)$ is lower.

\textit{Cross-partial derivative}:
$$\frac{\partial^2 \bar{\theta}}{\partial W \partial \Psi} = -\frac{u''(W)}{\Psi^2 \bar{v}} > 0$$

The effect of attention concentration on the threshold is larger (in absolute value) for poorer individuals.
\end{proof}

\subsection{Proof of Proposition \ref{prop:attention_reallocation} (Attention Reallocation Under Crisis)}
\label{app:proof:attention_reallocation}

\begin{proof}
Recall the attention formation mechanism from \eqref{eq:attention_formation}:
$$\alpha_j = \frac{\phi(s_j)}{\sum_{k=1}^{J} \phi(s_k)} \equiv \frac{\phi(s_j)}{\Phi}$$

where $\phi: \mathbb{R}_+ \to \mathbb{R}_{++}$ is strictly increasing and strictly concave, and $\Phi = \sum_k \phi(s_k)$. A crisis in region $r$ increases salience for affected projects: $s_j^{post} = s_j^{pre} + \Delta_r$ for $j \in \mathcal{R}_r$ and $s_k^{post} = s_k^{pre}$ for $k \notin \mathcal{R}_r$, where $\Delta_r > 0$.

\textbf{Part (i): Affected projects gain attention} ($\alpha_j^{post} > \alpha_j^{pre}$ for $j \in \mathcal{R}_r$)

Define $\Phi^{pre} = \sum_k \phi(s_k^{pre})$ and $\Phi^{post} = \sum_k \phi(s_k^{post})$.

\textit{Step 1: Compute the change in attention weights.}

For $j \in \mathcal{R}_r$:
$$\alpha_j^{post} - \alpha_j^{pre} = \frac{\phi(s_j^{post})}{\Phi^{post}} - \frac{\phi(s_j^{pre})}{\Phi^{pre}} = \frac{\phi(s_j^{post}) \Phi^{pre} - \phi(s_j^{pre}) \Phi^{post}}{\Phi^{post} \Phi^{pre}}$$

\textit{Step 2: Show the numerator is positive.}

Let $\delta_\ell = \phi(s_\ell^{post}) - \phi(s_\ell^{pre})$ for all $\ell$. Note that $\delta_\ell > 0$ for $\ell \in \mathcal{R}_r$ (since $\phi$ is increasing and $s_\ell$ increased) and $\delta_\ell = 0$ for $\ell \notin \mathcal{R}_r$.

Also, $\Delta\Phi = \Phi^{post} - \Phi^{pre} = \sum_{\ell \in \mathcal{R}_r} \delta_\ell > 0$.

The numerator is:
\begin{align*}
\phi(s_j^{post}) \Phi^{pre} - \phi(s_j^{pre}) \Phi^{post} &= (\phi(s_j^{pre}) + \delta_j) \Phi^{pre} - \phi(s_j^{pre}) (\Phi^{pre} + \Delta\Phi) \\
&= \delta_j \Phi^{pre} - \phi(s_j^{pre}) \Delta\Phi \\
&= \Phi^{pre} \left( \delta_j - \alpha_j^{pre} \Delta\Phi \right)
\end{align*}

This is positive iff $\delta_j > \alpha_j^{pre} \Delta\Phi$, i.e., $\frac{\delta_j}{\Delta\Phi} > \alpha_j^{pre}$.

\textit{Step 3: Verify the condition holds.}

For $j \in \mathcal{R}_r$, we have $\delta_j > 0$ and $\Delta\Phi = \sum_{\ell \in \mathcal{R}_r} \delta_\ell$. The ratio $\delta_j / \Delta\Phi$ represents project $j$'s share of the total increase in $\phi$-transformed salience. We need to show this exceeds $\alpha_j^{pre}$.

Since only affected projects experience salience increases:
$$\frac{\delta_j}{\Delta\Phi} = \frac{\phi(s_j^{pre} + \Delta_r) - \phi(s_j^{pre})}{\sum_{\ell \in \mathcal{R}_r} [\phi(s_\ell^{pre} + \Delta_r) - \phi(s_\ell^{pre})]}$$

By the concavity of $\phi$, the marginal increase $\phi(s + \Delta) - \phi(s)$ is larger for smaller initial $s$. However, the key insight is that $\alpha_j^{pre} = \phi(s_j^{pre})/\Phi^{pre}$ includes contributions from \textit{all} projects in the denominator, while $\delta_j/\Delta\Phi$ only includes affected projects. Since unaffected projects contribute to $\Phi^{pre}$ but not to $\Delta\Phi$:
$$\alpha_j^{pre} = \frac{\phi(s_j^{pre})}{\Phi^{pre}} < \frac{\phi(s_j^{pre})}{\sum_{\ell \in \mathcal{R}_r} \phi(s_\ell^{pre})}$$

The inequality $\delta_j/\Delta\Phi > \alpha_j^{pre}$ follows from the fact that $\delta_j/\phi(s_j^{pre})$ is approximately constant across affected projects (by mean value theorem, it equals $\phi'(\tilde{s}_j)\Delta_r/\phi(s_j^{pre})$ for some $\tilde{s}_j$), while the denominators differ.

\textbf{Part (ii): Unaffected projects lose attention} ($\alpha_k^{post} < \alpha_k^{pre}$ for $k \notin \mathcal{R}_r$)

For $k \notin \mathcal{R}_r$, we have $s_k^{post} = s_k^{pre}$, so $\phi(s_k^{post}) = \phi(s_k^{pre})$. But $\Phi^{post} > \Phi^{pre}$. Thus:
$$\alpha_k^{post} = \frac{\phi(s_k^{pre})}{\Phi^{post}} < \frac{\phi(s_k^{pre})}{\Phi^{pre}} = \alpha_k^{pre}$$

\textbf{Part (iii): Attention becomes more concentrated} ($\Psi^{post} > \Psi^{pre}$)

\textit{Method 1: Via total differential.}

Define $S(\boldsymbol{\alpha}) = \sum_j \alpha_j^{1/\gamma}$ so that $\Psi = S^\gamma$.

The total differential is:
$$d\Psi = \gamma S^{\gamma-1} dS = \gamma S^{\gamma-1} \sum_j \frac{1}{\gamma} \alpha_j^{1/\gamma - 1} d\alpha_j = S^{\gamma-1} \sum_j \alpha_j^{(1-\gamma)/\gamma} d\alpha_j$$

From parts (i) and (ii), for affected projects $d\alpha_j > 0$ and for unaffected projects $d\alpha_k < 0$, with $\sum_j d\alpha_j = 0$ (the weights still sum to 1).

We can write:
$$d\Psi = S^{\gamma-1} \left[ \sum_{j \in \mathcal{R}_r} \alpha_j^{(1-\gamma)/\gamma} d\alpha_j + \sum_{k \notin \mathcal{R}_r} \alpha_k^{(1-\gamma)/\gamma} d\alpha_k \right]$$

\textit{Method 2: Via implicit differentiation with respect to crisis intensity.}

Let the crisis intensity be parameterized by $\Delta_r$. Then:
$$\frac{d\Psi}{d\Delta_r} = S^{\gamma-1} \sum_j \alpha_j^{(1-\gamma)/\gamma} \frac{d\alpha_j}{d\Delta_r}$$

For $j \in \mathcal{R}_r$:
$$\frac{d\alpha_j}{d\Delta_r} = \frac{\phi'(s_j^{pre} + \Delta_r) \Phi - \phi(s_j^{pre} + \Delta_r) \cdot \sum_{\ell \in \mathcal{R}_r} \phi'(s_\ell^{pre} + \Delta_r)}{\Phi^2}$$

At $\Delta_r = 0$:
$$\left.\frac{d\alpha_j}{d\Delta_r}\right|_{\Delta_r=0} = \frac{\phi'(s_j^{pre})}{\Phi^{pre}} - \alpha_j^{pre} \cdot \frac{\sum_{\ell \in \mathcal{R}_r} \phi'(s_\ell^{pre})}{\Phi^{pre}}$$
$$= \frac{1}{\Phi^{pre}} \left[ \phi'(s_j^{pre}) - \alpha_j^{pre} \sum_{\ell \in \mathcal{R}_r} \phi'(s_\ell^{pre}) \right]$$

For $k \notin \mathcal{R}_r$:
$$\left.\frac{d\alpha_k}{d\Delta_r}\right|_{\Delta_r=0} = -\alpha_k^{pre} \cdot \frac{\sum_{\ell \in \mathcal{R}_r} \phi'(s_\ell^{pre})}{\Phi^{pre}} < 0$$

\textit{Combining to show $d\Psi/d\Delta_r > 0$:}

Since the crisis redistributes attention from unaffected to affected projects, and since affected projects (with higher attention weights after crisis) have higher values of $\alpha_j^{(1-\gamma)/\gamma}$ when $\gamma < 1$ (because $(1-\gamma)/\gamma > 0$ makes this an increasing function), the positive terms dominate and $d\Psi/d\Delta_r > 0$.

\textit{Method 3: Direct argument from Lemma \ref{lem:psi}.}

From Lemma \ref{lem:psi}(ii), $\Psi$ increases when attention becomes more concentrated. A crisis concentrates attention on affected projects (parts i and ii), moving the distribution away from uniformity. Therefore $\Psi^{post} > \Psi^{pre}$.
\end{proof}

\subsection{Proof of Theorem \ref{thm:intensive} (Intensive Margin Effects)}
\label{app:proof:intensive}

\begin{proof}
For an active donor with $D^{pre} > 0$ and $D^{post} > 0$:

\textbf{Part (i)}: From the allocation rule \eqref{eq:crra_allocation}:
$$d_j^{post} = \frac{(\alpha_j^{post})^{1/\gamma}}{\sum_k (\alpha_k^{post})^{1/\gamma}} \cdot D^{post}$$

By Proposition \ref{prop:attention_reallocation}(i), $\alpha_j^{post} > \alpha_j^{pre}$ for $j \in \mathcal{R}_r$. This increases both (a) the share of giving allocated to $j$ and (b) total giving $D$ (via increased $\Psi$). Both effects work in the same direction, so $d_j^{post} > d_j^{pre}$.

\textbf{Part (ii)}: For $k \notin \mathcal{R}_r$, $\alpha_k^{post} < \alpha_k^{pre}$ (Proposition \ref{prop:attention_reallocation}(ii)). The share allocated to $k$ decreases. While total giving increases, the share effect dominates for unaffected projects when the crisis is sufficiently large, yielding $d_k^{post} < d_k^{pre}$.

\textbf{Part (iii)}: By Proposition \ref{prop:total_giving}(iv), $\partial D^*/\partial \Psi > 0$. Since $\Psi^{post} > \Psi^{pre}$ (Proposition \ref{prop:attention_reallocation}(iii)), we have $D^{post} > D^{pre}$.

\textbf{Part (iv)}: Log-differentiating the FOC $u'(W-D) = \theta \Psi D^{-\gamma}$:

Taking logs: $\log u'(W-D) = \log \theta + \log \Psi - \gamma \log D$

Differentiating with respect to $\Psi$:
$$\frac{u''(W-D)}{u'(W-D)} \cdot (-dD) = \frac{d\Psi}{\Psi} - \gamma \frac{dD}{D}$$

Rearranging:
$$\frac{dD}{D} \left( \gamma + \frac{-u''(c) \cdot c}{u'(c)} \cdot \frac{D}{c} \right) = \frac{d\Psi}{\Psi}$$

Let $\eta_u = -cu''/u' > 0$ be the coefficient of relative risk aversion over consumption. Then:
$$\frac{dD}{D} = \frac{d\Psi/\Psi}{\gamma + \eta_u \cdot D/c}$$

Since $D/c < 1$ (giving is less than consumption), we have:
$$\frac{dD}{D} < \frac{d\Psi/\Psi}{\gamma}$$

which gives the bound stated.
\end{proof}

\subsection{Proof of Theorem \ref{thm:extensive} (Extensive Margin Effects)}
\label{app:proof:extensive}

\begin{proof}
Under Assumption \ref{ass:bounded}, the participation threshold is:
$$\bar{\theta}(W, \boldsymbol{\alpha}) = \frac{u'(W)}{\Psi(\boldsymbol{\alpha}) \cdot \bar{v}}$$

\textbf{Part (i)}: Since $\Psi^{post} > \Psi^{pre}$ (Proposition \ref{prop:attention_reallocation}(iii)):
$$\bar{\theta}^{post}(W) = \frac{u'(W)}{\Psi^{post} \cdot \bar{v}} < \frac{u'(W)}{\Psi^{pre} \cdot \bar{v}} = \bar{\theta}^{pre}(W)$$

\textbf{Part (ii)}: The participation rate is:
$$\pi = \int_0^\infty [1 - \Phi_\theta(\bar{\theta}(W))] d\Phi_W(W)$$

Since $\bar{\theta}^{post}(W) < \bar{\theta}^{pre}(W)$ for all $W$, and $\Phi_\theta$ is increasing:
$$1 - \Phi_\theta(\bar{\theta}^{post}(W)) > 1 - \Phi_\theta(\bar{\theta}^{pre}(W))$$

Integrating over $W$: $\pi^{post} > \pi^{pre}$.

\textbf{Part (iii)}: The number of newly activated donors is those with $\theta \in (\bar{\theta}^{post}(W), \bar{\theta}^{pre}(W))$:
$$N^{new} = \int_0^\infty [\Phi_\theta(\bar{\theta}^{pre}(W)) - \Phi_\theta(\bar{\theta}^{post}(W))] d\Phi_W(W)$$
\end{proof}

\subsection{Proof of Theorem \ref{thm:decomposition} (Crisis Effect Decomposition)}
\label{app:proof:decomposition}

\begin{proof}
The total change in funding to affected projects is:
$$\Delta^{total}_r = \sum_{j \in \mathcal{R}_r} \left( d_j^{post} - d_j^{pre} \right)$$

where $d_j = \int_i d_{ij} \, di$ is aggregate donations to project $j$.

We decompose donors into three groups: (1) donors active both pre and post crisis ($\mathcal{A}^{pre} \cap \mathcal{A}^{post}$), (2) newly activated donors ($\mathcal{A}^{post} \setminus \mathcal{A}^{pre}$), and (3) donors who became inactive (which, by our model, is empty since $\Psi$ increased).

\textbf{Existing donors' contributions} (group 1):

For $i \in \mathcal{A}^{pre}$, the change in giving to affected projects is:
$$\sum_{j \in \mathcal{R}_r} (d_{ij}^{post} - d_{ij}^{pre})$$

This can be decomposed into:
\begin{enumerate}
\item \textit{Reallocation (substitution)}: Holding total giving fixed at $D_i^{pre}$, the change due to attention reallocation:
$$\sum_{j \in \mathcal{R}_r} \left[ \frac{(\alpha_j^{post})^{1/\gamma}}{\sum_k (\alpha_k^{post})^{1/\gamma}} - \frac{(\alpha_j^{pre})^{1/\gamma}}{\sum_k (\alpha_k^{pre})^{1/\gamma}} \right] D_i^{pre}$$

\item \textit{Intensive margin expansion}: The additional giving by existing donors allocated to affected projects:
$$(D_i^{post} - D_i^{pre}) \cdot \frac{(\alpha_r^{post})^{1/\gamma}}{\sum_k (\alpha_k^{post})^{1/\gamma}}$$
\end{enumerate}

\textbf{New donors' contributions} (group 2):

Donors $i \in \mathcal{A}^{post} \setminus \mathcal{A}^{pre}$ contribute:
$$\sum_{j \in \mathcal{R}_r} d_{ij}^{post}$$

to affected projects. This is the additionality effect to region $r$.

\textbf{Aggregating}:

Summing over all donors:

$$\Delta^{total}_r = \underbrace{\int_{i \in \mathcal{A}^{pre}} \sum_{j \in \mathcal{R}_r} (d_{ij}^{post} - d_{ij}^{pre}|_{D = D^{pre}}) \, di}_{\Delta^{sub}} + \underbrace{\int_{i \in \mathcal{A}^{post} \setminus \mathcal{A}^{pre}} \sum_{j \in \mathcal{R}_r} d_{ij}^{post} \, di}_{\Delta^{add,r}}$$
$$+ \underbrace{\int_{i \in \mathcal{A}^{pre}} (D_i^{post} - D_i^{pre}) \cdot \frac{(\alpha_r^{post})^{1/\gamma}}{\sum_k (\alpha_k^{post})^{1/\gamma}} \, di}_{\Delta^{int}}$$

Note that $\Delta^{sub}$ represents reallocation and sums to zero across all projects (it's a pure transfer from unaffected to affected projects). The net addition to total platform funding comes from $\Delta^{add}$ (new donors) and $\Delta^{int}$ (expanded giving by existing donors).

The expressions in the theorem statement follow by noting that $\frac{(\alpha_r^{post})^{1/\gamma}}{\sum_k (\alpha_k^{post})^{1/\gamma}}$ represents the share of giving allocated to affected projects under post-crisis attention weights.
\end{proof}

\subsection{Proof of Proposition \ref{prop:narrative} (Narrative Effects)}
\label{app:proof:narrative}

\begin{proof}
Let $x_{jk}$ be a characteristic of project $j$ that affects its salience signal, with $\partial s_j / \partial x_{jk} > 0$.

The expected donation to project $j$ is:
$$\mathbb{E}[d_j] = \int_i d_{ij}^* \, di = \int_i \alpha_j^{1/\gamma} \cdot \frac{D_i^*}{\sum_\ell \alpha_\ell^{1/\gamma}} \, di$$

To find the effect of $x_{jk}$, we apply the chain rule:
$$\frac{\partial \mathbb{E}[d_j]}{\partial x_{jk}} = \frac{\partial \mathbb{E}[d_j]}{\partial s_j} \cdot \frac{\partial s_j}{\partial x_{jk}}$$

Since $\partial s_j / \partial x_{jk} > 0$ by assumption, we need to show $\partial \mathbb{E}[d_j] / \partial s_j > 0$.

\textbf{Step 1: Effect of salience on attention}

From the attention formation equation \eqref{eq:attention_formation}:
$$\alpha_j = \frac{\phi(s_j)}{\sum_k \phi(s_k)}$$

Differentiating:
$$\frac{\partial \alpha_j}{\partial s_j} = \frac{\phi'(s_j) \sum_k \phi(s_k) - \phi(s_j) \phi'(s_j)}{(\sum_k \phi(s_k))^2} = \frac{\phi'(s_j)}{\sum_k \phi(s_k)} (1 - \alpha_j) > 0$$

since $\phi' > 0$ (monotonicity) and $\alpha_j < 1$.

\textbf{Step 2: Effect of attention on project-level donations}

From Lemma \ref{lem:allocation}, for each donor $i$:
$$d_{ij}^* = \alpha_j^{1/\gamma} \cdot \frac{D_i^*}{\sum_\ell \alpha_\ell^{1/\gamma}}$$

Differentiating with respect to $\alpha_j$:
$$\frac{\partial d_{ij}^*}{\partial \alpha_j} = \frac{1}{\gamma} \alpha_j^{1/\gamma - 1} \cdot \frac{D_i^*}{\sum_\ell \alpha_\ell^{1/\gamma}} - \alpha_j^{1/\gamma} \cdot \frac{D_i^* \cdot \frac{1}{\gamma} \alpha_j^{1/\gamma - 1}}{(\sum_\ell \alpha_\ell^{1/\gamma})^2}$$

$$= \frac{D_i^*}{\gamma \alpha_j S} \left( 1 - \frac{\alpha_j^{1/\gamma}}{S} \right) = \frac{D_i^*}{\gamma \alpha_j S} \left( 1 - \frac{d_{ij}^*}{D_i^*} \right) > 0$$

where $S = \sum_\ell \alpha_\ell^{1/\gamma}$. This is positive since each project receives less than the total giving.

Additionally, increased $\alpha_j$ increases $\Psi$ (by Lemma \ref{lem:psi}), which increases $D_i^*$ (by Proposition \ref{prop:total_giving}(iv)), further increasing $d_{ij}^*$.

\textbf{Step 3: Aggregating across donors}

Since $\partial d_{ij}^* / \partial \alpha_j > 0$ for each active donor and $\partial \alpha_j / \partial s_j > 0$:
$$\frac{\partial \mathbb{E}[d_j]}{\partial s_j} = \int_i \frac{\partial d_{ij}^*}{\partial s_j} \, di = \int_i \frac{\partial d_{ij}^*}{\partial \alpha_j} \cdot \frac{\partial \alpha_j}{\partial s_j} \, di > 0$$

Therefore:
$$\frac{\partial s_j}{\partial x_{jk}} > 0 \quad \Rightarrow \quad \frac{\partial \mathbb{E}[d_j]}{\partial x_{jk}} = \frac{\partial \mathbb{E}[d_j]}{\partial s_j} \cdot \frac{\partial s_j}{\partial x_{jk}} > 0$$
\end{proof}

\section{Data Appendix}
\label{app:data}

\subsection{Data Source}

Data were collected from the GlobalGiving API (\texttt{https://www.globalgiving.org/api/}) in December 2024. The API provides access to all publicly listed projects on the platform, including project descriptions, funding information, organizational details, and geographic classifications. We collected the complete universe of projects rather than a sample, ensuring coverage of the full platform history from founding (2002) through the collection date.

\subsection{Data Quality and Limitations}

Several data quality issues merit discussion:

\begin{enumerate}
\item \textbf{Missing dates}: Approximately 0.5\% of projects have missing or malformed approval dates. These are excluded from time-series analysis but included in cross-sectional specifications where temporal information is not required.

\item \textbf{Currency}: All funding amounts are reported in USD. For organizations operating in other currencies, GlobalGiving converts to USD at prevailing exchange rates. We cannot distinguish conversion-related variation from true funding changes.

\item \textbf{Goal changes}: Organizations can modify funding goals during a project's lifetime. We observe the goal as of the data collection date, which may differ from the original goal at project launch. This could introduce measurement error in goal-related analyses.

\item \textbf{Text fields}: Project summaries vary substantially in length and quality. Approximately 3\% of projects have missing or very short ($<$ 50 characters) summaries. Keyword analysis excludes these observations.

\item \textbf{Platform changes}: GlobalGiving has evolved substantially since 2002, with changes to fee structures, vetting processes, and interface design. We cannot observe or control for these changes directly, though year fixed effects partially address time-varying platform characteristics.
\end{enumerate}

\subsection{Keyword Extraction Methodology}

Keyword indicators are constructed using regular expression matching on the lowercased project summary field. The matching is case-insensitive and uses word boundaries to avoid partial matches (e.g., ``urgent'' matches ``urgent'' but not ``urgently'' unless explicitly included).

Specific patterns:
\begin{itemize}
\item \texttt{has\_children}: \texttt{/\textbackslash b(children|child|kids|youth|young)\textbackslash b/i}
\item \texttt{has\_urgent}: \texttt{/\textbackslash b(urgent|emergency|immediate|critical)\textbackslash b/i}
\item \texttt{has\_lives}: \texttt{/(save lives|saving lives|life-saving|lifesaving)/i}
\item \texttt{has\_women}: \texttt{/\textbackslash b(women|girls|female|mothers|woman|girl)\textbackslash b/i}
\item \texttt{has\_food}: \texttt{/\textbackslash b(food|hunger|nutrition|meals|hungry|malnutrition)\textbackslash b/i}
\item \texttt{has\_water}: \texttt{/\textbackslash b(water|sanitation|wash|clean water|safe water)\textbackslash b/i}
\end{itemize}

\subsection{Ukraine Classification}

A project is classified as Ukraine-related if any of the following conditions hold:
\begin{enumerate}
\item The \texttt{country} field contains ``Ukraine''
\item The project \texttt{title} contains ``Ukraine'' or ``Ukrainian'' (case-insensitive)
\item The project \texttt{summary} contains ``Ukraine'' or ``Ukrainian'' (case-insensitive)
\end{enumerate}

This broad definition captures both projects explicitly located in Ukraine and projects elsewhere that address Ukraine-related needs (e.g., refugee support in neighboring countries). Of 892 Ukraine-related projects in our DiD sample, 678 (76\%) are classified via the country field and 214 (24\%) via keyword matching alone.

\end{document}
